\documentclass{article}
\def\version{1.8}
%
% This work may be distributed and/or modified under the
% conditions of the LaTeX Project Public License, either version 1.3
% of this license or (at your option) any later version.
% The latest version of this license is in
%   http://www.latex-project.org/lppl.txt
% and version 1.3c or later is part of all distributions of LaTeX
% version 2005/12/01 or later.
%
% This work has the LPPL maintenance status `maintained'.
%
% The Current Maintainer of this work is Steven B. Segletes.

% V1.01 Correct typos in documentation
% V1.1  Corrected usepackage dependencies; Significant doc rewrite.
% V1.11 Corrected typo in documentation
% V1.2  Added \scaleleftright and \stretchleftright; revised docs.
% V1.3  Corrected missing % in stretch routines, added \hstretch and
%       \vstretch commands; fixed version number in .sty file
% V1.4 -Auto-detect mathmode and style; use the proper math display
%       style; Allow \ignoremathstyle and \discernmathstyle to
%       revert to former approach in rare cases and return to new
%       approach.
%      -Added \Isnextbyte as a supplemental routine
%      -Added \scaleobj (like \scalebox, but defaults to math mode
%       using current display style).
% V1.5 -Eliminated mathstyle package because of incompatibilites with
%       many other packages.  Implemented \mathchoice approach instead.
% V1.6 -Made available \LMex and \LMpt lengths (Local Math ex and
%       Local Math pt), usable inside \ThisStyle arguments (1ex/pt in
%       \textstyle and \displaystyle, 0.7ex/pt in \scriptstyle, and
%       0.5ex/pt in \scriptscriptstyle).
%      -Revised/shortened/improved \Isnextbyte.
%      -Replaced ifthen package calls with etoolbox calls.
% V1.7 -Note that the V1.5 implementation with \mathchoice seemed to
%       have made the use of \ignoremathstyle and \discernmathstyle 
%       vestigial.  Thus, \ignoremathstyle was revised to be used
%       for the purpose of streamlining package efficiency when 
%       only textstyle (or displaystyle with \ignoremathstyle[D])
%       processing is required.
%      -Fixed use of dot {.} as null argument in \scaleleftright{}{}{} 
%       and \stretchleftright{}{}{}, which had been causing overflow.
%      -Made \hstretch and \vstretch routines more efficient, using
%       optional arguments of \scalebox, instead of \scalerel.
% V1.8 -Fixed divide by 0 bugs that would arise when object to be
%       scaled fell completely below the baseline (zero height).
%       Required revisions to \scalerelplain, \stretchrelplain,
%       \scaleto, and \stretchto.
%      -Fixed aspect-ratio bug, that calculated aspect for stretched object
%       based only upon portion above baseline
\parskip 1em
\parindent 0em
\newcommand\rl{\rule{2em}{0in}}
\usepackage{scalerel}
\usepackage{verbatim}
\usepackage[only,llbracket,rrbracket,llparenthesis,rrparenthesis]{stmaryrd}
\usepackage{stackengine}
\usepackage{accsupp} % for ensuring the right Unicode codepoint upon pasting
\def\sr{\textsf{scalerel}}
\let\vb\verb
\reversemarginpar
\marginparwidth 1.6in
\newcommand\margtt[1]{\marginpar{\hfill\ttfamily#1}}
\newcommand\margcmd[1]{\marginpar{\hfill\ttfamily\char'134#1}}
\newcommand\cmd[1]{\texttt{\char'134#1}}
\begin{document}
\begin{center}
\LARGE The \textsf{scalerel} Package\\
\rule{0em}{.7em}\small Routines for constrained scaling and stretching
of objects,\\relative to a reference object or in absolute terms\\
\rule{0em}{2.7em}\large Steven B. Segletes\\
steven.b.segletes.civ@mail.mil\\
\rule{0em}{1.7em}\today\\
v\version
\end{center}

\section{Commands and Description}

The \textsf{scalerel} package is used to scale and vertically stretch
  objects, either relative to other objects, or in absolute terms.  
Its commands may be invoked in either math mode or text mode (if there 
  is no math in the objects to manipulate); 
however, the objects it manipulates will, by default, be processed in 
  math mode.  
Nonetheless, one may process the objects in text mode (assuming they 
  do not contain math-specific characters or commands) by 
  [non-intuitively] specifying arguments as \vb|{$object$}|, instead
  of the usual \vb|{object}|.

There are four basic commands with the \textsf{scalerel} package, two
of which have star variants:
\begin{verbatim}
  \scalerel[*][max_width]{object}{reference}
  \stretchrel[*][narrowest_aspect]{object}{reference}
  \scaleto[max_width]{object}{height}
  \stretchto[narrowest_aspect]{object}{height}
\end{verbatim} 
In addition, there are a number of added commands that have been
implemented as composites of the four basic commands, they are
\begin{verbatim}
  \scaleleftright[max_width]{left-obj.}{reference}{right-obj.}
  \stretchleftright[narrowest_aspect]{left-obj.}{reference}{right-obj.}
  \hstretch{scale}{object}
  \vstretch{scale}{object}
  \scaleobj{scale}{object}
\end{verbatim}
Unless delimited for text mode, \texttt{object} is assumed to be a math
  object and will be processed in the \textit{current} math style.%
\footnote{By \LaTeX{} math styles, we mean \cmd{displaystyle}, \cmd{textstyle},
  \cmd{scriptstyle}, and \cmd{scriptscriptstyle}.}
On the other hand, \texttt{reference} will take on the current mode (text
  or math).  
If in math mode, \texttt{reference} will likewise be taken in the current
  math style.

There are also several auxiliary commands, useful in their own
right:
\begin{verbatim}
  \ThisStyle{...\SavedStyle...}
  \LMex
  \LMpt
  \def\scriptstyleScaleFactor{0.7}
  \def\scriptscriptstyleScaleFactor{0.5}
  \Isnextbyte[[v] OR q]{test byte}{string}
  \ignoremathstyle[D]
  \discernmathstyle
\end{verbatim}

In the case of all \sr{} macros, \vb|object| is the object to be scaled or 
stretched.  It can be as simple as a symbol, like a summation sign
($\sum$), or it can be an object of complex description.  Likewise, in
cases of \textit{relative} scaling and stretching, \vb|reference| is
the reference object in relation to which the manipulated
\vb|object| is scaled or vertically stretched.  The \vb|reference|
may likewise be of complex description (such as a mathematical
expression).  When an \vb|object| is scaled or stretched
\textit{relative} to a \vb|reference| object, it is also vertically
shifted, if necessary, so that its vertical extent conforms to that of
the \vb|reference| object.

In cases of scaling or stretching \textit{to} a specified size,
  \vb|height| will be the \textit{total} vertical height${}+{}$depth of 
  the \vb|object| following a scale or stretch manipulation.  
Since scaling or stretching to an
  absolute size provides no \vb|reference| object, the baseline of
  the manipulated \vb|object| remains unchanged.

In cases of scaling (relative or absolute), the constraint
  \vb|max_width| is optionally specified as the maximum width allowable
  for the manipulated object.  
If the manipulated width would otherwise exceed this limit, the
  \vb|object| width is reduced to meet this limit.  
If the \vb|object| width is reduced, its aspect ratio will change with 
  respect to its original shape.

By definition, vertical stretching will increase the (height-to-width) 
  aspect ratio
  of an object.  
In the case of stretching (relative or absolute), the constraint
  \vb|narrowest_aspect| is the narrowest aspect ratio allowed by the
  stretch.%
\footnote{Prior to V1.8 of the package, the limiting aspect ratio was
  erroneously based on that portion of the stretched object 
  \textit{above} the baseline, rather than the object's full vertical extent.}  
Its value is given in \%, such that a parameter value of \texttt{100}
  indicates 100\% or an aspect ratio of 1.  
If the stretch would otherwise cause the manipulated \vb|object|'s 
  aspect ratio to become more narrow than (\textit{i.e.}, larger 
  than) this value, the width of the manipulated \vb|object| is 
  increased to meet this threshold.  
The value of \vb|narrowest_aspect| must be an integer.

Because \textit{relative} scaling or stretching is done
\textit{relative} to a reference object, the \vb|\scalerel| and
\margcmd{scalerel}\margcmd{stretchrel}%
\vb|\stretchrel| commands will, by default, print out the
manipulated \vb|object| followed immediately by the \vb|reference|
object.  Because this may not always be desired, the star (*) version of
these commands suppresses the output of the \vb|reference| object, so
that only the \vb|object| that was manipulated is output.

The \vb|\scaleleftright| and \vb|\stretchleftright| commands
\margcmd{scaleleftright}\margcmd{stretchleftright}%
are intended to provide the functional capability of the \vb|\left| and
\vb|\right| commands in math mode, for delimiting symbols not
otherwise supported by those commands.

The \vb|\hstretch| and \vb|\vstretch commands| merely take the
\margcmd{hstretch}\margcmd{vstretch}%
second argument and either horizontally or vertically stretch it by a
scale factor given by the first argument.  The scale is not given in
percent, like some aforementioned commands.  Remember also, that these
commands are, by default processed in math mode.  In a like manner,
\margcmd{scaleobj}%
\vb|\scaleobj| performs a scale of the object.  It is like the
\vb|\scalebox| command of the \textsf{graphicx} package, except that
its argument (like others in this package) defaults to math mode, and is
displayed in the current math display style.

A useful auxiliary command \vb|\ThisStyle{}| has been introduced for
\margcmd{ThisStyle}%
remembering the current math style.  When invoked, this command will
remember the math style present at the time of its invocation and then
execute its argument.  Within the extent of its argument, an invocation
\margcmd{SavedStyle}%
of the macro \vb|\SavedStyle| will reinstate the math style that was
active upon invocation.  This command is very useful when \LaTeX{} boxes
are being set within the math environment, since the prevailing math
style does not carry into the box, without the use of
\vb|\SavedStyle|.

The lengths \verb|\LMex| and \verb|\LMpt| work in conjunction with 
\verb|\ThisStyle|
\margcmd{LMex}%
\margcmd{LMpt}%
and are what I am calling the ``local-mathstyle `ex'\,''
and ``local-mathstyle `pt','' respectively.
They are lengths, available for use within the argument of \verb|\ThisStyle|,
 that scale with the current mathstyle, equalling 1ex and 1pt in
\verb|\displaystyle| and \verb|\textstyle| math, and scaled by a 
factor of 0.7 in \verb|\scriptstyle| and 0.5 in \verb|\scriptscriptstyle|.
Such a scalable length is useful when {\sr} is used to place objects 
\margcmd{scriptstyleScaleFactor}%
relative to each other across various mathstyles.  The 0.7 and 0.5
relative height factors apply to the default Computer Modern font.
\margcmd{\footnotesize scriptscriptstyleScaleFactor}%
For other fonts, these scale factors may be reset with a \verb|\def|
on the macros \verb|\scriptstyleScaleFactor| and 
\verb|\scriptscriptstyleScaleFactor|.

A service routine \vb|\Isnextbyte| was employed by this package.
\margcmd{Isnextbyte}%
Because of its more general utility, it is made accessible to the user.
It is an improved version of the command \vb|\isnextbyte| from the
\textsf{stringstrings} package.  It will determine if the first
byte from a passed string argument matches a specified
\textit{match}-character.  It produces a \vb|T| or \vb|F| result,
which is printed out by default.  With the use of the \vb|q| optional
argument, the output can be suppressed, with the result instead being
stored in a macro \vb|\theresult|.

\subsection{Efficiency of nested \sr{} macros\label{s:nm}}

As of version 1.4, {\sr} commands will autodetect the current math
display style (so-called display-style, text-style, script-style, and/or
scriptscript-style) and process its arguments in the same mode.  The
overhead associated with this capability is an invocation of 
\vb|\mathchoice|, which must create 4 boxes (one in each math style) 
before selecting one
to typeset.  While a small overhead in itself, an efficiency issue
can arise when \sr{} macros are nested, since if 3 \sr{} macros
are nested, \vb|\mathchoice| will build $4^3=64$ boxes in order
to select the proper one to typeset (and a nesting level of 4 yields
256 boxes).

With the use of \vb|\ignoremathstyle|, the math style of the \sr{}
\margcmd{ignoremathstyle}%
processing can be fixed in a single math style, thus eliminating the
construction of \vb|\mathchoice| boxes.  The increase in processing
efficiency can be noticeable when operating on deeply nested \sr{}
macros.

The invocation of \vb|\ignoremathstyle| will force all subsequent 
\sr{} calls to process its arguments in \vb|\textstyle| math.  
Alternately, \vb|\ignoremathstyle[D]| will force the processing
of \sr{} macros in \vb|\displaystyle| math.

\begin{sloppypar}
This streamlining can be undone, reinstating the more general
math-style preserving approach, with the invocation of
\vb|\discernmathstyle|.
\margcmd{discernmathstyle}%
These macros can be used anywhere in a document by a user, to turn off 
and on the \vb|\mathchoice| feature of \sr{}, or they may be
used within user macro definitions, to fix the math style by which
\sr{} processes its macros within the confines of the user macro.
\end{sloppypar}

\section{Usage Examples}

Now for a few examples.  Let us define
\begin{verbatim}
    \def\preblob{\displaystyle\sum_{i=0}^3}
    \def\blob{\displaystyle\frac{\displaystyle\frac{x^3}{z+r^3}}%
              {\displaystyle\frac{y}{x^2}}%
    }
\end{verbatim}
    \def\preblob{\displaystyle\sum_{i=0}^3}
    \def\blob{\displaystyle\frac{\displaystyle\frac{x^3}{z+r^3}}%
              {\displaystyle\frac{y}{x^2}}%
    }
Here are the raw definitions of \vb|$\preblob\blob$|, unscaled:

    \rl$\preblob\blob$

\subsection{The $\backslash$\texttt{scalerel} Command}

Now we employ \vb|$\scalerel{\preblob}{\blob}$| to scale the summation 
  to match the size of the acted-upon term:

    \rl$\scalerel{\preblob}{\blob}$

If we wish constrain the width of the summation to 3ex, we employ\\
\vb|$\scalerel[3ex]{\preblob}{\blob}$|

    \rl$\scalerel[3ex]{\preblob}{\blob}$

Of course, if the manipulated object contains text symbols (as in
  this case), a width constraint will change their aspect ratio, 
  which is generally not desirable.

Now let's say you wanted to introduce notation to bound mathematical
expressions by triangles. After defining
\vb|\blob| as before, you could use\\
\vb|$\scalerel[3ex]{\triangleleft}{\blob}|\\
\vb|\scalerel*[3ex]{\triangleright}{\blob}$|

\rl%
$\scalerel[3ex]{\triangleleft}{\blob}\scalerel*[3ex]{\triangleright}{\blob}$

Here, the second call to \vb|\scalerel| was with the star (\vb|*|)
option, indicating that \vb|\blob| should not be printed out following
the right-hand delimiter.  A less-tall expression would appear in those
same delimiters as

\rl%
$\scalerel[3ex]{\triangleleft}{Q}\scalerel*[3ex]{\triangleright}{Q}$

Because the width limit had not been reached, no horizontal compression
of the object was required.  Note: see the \vb|\scaleleftright| and
\vb|\stretchleftright| commands for a more direct way to employ these
sort of transformations.

\subsection{The $\backslash$\texttt{stretchrel} Command}

In its most simple application \vb|\stretchrel| can be used in the
fashion of a \vb|\left\{| in math mode.  While there is no reason to
replace that more efficient usage supplied by \LaTeX{}, it is
nonetheless instructional to see a comparison of 
\vb|$\left\{\blob\right.$| and \vb|$\stretchrel[550]{\{}{\blob}$|, to
see how the aspect-ratio limiting option can be employed to avoid an
overly stretched (narrow) manipulation:

\rl$\left\{\blob\right.$  ~~~~~ $\stretchrel[550]{\{}{\blob}$

The expression on the right uses a standard \{ character, which has 
  been vertically stretched, but limited to an aspect ratio no 
  narrower than 5.5.  
To use symbols for which the \vb|\left| nomenclature will not work,
  \vb|\stretchrel| provides a viable alternative, as shown in this
  stick-figure example:

\vb|3m $\stretchrel[600]{\updownarrow\,}|\\
\vb|{\fbox{\rule[-1.8em]{0ex}{4em}Work Table}}$|

\rl3m $\stretchrel[600]{\updownarrow\,}
  {\fbox{\rule[-1.8em]{0ex}{4em}Work Table}}$

or in this use, \vb|$\stretchrel[350]{\int}{\blob} dx$|, of the
stretched integral sign (aspect limited to 3.5):

\rl$\stretchrel[350]{\int}{\blob} dx$


\subsection{Baseline Shifts of \textit{Relative} Scales and Stretches}

It was mentioned that when \vb|\scalerel| and \vb|\stretchrel| are
employed, the manipulated object gets vertically shifted to match the
extent of the reference object.  To see how this works, we provide the
following example, that employs some inline math and scaling.  

\rl\underline{The baseline is the line that runs under this text}
\fbox{$\scalerel{i}{\blob}$}\underline{\scaleto{i}{35pt}}

The first case of the large letter ``i'' was generated with 
  \vb|\scalerel|, and so the baseline of the ``i'', normally not a
  descending letter, was dropped to conform to the descended reference
  equation.  
All of this was set in an \vb|\fbox| to show the extent of
  the objects.  
For comparison, an ``i'', scaled to 35pt, was then placed
  using a \vb|\scaleto| command, which does not change the baseline of
  the original object.

\subsection{The $\backslash$\texttt{scaleto} and
$\backslash$\texttt{stretchto} Commands}

The \vb|\scaleto| and \vb|\stretchto| commands are comparable to the
  \vb|\scalerel| and \vb|\stretchrel| commands, except they do not
  accept a reference object as their second mandatory argument.  
Rather, they take an absolute height (\textit{i.e.}, height${}+{}$depth),
  specified with units.  
These commands use the same optional arguments as \vb|\scalerel| and
  \vb|\stretchrel| to constrain the width and/or the aspect ratio, 
  respectively, of the manipulated object.

As was mentioned in the prior section, a difference between these and
  the \textit{relative} scale/stretch commands is that an object's 
  baseline is not altered with the use of \vb|\scaleto| and 
  \vb|\stretchto|.  
Because there is no reference object employed with these commands, 
  there is no need for starred (\vb|*|) versions of these commands,
  which would otherwise suppress the printing of the reference object.

Examples follow the established pattern established in prior sections.
First, we scale the equation blob used as an example in this
documentation to a vertical extent of 80pt, preserving the original
aspect ratio,

\rl\vb|$\scaleto{\blob}{80pt}$      | $\scaleto{\blob}{80pt}$

Below, we stretch the Greek letter, capital phi, to 60pt, but constrain
the aspect ratio to no less than 1.75.

\rl\vb|$\stretchto[175]{\Phi}{60pt}$      | $\stretchto[175]{\Phi}{80pt}$

\subsection{The $\backslash$\texttt{scaleleftright} and
 $\backslash$\texttt{stretchleftright}\\Commands}

These commands are composites of several \vb|\scalerel| and
\vb|\stretchrel| commands, intended to provide the functionality of
the \vb|\left ...\right| syntax of mathmode, using symbols for which
the \vb|\left ...\right| syntax does not work.  It uses the same
optional arguments as \vb|\scalerel| and \vb|\stretchrel| in order
to limit the width or aspect ratio of the respective manipulated
objects.  If one of the symbols on the left or right end of the enclosed
reference block is to be blank, these commands support the use of the
period argument \vb|{.}| as a blank placeholder, in a fashion
analogous to the \vb|\left ...\right| syntax of mathmode.

\rl\vb|$\scaleleftright[3ex]{\prod}{\blob}{\coprod}$  |
$\scaleleftright[3ex]{\prod}{\blob}{\coprod}$

\rl\vb|$\stretchleftright[450]{.}{\blob}{\in}$        |
$\stretchleftright[450]{.}{\blob}{\in}$

\subsection{The $\backslash$\texttt{hstretch},
 $\backslash$\texttt{vstretch}, and $\backslash$\texttt{scaleobj}
 Commands}

These commands simply perform horizontal or vertical stretching,
respectively.  They use a scale factor, rather than an absolute length.
If the stretch factor is greater than 1, the stretched length is larger
than the original.  If less than one, it is compressed with respect to
the original.

\rl\vb|$\otimes\hstretch{3}{\otimes}\hstretch{0.5}{\otimes}$ |
$\otimes\hstretch{3}{\otimes}\hstretch{0.5}{\otimes}$

\rl\vb|$\otimes\vstretch{3}{\otimes}\vstretch{0.5}{\otimes}$ |
$\otimes\vstretch{3}{\otimes}\vstretch{0.5}{\otimes}$

The \vb|\scaleobj| command performs scaling in the fashion of a
\vb|\scalebox| command, except that its argument is processed in math
mode of the current math display style, by default.  Thus, the
difference between the following two invocations:
\begin{verbatim}
  \(\scaleobj{2}{\sum_{i=0}^{n}}\)
  \[\scaleobj{2}{\sum_{i=0}^{n}}\]
\end{verbatim}
is given as
\(\scaleobj{2}{\sum_{i=0}^{n}}\) versus
\[\scaleobj{2}{\sum_{i=0}^{n}}\]

\subsection{The \textbackslash\texttt{ThisStyle}, \textbackslash\texttt{SavedStyle},
\textbackslash\texttt{LMex}, and \textbackslash\texttt{LMpt} Commands}

These are very powerful macros that provide the ability to save the current
math style and to import it to various places (for example, inside boxes and other
macros) where the math style would otherwise be lost.  These commands are
part of this package because the various {\sr} macros employ them to
automatically import the current math style into their arguments.

The easiest way to show how these macros are used is to provide a working 
example, such as the one given below.

\begin{verbatim}
\documentclass{article}
\usepackage{stackengine}
\usepackage{graphicx}
\usepackage{scalerel}
\stackMath
\def\ccdot{\scalebox{1.15}{$\SavedStyle\cdot$}}
\def\altdiv{\mathbin{\ThisStyle{%
  \stackunder[-.6\LMex]{%
    \stackon[-.45\LMex]{\SavedStyle\sim}{\ccdot}%
  }{\ccdot}}}}
\begin{document}
$A \altdiv B ~~A \div B$\par
$\scriptstyle A \altdiv B ~~A \div B$\par
$\scriptscriptstyle A \altdiv B ~~A \div B$\par
\end{document}
\end{verbatim}

The output of this example shows a fabricated glyph (\verb|\altdiv|) across the 
math styles, comparing it to \verb|\div|:

\stackMath
\def\ccdot{\scalebox{1.15}{$\SavedStyle\cdot$}}
\def\altdiv{\mathbin{\ThisStyle{%
  \stackunder[-.6\LMex]{\stackon[-.45\LMex]{\SavedStyle\sim}{\ccdot}}{\ccdot}}}}
\rl\scalebox{3}{\parbox{2in}{
$A \altdiv B ~~A \div B$\par
$\scriptstyle A \altdiv B ~~A \div B$\par
$\scriptscriptstyle A \altdiv B ~~A \div B$\par
}}

The difficulties that must be overcome are several:  the composite glyph is created by
way of a stack, which processes its arguments in either text mode or \verb|\textstyle|
math mode; and the stacking distances between the components of the composite
glyph need to vary with the mathstyle.  To overcome the first issue, the \verb|\altdiv|
macro is wrapped in a \verb|\ThisStyle{...}| wrapper.  Whenever one of the glyph
components (\textit{e.g.}, \verb|\sim| or \verb|\cdot|) is employed within the 
\textsf{stackengine} where the mathstyle is
otherwise lost, it is prepended with a \verb|\SavedStyle| to reinstate the current
math style.  Thus, the glyph components will present in the proper mathstyle.  
Likewise, the vertical stacking gaps between the glyph components, 
given as optional arguments to the stacking commands, are here specified in 
multiples of \verb|\LMex|.  In this way, the stacking gaps will shrink with the 
smaller math styles, keeping the spacings on the same relative scale as the size
of the smaller glyphs.

\section{Real World Application}

\subsection{White Curly Brackets}

Ref: http://tex.stackexchange.com/questions/100966/defining-
scalable-white- curly-brackets-and-and

\newcommand*{\llbrace}{%
  \BeginAccSupp{method=hex,unicode,ActualText=2983}%
    \textnormal{\usefont{OMS}{lmr}{m}{n}\char102}%
    \mathchoice{\mkern-4.05mu}{\mkern-4.05mu}{\mkern-4.3mu}{\mkern-4.8mu}%
    \textnormal{\usefont{OMS}{lmr}{m}{n}\char106}%
  \EndAccSupp{}%
}
\newcommand*{\rrbrace}{%
  \BeginAccSupp{method=hex,unicode,ActualText=2984}%
    \textnormal{\usefont{OMS}{lmr}{m}{n}\char106}%
    \mathchoice{\mkern-4.05mu}{\mkern-4.05mu}{\mkern-4.3mu}{\mkern-4.8mu}%
    \textnormal{\usefont{OMS}{lmr}{m}{n}\char103}%
  \EndAccSupp{}%
}

A user defined two new symbols made by combining the font glyphs of a
brace and a vertical strut, calling them \vb|\llbrace| and
\vb|\rrbrace|, to look as follows around an argument (in this case
$x$):

\def\wbrace{\scalebox{5}{$\llbrace x\rrbrace$}}
\rl\wbrace

But they found that when represented in reduced size fonts, the symbol
glyphs of the brace and the strut separated (note: \vb|\tiny| font has
been scaled up to match prior figure's height):

\def\tinywbrace{\scalebox{3}{\tiny$\llbrace x\rrbrace$}}
\rl$\scalerel*{\tinywbrace}{\wbrace}$

This happens because font designers don't merely scale their fonts when
shrunk down in size, but they do things (like thickening up lines and
changing aspect ratios) in an attempt to retain legibility.  In this
case, the font's auto-adjustment at smaller scale caused this composite
representation to change in unacceptable ways.

The \textsf{scalerel} solution was to use the image from the normal-size
representation and to shrink it down as needed.  So, in this case, a
normal-sized copy of the braces were stored in boxes called
\vb|\lXbrace| and \vb|\rXbrace|, resepctively.  Then the following
terms were defined:
\begin{verbatim}
\def\lxbrace{\scalerel*{\usebox{\lXbrace}}{\llbrace}}
\def\rxbrace{\scalerel*{\usebox{\rXbrace}}{\rrbrace}}
\end{verbatim} 
Now, in lieu of using \vb|\llbrace| directly, the use of
\vb|\lxbrace| would take a copy of the normal-sized \vb|\lXbrace|
and scale it down to the vertical height of the malformed tiny
\vb|\llbrace|, and substitute for it.  Here, we show the result at
both normal size and tiny size:

\newsavebox{\lXbrace}
\savebox{\lXbrace}{$\llbrace$}
\newsavebox{\rXbrace}
\savebox{\rXbrace}{$\rrbrace$}
\def\lxbrace{\scalerel*{\usebox{\lXbrace}}{\llbrace}}
\def\rxbrace{\scalerel*{\usebox{\rXbrace}}{\rrbrace}}
\rl$\scalerel*{\lxbrace x\rxbrace}{\wbrace}$
{\tiny$\scalerel*{\lxbrace x\rxbrace}{\wbrace}$}

Note that the brace is the properly formed scale of the normal-sized
brace in both cases, even as the $x$ adjusts to the fontsize reduction.

If one wanted the default braces narrower (to match the look of the
comparable symbols in the literature), one could use
\vb|\hstretch| in the definitions to achieve that quickly as
\begin{verbatim}
\def\lxbrace{%
   \hstretch{0.6}{\scalerel*{\usebox{\lXbrace}}{\llbrace}}}
\def\rxbrace{%
   \hstretch{0.6}{\scalerel*{\usebox{\rXbrace}}{\rrbrace}}}
\end{verbatim}

to give the following

\def\lxbrace{\hstretch{0.6}{\scalerel*{\usebox{\lXbrace}}{\llbrace}}}
\def\rxbrace{\hstretch{0.6}{\scalerel*{\usebox{\rXbrace}}{\rrbrace}}}
\rl$\scalerel*{\lxbrace x\rxbrace}{\wbrace}$
{\tiny$\scalerel*{\lxbrace x\rxbrace}{\wbrace}$}

Likewise, the \vb|\left| and \vb|\right| features of equation mode
cannot be used with the white braces, 

\rl \(\llbrace \left\llbracket\left\{\left[(a) \cdot
bc^{2^3}\right]^4\right\}^5\right\rrbracket \rrbrace\)

but \textsf{scalerel} can fix that, too:

\def\core{ \left\llbracket\left\{\left[(a) \cdot
bc^{2^3}\right]^4\right\}^5\right\rrbracket}%
\rl
\(\scaleleftright[1.5ex]{\lxbrace}{\core}{\rxbrace}\)

with 
\begin{verbatim}
  \(\scaleleftright[1.5ex]{\lxbrace}{\core}{\rxbrace}\)
\end{verbatim}
where \vb|\core| is the core inner equation to which the white braces
need to be scaled.  In this case, the scaled white braces were
width-limited to 1.5ex.

\subsection{Extra-Wide Oversymbols}

Ref:\\
http://tex.stackexchange.com/questions/100574/really-wide-hat-symbol/101136\#101136

There are many oversymbols defined to stand over a single-letter
variable.  There are a smaller number of extra-wide oversymbols to
extend over small groups of symbols.  Occasionally, however, a need
arises for an oversymbol to stretch over a larger group of symbols.

With these definitions,
\begin{verbatim}
\usepackage{scalerel,stackengine}

\newcommand\reallywidehat[1]{\ensurestackMath{%
\savestack{\tmpbox}{\stretchto{%
  \scaleto{%
    \scalerel*[\widthof{\ensuremath{#1}}]{\kern-.6pt\bigwedge\kern-.6pt}%
    {\rule[-\textheight/2]{1ex}{\textheight}}%WIDTH-LIMITED BIG WEDGE
  }{\textheight}% 
}{0.5ex}}%
\stackon[1pt]{#1}{\tmpbox}%
}}

\newcommand\reallywideparen[1]{\ensurestackMath{%
\savestack{\tmpbox}{\stretchto{%
  \scaleto{%
    \scalerel*[\widthof{\ensuremath{#1}}]{\kern-.6pt\frown\kern-.6pt}%
    {\rule[-\textheight/2]{1ex}{\textheight}}%WIDTH-LIMITED BIG FROWN
  }{\textheight}% 
}{0.9ex}}%
\stackon[-.5pt]{#1}{\tmpbox}%
}}
\end{verbatim}
arbitrarily wide oversymbols may be constructed:

\newcommand\reallywidehat[1]{\ensurestackMath{%
\savestack{\tmpbox}{\stretchto{%
  \scaleto{%
    \scalerel*[\widthof{\ensuremath{#1}}]{\kern-.6pt\bigwedge\kern-.6pt}%
    {\rule[-\textheight/2]{1ex}{\textheight}}%WIDTH-LIMITED BIG WEDGE
  }{\textheight}% 
}{0.5ex}}%
\stackon[1pt]{#1}{\tmpbox}%
}}

\newcommand\reallywideparen[1]{\ensurestackMath{%
\savestack{\tmpbox}{\stretchto{%
  \scaleto{%
    \scalerel*[\widthof{\ensuremath{#1}}]{\kern-.6pt\frown\kern-.6pt}%
    {\rule[-\textheight/2]{1ex}{\textheight}}%WIDTH-LIMITED BIG FROWN
  }{\textheight}% 
}{0.9ex}}%
\stackon[-.5pt]{#1}{\tmpbox}%
}}

\rl
$\reallywideparen{POQ}$ $+ \reallywidehat{QOR} + \reallywidehat{ROP}=
2\pi r$

\rl $\reallywidehat{zbcdefghijklm} \ne \reallywideparen{zbcdefghijklm}$

\section{Future Development}

It would be relatively straightforward to extend this approach to
horizontal scaling problems.  However, it is not exactly clear to the
author what format is best suited for user needs.  If he gets feedback
in that regard, it will inform him how best to proceed.

\section{Code Listing}

\small\verbatiminput{scalerel.sty}

\end{document}
