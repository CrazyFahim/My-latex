%--------------------------------------------
%
% Package pgfplots
%
% Provides a user-friendly interface to create function plots (normal
% plots, semi-logplots and double-logplots).
%
% It is based on Till Tantau's PGF package.
%
% Copyright 2010 by Christian Feuersänger.
%
% This program is free software: you can redistribute it and/or modify
% it under the terms of the GNU General Public License as published by
% the Free Software Foundation, either version 3 of the License, or
% (at your option) any later version.
%
% This program is distributed in the hope that it will be useful,
% but WITHOUT ANY WARRANTY; without even the implied warranty of
% MERCHANTABILITY or FITNESS FOR A PARTICULAR PURPOSE.  See the
% GNU General Public License for more details.
%
% You should have received a copy of the GNU General Public License
% along with this program.  If not, see <http://www.gnu.org/licenses/>.
%
%--------------------------------------------

% 1. OVERVIEW
%
% Plot handlers are extended versions of the plot handlers of PGF with
% backwards compatibility.
%
% To remind ourselfes: PGF plot handlers work like this
% \pgfplotstreamstart
% \pgfplotstreampoint{...}
% \pgfplotstreampoint{...}
% \pgfplotstreampoint{...}
% \pgfplotstreampoint{...}
% \pgfplotstreamend
%
% and that's it.
%
% PGFPlots plot handlers also contain these three macros. The tasks
% are (of course) the same. In addition, they support a set of further
% macros for every plot handler:
%
% \pgfplotsplothandlersurveystart
% \pgfplotsplothandlersurveypoint
% \pgfplotsplothandlersurveyend
%
% and serialization methods
% \pgfplotsplothandlerserializepointto
% \pgfplotsplothandlerdeserializepointfrom
% \pgfplotsplothandlerserializestateto
% \pgfplotsplothandlerdeserializepointfrom
%
% in addition, there are utility macros
% \pgfplotsplothandlersurveydifflen#1#2
% \pgfplotsplothandlersurveypointattime#1#2#3
%
% The idea is as follows:
% During \addplot, PGFPLots performs a survey phase. Survey means:
% nothing will be drawn, only stats will be collected. In this phase,
% the \pgfplotsplothandlersurvey* methods will be invoked; followed by
% a serialization.
%
% Then, when every plot has been surveyed, PGFPlots calls
% \pgfplotsplothandlerdeserializestatefrom{<serialized state>}
% \pgfplotstreamstart
% foreach <serizalized point> {
% 	\pgfplotsplothandlerdeserializepointfrom{<serialized point>}
% 	apply data transformations
% 	handle plot marks
% 	\pgfplotstreampoint{\pgfplotsqpointxy{<x>}{<y}}
% }
% \pgfplotstreamend
%
% See below for details about the survey phase and the visualization
% phase and the API of an axis.
%
% Thus, the PGF basic level streams can (but don't need to) make use
% of the macros assigned by the
% \pgfplotsplothandlerdeserializepointfrom and
% \pgfplotsplothandlerdeserializestatefrom.
%
% Furthermore, a plot handler *can* redefine any of the survey and/or
% serialization methods to add further functionality. But it doesn't
% need to, it's ok if it only relies on x,y as standard pgf plot
% handlers do.
%
% If it overwrites something, it should do so in its main method, the
% one in which it also defines the \pgf@plotstreamstart etc. methods
% (see pgfmoduleplot.code.tex for example). BUT: do NOT overwrite
% these methods globally!
%
% 2. A Description of the two phases:
%
% THE SURVEY PHASE:
%   The survey phase works as follows.
%   PRECONDITION:
%   	- we are `inside' of an axis.
%   	- \addplot starts the survey phase.
%
%   The following happens:
%   	- the plot handler is determined and activated such that its
%   	methods are usable ("class is instantiated")
%   	- \pgfplots@PREPARE@COORD@STREAM is the internal, low level
%   	entry point where pgfplots controls the survey phase.
%   	- the coordinate input routine is initialised. It issues a
%   	series of commands (all within the *same* TeX scope):
%   		\pgfplots@coord@stream@start
%   		\pgfplots@coord@stream@coord
%   		\pgfplots@coord@stream@coord
%   		\pgfplots@coord@stream@coord
%   		\pgfplots@coord@stream@coord
%   		\pgfplots@coord@stream@coord
%   		\pgfplots@coord@stream@coord
%   		...
%   		\pgfplots@coord@stream@end
%   	- \pgfplotsplothandlersurveystart is called in
%   	\pgfplots@coord@stream@start.
%   	- \pgfplots@coord@stream@coord calls
%   	\pgfplotsplothandlersurveypoint
%   	- \pgfplots@coord@stream@end calls
%   	\pgfplotsplothandlersurveyend
%   	\pgfplotsplothandlerserializestateto<\macro>
%   	Furthermore, it remembers the <\macro> such that it can be
%   	deserialized later.
%   Then, the survey phase ends. The main point of interest is the
%   \pgfplotsplothandlersurveypoint routine, especially its
%   communication with the axis. It is described below.
%
%   POSTCONDITION:
%   	- the state of the axis is now aware of the new plot (limits,
%   	stacking, ...).
%   	- The plot's survey state is stored using its serialized
%   	representation.
%
% THE VISUALIZATION PHASE:
% FIXME
%
% 2.1 The API of an axis
%
% As described above, the coordinate input routine fires a lot of
% \pgfplots@coord@stream@coord commands, which, in turn, invoke
% \pgfplotsplothandlersurveypoint. Somehow this should update the
% axis' state to reflect each point. But the 'data point' is a rather
% abstract thing. Usually, it will contain at least (x,y) (or maybe z)
% coordinates. But it may be more complex.
%
% So, the coordinate input routine provides whatever the user has
% chosen. Let's assume, we are using \addplot table. Then, we can
% access every cell in the current row (using \thisrow{<colname>} for
% example). The plot handler knows how to extract its information from
% this state. In general, the following steps are taken:
% - the plot handler assembles coordinates.
% - every assembled coordinate should be reported to the axis by
%   defining \pgfplots@current@point@[xyz] to its coordinates and
%   calling
%   \pgfplotsaxisparsecoordinate{}
%   This will apply coordinate filters, parse the single coordinates
%   and apply high level transformations and any logarithms.
%   It is some sort of advanced coordinate parser which works only for
%   (x,y) or for (x,y,z).
%   It yields (x,y,z). But the axis might need to change its
%   components! Thus, you also need to call
%   \pgfplotsaxispreparecoordinate{}.
%   This will, for example, apply the "stack plots" feature or the
%   'data cs' feature.
%
%   If necessary, the plot handler calls
%   \pgfplotsaxisparsecoordinate{} and
%   \pgfplotsaxispreparecoordinate{} multiple times, once for each
%   encountered coordinate.
%
%	It might occasionally be too much to call
%	\pgfplotsaxispreparecoordinate.
%
%   It might happen that a coordinate filter discards a coordinate.
%   This is returned in the \ifpgfplotsaxisparsecoordinateok boolean
%   and has to be checked by the plot handler.
%
% - the plot handler knows which of the coordinates contribute to the
%   final plot. It invokes
%   \ifpgfplotsaxisparsecoordinateok
%   	\pgfplotsaxisupdatelimitsforcoordinate{<x>}{<y>}{<z>}
%   \fi
%   for each of these coordinates. This has to be done for final
%   coordinates only, i.e. after \pgfplotsaxispreparecoordinate.
%
% - eventually, the plot handler is satisfied and considers a data
%   point as "readily surveyed". It is allowed if this does *not*
%   happen inside of \pgfplotsplothandlersurveypoint, but it must
%   happen before \pgfplotsplothandlersurveyend is finished.
%
%   The plot handler invokes \pgfplotsaxisdatapointsurveyed.
%   This tells the axis that it can perform its own surveying tasks
%   (see below) and furthermore, that it can serialize the data point.
%   Consequently, it will invoke
%   \pgfplotsplothandlerserializepointto{<\macro>}
%   and it will remember that \macro internally. This serialization is
%   employed to place plot marks and to apply z buffering techniques
%   (that's why it is done by the axis and not by the plot handler on
%   its own).
%
%   The axis does its own surveying task, initiated by
%   \pgfplotsaxisdatapointsurveyed (which is, turn, invoked by the
%   plot handler). This command handles the |point meta| feature, that
%   is: it queries the |point meta| input source and updates the meta
%   limits. Furthermore, the error bar feature is processed at this
%   point (using the final data point's (x,y,z) coordinates as basis).
%   The |xtick=data| feature is also prepared at this stage.
%
% - Later, the coordinate input routine invokes
%   \pgfplots@coord@stream@end indicating the end-of-input. This will
%   finalize the survey phase.
%
% A simple example looks like this:
% \pgfplotsplothandlersurveystart: does nothing in the simple example.
% \pgfplotsplothandlersurveypoint:
%   parses the input format somehow to get (x,y,z) in raw, symbolic format
%   calls \pgfplotsaxisparsecoordinate
% 	calls \pgfplotsaxispreparecoordinate
% 	calls \ifpgfplotsaxisparsecoordinateok	\pgfplotsaxisupdatelimitsforcoordinate{<x>}{<y>}{<z} \fi
% 	calls \pgfplotsaxisdatapointsurveyed
% \pgfplotsplothandlersurveyend: does nothing in the simple example.
%
% In fact, this is more or less the initial configuration used for
% lineto or similar plot handlers.
%
% 2.2 Further Survey Phase API functions
% 	\pgfplotssurveyphaseinputclass
% 	expands to the actual name of the input method (like
% 	'coordinates', 'expression' or 'table')
%
% 	\pgfplotsplothandlernotifyscanlinecomplete
% Called whenever an end-of-scanline marker has been processed, i.e.
% whenever the 'empty line' key has been triggered. The default is to
% do nothing.
%
%   \pgfplotsaxissurveysetpointmeta
% processes the 'point meta' key, i.e. it assigns
% \pgfplots@current@point@meta. This macro can only be called once for
% each coordinate (it will be deactivated afterwards). PGFPlots calls
% it in \pgfplotsaxisdatapointsurveyed .
%
%   \pgfplotsplothandlersurveyaddoptions{<options>}
% which allows to change the current plot style from within API
% functions. It sets <options> and remembers them for the
% visualization phase.
%
%   \pgfplotsaxisupdatelimitsforpointmeta{<meta>},
% provided there is a point meta input handler (which is numeric).
% Otherwise, the command is equal to \relax.
%
%
% 3. Details about the VISUALIZATION phase
%
% The visualization phase consists of
% <installation of the plot handler>
% \pgfplotstreamstart
% foreach serialized coordinate {
% 	pgfplots calls \pgfplotsplothandlerdeserializestatefrom{<serialized repr>}
% 	if coordinate is empty ("unbounded")
% 		pgfplots call \pgfplotsplothandlervisualizejump
% 	else
% 		pgfplots calls \pgfplotsaxisvisphasetransformcoordinate
% 		pgfplots calls \pgfplotsaxisvisphasepreparedatapoint
% 		pgfplots calls either \pgfplotsqpointxyz or \pgfplotsqpointxy
% 	fi
% 	\pgfplotstreampoint
% }
% \pgfplotstreamend
%
% User defined plot handlers might need to invoke
% \pgfplotsaxisvisphasetransformcoordinate on their own.
%
% During the visualization phase, the following macros can be used:
%
% - \pgfplotsaxisvisphasetransformpointmeta to set up point meta.
% Use this only if there *is* point meta, see
% \pgfplotsaxisifhaspointmeta{<true code>}{<false code>}.
%
% - \pgfplotsaxisvisphasegetpoint
%   does not take arguments. It takes the current point as input and
%   sets \pgf@x, \pgf@y to the final result.
%
% 4. API Functions of pgfplots to work with the visualization phases
%
% \pgfplotsifissurveyphase{<true code>}{<false code>}
% \pgfplotsifisvisualizationphase{<true code>}{<false code>}
%
% \pgfplotsaxisfilteredcoordsaway
%   This macro expands to '1' if all points have been surveyed
%   successfully. It expands to '0' if at least one point has been
%   filtered away (for whatever reasons). This does not apply to
%   jumps.
% \pgfplotsaxisplothasjumps
% 	This macro expands to '1' if the current plot has jumps and '0'
% 	if not.During the visualization phase, a jump is usually indicated
% 	by an empty coordinate.
% \pgfplotsaxisplothasunboundedpointmeta
% 	This macro expands to '1' if the current plot has a point with
% 	unbounded point meta and '0'
% 	if not. Currently used by some plot handlers as synomym for
% 	"visualize a jump"

\def\pgfplotssurveyphaseinputclass{direct}

\def\pgfplotsplothandlers@tikz@with@snap@to@nearest{%
	\pgfplothandlerdiscard,%
	\pgfplothandlermark,%
	\pgfplothandlermarklisted,%
	\pgfplothandlerxbar,%
	\pgfplothandlerybar,%
	\pgfplothandlerxbarinterval,%
	\pgfplothandlerybarinterval%
}%

\def\pgfplotsplothandlers@tikz@std{%
	\pgfplothandlerdiscard,%
	\pgfplothandlermark,%
	\pgfplothandlermarklisted,%
	\pgfplothandlerxbar,%
	\pgfplothandlerybar,%
	\pgfplothandlerxbarinterval,%
	\pgfplothandlerybarinterval,%
	\pgfplothandlerlineto,%
	\pgfplothandlercurveto,%
	\pgfplothandlerconstantlineto,%
	\pgfplothandlerconstantlinetomarkright,%
	\pgfplothandlerconstantlinetomarkmid,%
	\pgfplothandlerpolarcomb,%
	\pgfplothandlerjumpmarkmid,%
	\pgfplothandlerjumpmarkleft,%
	\pgfplothandlerjumpmarkright%
}%

% Defines \pgfplotsretval to be the csname for the BACKUP of plot
% handler #1.
%
% For example:
% \pgfplotsplothandlers@get@tikz@backup@name{\pgfplothandlerlineto}
% will return \pgfplotsretval={\\pgfplothandlerlineto@tikz}
% (up to \escapechar)
\def\pgfplotsplothandlers@get@tikz@backup@name#1{%
	\begingroup
		%\escapechar=-1 % drop the leading backslash
		\expandafter\gdef\expandafter\pgfplotsretval\expandafter{\csname \string#1@tikz\endcsname}%
		\pgfmath@smuggleone\pgfplotsretval
	\endgroup
}%

% Takes the CURRENT \tikz@plot@handler and checks if it is an UNPATCHED tikz plot
% handler. If so, it replaced it with the correct patched version.
%
% The motivation is that uf \tikz@plot@handler has been set outside of
% an axis, it will not reflect the most recent changes (which are only
% applied within an axis).
\def\pgfplotsplothandlers@init@map@to@patched@versions{%
	% iterate through all tikz plot handlers...
	\expandafter\pgfplotsutilforeachcommasep\expandafter{\pgfplotsplothandlers@tikz@std}%
	\as\pgfplots@loc@TMPa{%
		% ... get the name of the backup (see \pgfplotsplothandlers@init)
		\expandafter\pgfplotsplothandlers@get@tikz@backup@name\pgfplots@loc@TMPa
		\expandafter\ifx\pgfplotsretval\tikz@plot@handler
			% AH! \tikz@plot@handler is the same as some backup name!
			% Replace it:
			\expandafter\let\expandafter\tikz@plot@handler\pgfplots@loc@TMPa
		\fi
	}%
}%

\def\pgfplotsplothandlers@init{%
	\def\pgfplotsplothandlers@init@##1{%
		\pgfplotsutil@add@to@macro##1{%
			\let\pgfplotsplothandlersurveydifflen=\pgfplotsplothandlersurveydifflen@snaptonearest
			\let\pgfplotsplothandlersurveypointattime=\pgfplotsplothandlersurveypointattime@snaptonearest
		}%
	}%
	\def\pgfplotsplothandlers@init@@##1{%
		% create backup:
		%
		% if ##1 = \pgfplothandlerlineto, this defines
		% \pgfplothandlerlineto@tikz as backup.
		\pgfplotsplothandlers@get@tikz@backup@name{##1}%
		\expandafter\let\pgfplotsretval=##1%
		%
		%
		% assign more suitable names:
		\pgfplotsutil@add@to@macro##1{%
			\begingroup
			\escapechar=-1 % drop the leading backslash
			\edef\pgfplotsplothandlername{\string##1}%
			\pgfmath@smuggleone\pgfplotsplothandlername
			\endgroup
			%
			\def\pgfplotsplothandlerLUAfactory{function(axis, pointmetainputhandler) return pgfplots.GenericPlothandler.new("\pgfplotsplothandlername", axis,pointmetainputhandler) end}%
			\def\pgfplotsplothandlerLUAvisualizerfactory{pgfplots.defaultPlotVisualizerFactory}%
		}%
	}%
	% Patch all TikZ plot handlers:
	% this here also creates backups
	\expandafter\pgfplotsutilforeachcommasep\expandafter{\pgfplotsplothandlers@tikz@std}%
	\as\pgfplots@loc@TMPa{%
		\expandafter\pgfplotsplothandlers@init@@\expandafter{\pgfplots@loc@TMPa}%
	}%
	%
	% Patch only selected ones:
	\expandafter\pgfplotsutilforeachcommasep\expandafter{\pgfplotsplothandlers@tikz@with@snap@to@nearest}%
	\as\pgfplots@loc@TMPa{%
		\expandafter\pgfplotsplothandlers@init@\expandafter{\pgfplots@loc@TMPa}%
	}%
	%
	\if1\b@pgfplots@compat@bar@width@units
	\else
		\def\pgfplotbarwidth{\pgfplots@bar@width@not@in@context}%
		\def\pgfplotbarshift{\pgfplots@bar@shift@not@in@context}%
		\pgfplotsutil@add@to@macro\pgfplothandlerxbar{%
			\def\pgfplotbarwidth{\pgfplots@xbar@width}%
			\def\pgfplotbarshift{\pgfplots@xbar@shift}%
		}%
		\pgfplotsutil@add@to@macro\pgfplothandlerybar{%
			\def\pgfplotbarwidth{\pgfplots@ybar@width}%
			\def\pgfplotbarshift{\pgfplots@ybar@shift}%
		}%
	\fi
	%
	\pgfplotsplothandlers@init@map@to@patched@versions
	%
	%
%	\pgfkeys{%
%		/pgf/at begin bar/.add={}{%
%			\if1\b@has@pgfplots@colordinate@style
%				\expandafter\pgfplots@bar@extra
%			\fi
%		},%
%	}%
	%
}%

\def\pgfplots@bar@extra#1\pgfkeysvalueof#2{%
	\expandafter\tikz\expandafter[\pgfplots@current@point@coordinatestyle] \pgfextra{#1};%
}%

\def\pgfplots@xbar@width{pgfplotsxbarwidth}
\def\pgfplots@ybar@width{pgfplotsybarwidth}
\def\pgfplots@xbar@shift{pgfplotsxbarshift}
\def\pgfplots@ybar@shift{pgfplotsybarshift}
\def\pgfplots@bar@width@not@in@context{pgfplotsbarwidthgeneric}
\def\pgfplots@bar@shift@not@in@context{pgfplotsbarshiftgeneric}

\pgfplotsmathdeclarepseudoconstant{pgfplotsxbarwidth}{\pgfplots@bar@mathparse@{y}{bar width}}%
\pgfplotsmathdeclarepseudoconstant{pgfplotsybarwidth}{\pgfplots@bar@mathparse@{x}{bar width}}%
\pgfplotsmathdeclarepseudoconstant{pgfplotsxbarshift}{\pgfplots@bar@mathparse@{y}{bar shift}}%
\pgfplotsmathdeclarepseudoconstant{pgfplotsybarshift}{\pgfplots@bar@mathparse@{x}{bar shift}}%
\pgfplotsmathdeclarepseudoconstant{pgfplotsbarwidthgeneric}{\pgfplots@bar@mathparse@{N}{bar width}}%
\pgfplotsmathdeclarepseudoconstant{pgfplotsbarshiftgeneric}{\pgfplots@bar@mathparse@{N}{bar shift}}%

\def\pgfplots@bar@mathparse@#1#2{%
	\pgfmathparse{\pgfkeysvalueof{/pgf/#2}}%
	\ifpgfmathunitsdeclared
	\else
		\edef\pgfplots@bar@direction@choice@{#1}%
		\if N\pgfplots@bar@direction@choice@%
			\if a\pgfplots@bar@direction@choice
			\else
				\if x\pgfplots@bar@direction@choice
					\def\pgfplots@bar@direction@choice@{y}%
				\else
					\if y\pgfplots@bar@direction@choice
						\def\pgfplots@bar@direction@choice@{x}%
					\else
						\pgfplotsthrow{invalid argument}{\pgfplots@bar@direction@choice@}{Sorry, the value of 'bar direction' is invalid}\pgfeov%
					\fi
				\fi
			\fi
		\fi
		\if N\pgfplots@bar@direction@choice@%
			\pgfplots@bar@mathparse@error{#1}{#2}%
		\else
			\let\pgfplots@loc@TMPa=\pgfmathresult
			\csname pgfplotstransformdirection\pgfplots@bar@direction@choice@\endcsname{\pgfplots@loc@TMPa}%
			\let\pgfplots@loc@TMPa=\pgfmathresult
			\if\pgfplots@bar@direction@choice@ x%
				\pgfqpointxy@orig{\pgfplots@loc@TMPa}{0}%
				\edef\pgfmathresult{\pgf@sys@tonumber\pgf@x}%
			\else
				\pgfqpointxy@orig{0}{\pgfplots@loc@TMPa}%
				\edef\pgfmathresult{\pgf@sys@tonumber\pgf@y}%
			\fi
			%\edef\pgfplots@loc@TMPa{{\pgf@sys@tonumber\pgf@x}{\pgf@sys@tonumber\pgf@y}}%
			%\expandafter\pgfmathveclen@\pgfplots@loc@TMPa
		\fi
	\fi
}%

\def\pgfplots@bar@mathparse@error#1#2{%
	\pgfplotsthrow{invalid argument}{\pgfplots@bar@direction@choice@}{Sorry, the value '#2=\pgfkeysvalueof{/pgf/#2}' is given in terms of a unit -- but I do not know which axis! Next steps: either (a) set one of 'xbar' or 'ybar' before evaluating the value of '#2' or (b) define 'bar direction=x or y'}\pgfeov%
}%

% Resets the plot handler routines.
%
% This is necessary before installing a new plot handler!
\def\pgfplotsresetplothandler{%
	\let\pgfplotsplothandlersurveystart=\pgfplotsplothandlersurveystart@default
	\let\pgfplotsplothandlerLUAfactory=\pgfplotsplothandlerLUAfactory@default
	\let\pgfplotsplothandlerLUAvisualizerfactory=\pgfplotsplothandlerLUAvisualizerfactory@default
	\let\pgfplotsplothandlername=\pgfplotsplothandlername@default
	\let\pgfplotsplothandlersurveyend=\pgfplotsplothandlersurveyend@default
	\let\pgfplotsplothandlersurveypoint=\pgfplotsplothandlersurveypoint@default
	\let\pgfplotsplothandlerpointtokeys=\pgfplotsplothandlerpointtokeys@default
	\let\pgfplotsplothandlerserializepointto=\pgfplotsplothandlerserializepointto@default
	\let\pgfplotsplothandlerdeserializepointfrom=\pgfplotsplothandlerdeserializepointfrom@default
	\let\pgfplotsplothandlerserializestateto=\pgfplotsplothandlerserializestateto@default
	\let\pgfplotsplothandlerdeserializestatefrom=\pgfplotsplothandlerdeserializestatefrom@default
	\let\pgfplotsplothandlervisualizejump=\pgfplotsplothandlervisualizejump@default
	\let\pgfplotsplothandlernotifyscanlinecomplete=\relax
	\let\pgfplotsplothandlersurveydifflen=\pgfplotsplothandlersurveydifflen@default
	\let\pgfplotsplothandlersurveypointattime=\pgfplotsplothandlersurveypointattime@default
	\let\pgfplotsplothandlertransformslopedattime=\pgfplotsplothandlertransformslopedattime@default
	\let\pgfplotsplothandlerifcurrentpointcanbefirstlast=\pgfplotsplothandlerifcurrentpointcanbefirstlast@default
	%
	\let\pgfplotsplothandlersurveybeforesetpointmeta=\pgfplotsplothandlersurveybeforesetpointmeta@default
	\let\pgfplotsplothandlersurveyaftersetpointmeta=\pgfplotsplothandlersurveyaftersetpointmeta@default
}%

\def\pgfplotsplothandlersurveybeforesetpointmeta@default{}
\def\pgfplotsplothandlersurveyaftersetpointmeta@default{}

% \pgfplotsplothandlersurveystart
\def\pgfplotsplothandlersurveystart@default{}%

% \pgfplotsplothandlername
\def\pgfplotsplothandlername@default{%
	[tikz@plot@handler: \meaning\pgf@plotstreamstart]%
}%

% This should expand to a LUA function which takes the axis and the point
% meta handler.
% Use empty if there is none.
\def\pgfplotsplothandlerLUAfactory@default{}%

% This should expand to a LUA function which takes an instance of
% Plothandler and which should return a PlotVisualizer.
% Use empty if there is none.
%
% @see LUA: pgfplots.PlotVisualizer
\def\pgfplotsplothandlerLUAvisualizerfactory@default{}%

% \pgfplotsplothandlerifcurrentpointcanbefirstlast : can be used to
% check if the current point of a plot handler can be the global first
% or last segment.
% It will execute #1 if that is the case and #2 if not.
\def\pgfplotsplothandlerifcurrentpointcanbefirstlast@default#1#2{#1}%


% \pgfplotsplothandlersurveyend
% Called at the end of each survey phase.
\def\pgfplotsplothandlersurveyend@default{}

% A callback which will be called as soon as a scanline is complete.
%
% The callback can differ from plot handler to plot handler; its
% purpose is to update data structures.
%
% Implementational note for those who _call_ the callback: the
% The callback will be triggered by the 'empty line' handling, see
% \pgfplotsscanlinelengthinitzero. However, a coordinate generator can
% safely invoke it directly. It must not be invoked twice for the same
% scanline.
%
% FIXME: only 'empty line=scanline' currently calls this callback!
\let\pgfplotsplothandlernotifyscanlinecomplete=\relax%

% \pgfplotsplothandlersurveypoint is called for each encountered data
% point.
%
% The data point as such is available using the current state of any
% macros which are assigned during the survey phase (during \addplot).
% This includes any table macros etc.
% PGFPlots stores the x,y and z coordinates into \pgfplots@current@point@[xyz].
% The point meta coordinate is in \pgfplots@current@point@meta.
%
% Note that since any currently assigned macro can be used here, the
% new DV engine of PGF is also valid (and will be supported
% eventually). This DV engine stores data point entries in keys,
% namely those in the key path /data point. See the pgf manual.
\def\pgfplotsplothandlersurveypoint@default{%
	\ifpgfplots@LUA@backend@supported
		\pgfplots@LUA@survey@point
	\else
		\pgfplotsplothandlersurveypoint@default@
	\fi
}%
\def\pgfplotsplothandlersurveypoint@default@{%
	% reset it. NOTE: this migh be done multiple times. But better one
	% too much than one too few...
	\def\pgfplots@set@perpointmeta@done{0}%
	%
	\pgfplotsplothandlersurveypoint@default@noreset@of@pointmeta
}%
\def\pgfplotsplothandlersurveypoint@default@noreset@of@pointmeta{%
	\pgfplotsaxisparsecoordinate
	\pgfplotsaxispreparecoordinate
	\ifpgfplotsaxisparsecoordinateok
		\pgfplotsaxisupdatelimitsforcoordinate\pgfplots@current@point@x\pgfplots@current@point@y\pgfplots@current@point@z
	\fi
	\pgfplotsaxisdatapointsurveyed
}%

% \pgfplotsplothandlerserializepointto{<\macro>}
% should save a complete data point to <\macro> such that it can be
% de-serialized later.
%
% #1: a macro name. Will be filled with (expandable) data.
% 	The format can be arbitrary, but you should be able to extra it.
%
% @PRECONDITION
% 	this macro will be invoked in a context where the current data
% 	point has been processed completely, including any preparations.
% 	The required data which should be saved depends on the plot
% 	handler. Usually, all plot handlers require
% 	\pgfplots@current@point@[xyz] and \pgfplots@current@point@meta.
%   This macro should only assign keys which have been defined or
%   validated by any of the plot handler relevant methods (including
%   the de-serialization or survey methods).
\def\pgfplotsplothandlerserializepointto@default#1{%
	% Store normalized point for list:
	% We need
	% xi,yi,zi;
	% where zi may be empty.
	%
	% Note that per-point meta information is stored in
	% \pgfplotsaxisserializedatapoint .
	\edef#1{\pgfplots@current@point@x,\pgfplots@current@point@y,\pgfplots@current@point@z}%
}%

% \pgfplotsplothandlerdeserializepointfrom{<\macro>}
% the counterpart for \pgfplotsplothandlerserializepointto.
% It restores the state as it was before the serialization.
%
% #1: the serialized information.
\def\pgfplotsplothandlerdeserializepointfrom@default#1{%
	\expandafter\pgfplotsplothandlerdeserializepointfrom@default@#1\relax
}%

% \pgfplotsplothandlerpointtokeys{<key prefix>}
%
% Takes the current point and copies its values to a set of keys.
%
% For example, if the current point has the three coordinates x=1,
% y=2, z=3,
% \pgfplotsplothandlerpointtokeys{/data point/first/}
% will define the keys
% /data point/first/x/.initial=1
% /data point/first/y/.initial=2
% /data point/first/z/.initial=3
%
\def\pgfplotsplothandlerpointtokeys@default#1{%
	\pgfkeyslet{#1/x}\pgfplots@current@point@x
	\pgfkeyslet{#1/y}\pgfplots@current@point@y
	\pgfkeyslet{#1/z}\pgfplots@current@point@z
}%

\def\pgfplotsplothandlerdeserializepointfrom@default@#1,#2,#3\relax{%
	\def\pgfplots@current@point@x{#1}%
	\def\pgfplots@current@point@y{#2}%
	\def\pgfplots@current@point@z{#3}%
}%
% \pgfplotsplothandlerserializestateto{<\macro>}
% should save the state of the current plot handler such that it can
% be de-serialized later.
%
% The state does usually NOT contain a coordinate stream, this is
% accomplished by \pgfplotsplothandlerserializepointto.
%
% #1: a macro name. Can be filled with anything, including
% non-expandable macro invocations.
\def\pgfplotsplothandlerserializestateto@default#1{%
	\def#1{}%
}%

%
\def\pgfplotsplothandlerdeserializestatefrom@default#1{%
	#1%
}%

\def\pgfplotsplothandlervisualizejump@default{%
	\pgfplotstreamend
	\pgfplotstreamstart
}%

% \pgfplotsplothandlersurveypointattime#1#2#3
%
% sets the current environment to a point which is between points #2
% and #3, using the fraction #1.
%
% #1 a fraction (a number between 0 and 1) in the format of
% \pgfplotscoordmath{default}
% #2 a serialized point denoting the start
% #3 a serialized point denoting the end
%
% In other words: #1 = 0.0 should result in #2 and #1 = 1.0 should
% result in #3.
%
% POSTCONDITION: the current point will be set to the point
% in-between. The current point is set in terms of logical coordinates
% (i.e. \pgfplots@current@point@x and its variants)
\def\pgfplotsplothandlersurveypointattime@default#1#2#3{%
	\begingroup
	\pgfplotsplothandlerdeserializepointfrom{#2}%
	\let\pgfplots@last@x=\pgfplots@current@point@x
	\let\pgfplots@last@y=\pgfplots@current@point@y
	\let\pgfplots@last@z=\pgfplots@current@point@z
	\pgfplotsplothandlerdeserializepointfrom{#3}%
	%
	\def\pgfplots@loc@TMPa##1{%
		\pgfplotscoordmath{##1}{op}{subtract}{{\csname pgfplots@current@point@##1\endcsname}{\csname pgfplots@last@##1\endcsname}}%
		\let\pgfplots@diff=\pgfmathresult
		\pgfplotscoordmath{##1}{parsenumber}{#1}%
		\pgfplotscoordmath{##1}{op}{multiply}{{\pgfmathresult}{\pgfplots@diff}}%
		\pgfplotscoordmath{##1}{op}{add}{{\csname pgfplots@last@##1\endcsname}{\pgfmathresult}}%
		\expandafter\let\csname pgfplots@current@point@##1\endcsname=\pgfmathresult
	}%
	\pgfplots@loc@TMPa x%
	\pgfplots@loc@TMPa y%
	\ifpgfplots@curplot@threedim
		\pgfplots@loc@TMPa z%
	\fi
	\xdef\pgfplots@glob@TMPb{%
		\noexpand\def\noexpand\pgfplots@current@point@x{\pgfplots@current@point@x}%
		\noexpand\def\noexpand\pgfplots@current@point@y{\pgfplots@current@point@y}%
		\noexpand\def\noexpand\pgfplots@current@point@z{\pgfplots@current@point@z}%
	}%
	\endgroup
	\pgfplots@glob@TMPb
}%

% \pgfplotsplothandlersurveydifflen#1#2
%
% computes the length between two points which are given in logical
% coordinates.
% #1 a serialized point
% #2 a serialized point
%
% The return value is assigned to \pgfmathresult in
% \pgfplotscoordmath{default} format.
\def\pgfplotsplothandlersurveydifflen@default#1#2{%
	\begingroup
		\pgfplotsplothandlerdeserializepointfrom{#1}%
		\let\pgfplots@last@x=\pgfplots@current@point@x
		\let\pgfplots@last@y=\pgfplots@current@point@y
		\let\pgfplots@last@z=\pgfplots@current@point@z
		\pgfplotsplothandlerdeserializepointfrom{#2}%
		\pgfplotscoordmathparsemacro{default}\pgfplots@last@x
		\pgfplotscoordmathparsemacro{default}\pgfplots@last@y
		\pgfplotscoordmathparsemacro{default}\pgfplots@current@point@x
		\pgfplotscoordmathparsemacro{default}\pgfplots@current@point@y
		\ifpgfplots@curplot@threedim
			\pgfplotscoordmathparsemacro{default}\pgfplots@last@z
			\pgfplotscoordmathparsemacro{default}\pgfplots@current@point@z
		\fi
		\pgfplotscoordmath{default}{op}{subtract}{{\pgfplots@current@point@x}{\pgfplots@last@x}}%
		\pgfplotscoordmath{default}{op}{multiply}{{\pgfmathresult}{\pgfmathresult}}%
		\let\pgfplots@diff@x=\pgfmathresult
		\pgfplotscoordmath{default}{op}{subtract}{{\pgfplots@current@point@y}{\pgfplots@last@y}}%
		\pgfplotscoordmath{default}{op}{multiply}{{\pgfmathresult}{\pgfmathresult}}%
		\let\pgfplots@diff@y=\pgfmathresult
		\pgfplotscoordmath{default}{op}{add}{{\pgfplots@diff@x}{\pgfplots@diff@y}}%
		\let\pgfplots@len=\pgfmathresult
		\ifpgfplots@curplot@threedim
			\pgfplotscoordmath{default}{op}{subtract}{{\pgfplots@current@point@z}{\pgfplots@last@z}}%
			\pgfplotscoordmath{default}{op}{multiply}{{\pgfmathresult}{\pgfmathresult}}%
			\let\pgfplots@diff@z=\pgfmathresult
			\pgfplotscoordmath{default}{op}{add}{{\pgfplots@len}{\pgfplots@diff@z}}%
			\let\pgfplots@len=\pgfmathresult
		\fi
		\pgfplotscoordmath{default}{op}{sqrt}{{\pgfplots@len}}%
		\pgfmath@smuggleone\pgfmathresult
	\endgroup
}%

% \pgfplotsplothandlertransformslopedattime{<time fraction>}{<start>}{<end>}
%
% Installs a PGF rotation matrix such that it fits the gradient of the
% current plot segment between <start> and <end>.
%
% #1: a fraction such that 0.0 is <start> and 1.0 is <end>
% #2: the <start> point (a macro containing the result of \pgfplotsplothandlerserializepointto)
% #3: the <end> point   (a macro containing the result of \pgfplotsplothandlerserializepointto)
%
\def\pgfplotsplothandlertransformslopedattime@default#1#2#3{%
    \pgf@process{%
		\pgfplotsplothandlerdeserializepointfrom{#2}%
		\pgfplotsaxisvisphasegetpoint
	}%
    \pgf@xa=\pgf@x% xb/yb = start point
    \pgf@ya=\pgf@y%
    \pgf@process{%
		\pgfplotsplothandlerdeserializepointfrom{#3}%
		\pgfplotsaxisvisphasegetpoint
	}%
    \advance\pgf@x by-\pgf@xa%
    \advance\pgf@y by-\pgf@ya%
    \ifpgfallowupsidedownattime%
    \else%
      \ifdim\pgf@x<0pt%
        \pgf@x=-\pgf@x%
        \pgf@y=-\pgf@y%
      \fi%
    \fi%
    \pgfpointnormalised{}% x/y = normalised vector
    \pgf@ya=-\pgf@y%
    \pgftransformcm%
    {\pgf@sys@tonumber{\pgf@x}}{\pgf@sys@tonumber{\pgf@y}}%
    {\pgf@sys@tonumber{\pgf@ya}}{\pgf@sys@tonumber{\pgf@x}}{\pgfpointorigin}%
}%



\pgfplotsresetplothandler

% The following two methods constitutes implementations for
% 'node[pos=<faction>]' which do not interpolate. They only snap to
% the nearest coordinate.
%
% For example, \addplot[scatter] ... node[pos=0.5] {}  should use a
% unit distance and should not interpolate between scatter points.
\def\pgfplotsplothandlersurveydifflen@snaptonearest#1#2{%
	% FIXME : this implies that #1 and #2 are "adjacent" in the
	% coordinate stream!
	\pgfplotscoordmath{default}{one}%
}%
\def\pgfplotsplothandlersurveypointattime@snaptonearest#1#2#3{%
	\pgfplotscoordmath{default}{parsenumber}{0.5}%
	\let\pgfplots@loc@TMPa=\pgfmathresult
	\pgfplotscoordmath{default}{parsenumber}{#1}%
	\pgfplotscoordmath{default}{if less than}{\pgfmathresult}{\pgfplots@loc@TMPa}{%
		\pgfplotsplothandlerdeserializepointfrom{#2}%
	}{%
		\pgfplotsplothandlerdeserializepointfrom{#3}%
	}%
}%
% ==================================

% Defines
% - a generic update limits routine,
%   \pgfplotsaxisupdatelimitsforcoordinate#1#2#3
%   if #3 is empty, it will assume a 2d point, otherwise a 3d point
%   and the axis will be three dimensional as well.
%   During \addplot, this auto-detection will be disabled in favor of
%   the '\addplot3' versus' \addplot' syntax.
%
\def\pgfplots@prepare@axis@API{%
	\pgfplots@curplot@threedimtrue
	\pgfplots@prepare@axis@API@
	\let\pgfplotsaxisupdatelimitsforcoordinatethreedim=\pgfplotsaxisupdatelimitsforcoordinate@
	\let\pgfplotsaxisparsecoordinatethreedim=\pgfplotsaxisparsecoordinate@
	%
	\pgfplots@curplot@threedimfalse
	\pgfplots@prepare@axis@API@
	\let\pgfplotsaxisupdatelimitsforcoordinatetwodim=\pgfplotsaxisupdatelimitsforcoordinate@
	\let\pgfplotsaxisparsecoordinatetwodim=\pgfplotsaxisparsecoordinate@
	%
	\def\pgfplotsaxisupdatelimitsforcoordinate##1##2##3{%
		\pgfplots@ifempty{##3}{%
			\pgfplotsaxisupdatelimitsforcoordinatetwodim{##1}{##2}{}%
		}{%
			\global\pgfplots@threedimtrue
			\pgfplotsaxisupdatelimitsforcoordinatethreedim{##1}{##2}{##3}%
		}%
	}%
	\def\pgfplotsaxisparsecoordinate{%
		\ifx\pgfplots@current@point@z\pgfutil@empty
			\pgfplotsaxisparsecoordinatetwodim
		\else
			\global\pgfplots@threedimtrue
			\pgfplotsaxisparsecoordinatethreedim%
		\fi
	}%
}%
\def\pgfplots@prepare@axis@API@{%
	\begingroup
	%%%%%%%%%%%%%%%%%%%%%%%%%%%%%%%%%%%%%%%%%%%%%%%%%%%%%%%%%%%%%%%%%%%%
	\let\E=\noexpand
	% Setup Just-In-Time-Macro Compilation:
	% I compile a set of macros which is highly optimized for this
	% particular axis configuration.
	%
	% \pgfplotsaxisupdatelimitsforcoordinate
	% Updates the current x and y limits for point (#1,#2).
	%
	% To eliminate all those case distinctions, it is created with
	% 'edef' and a lot of '\noexpand' calls here:
	%
	%
	% The point coordinates are given in floating point format (FIXME)
	%
	% Please note that if user specified limits are given, automatic
	% limits are only applied to points which fall into the user specified
	% clipping region.
	%
	% PRECONDITIONS:
	% - the input coordinates have been parsed correctly (floating point
	%   format for linear axis, log applied for logarithmic ones)
	%
	% Arguments:
	% #1,#2,#3 the x,y and z coordinate. z is ignored for 2d plots.
	\xdef\pgfplotsaxisupdatelimitsforcoordinate@##1##2##3{%
%\E\tracingmacros=2\E\tracingcommands=2
%\E\pgfplots@message{Updating limits for (##1,##2) ...}%
		%
		% VIM SEARCH PATTERN:
		%   [^E]\zs\\\ze[^E]
		% -> this finds '\' which is neither '\E' nor is it prefixed
		%  by 'E'.
		%
		%
		%
		\E\pgfplots@update@limits@for@one@point@ISCLIPPEDfalse
		% check whether we need to clip limits:
		\ifpgfplots@clip@limits@x
			\ifpgfplots@autocompute@xmin
			\else
				\ifpgfplots@xislinear
					\E\pgfmathfloatlessthan@{##1}{\E\pgfplots@xmin}%
					\E\ifpgfmathfloatcomparison
						\E\pgfplots@update@limits@for@one@point@ISCLIPPEDtrue
					\E\fi
				\else
					\E\pgfplotsmathlessthan{##1}{\E\pgfplots@xmin}%
					\E\ifpgfmathfloatcomparison
						\E\pgfplots@update@limits@for@one@point@ISCLIPPEDtrue
					\E\fi
				\fi
			\fi
			\ifpgfplots@autocompute@xmax
			\else
				\ifpgfplots@xislinear
					\E\pgfmathfloatlessthan@{\E\pgfplots@xmax}{##1}%
					\E\ifpgfmathfloatcomparison
						\E\pgfplots@update@limits@for@one@point@ISCLIPPEDtrue
					\E\fi
				\else
					\E\pgfplotsmathlessthan{\E\pgfplots@xmax}{##1}%
					\E\ifpgfmathfloatcomparison
						\E\pgfplots@update@limits@for@one@point@ISCLIPPEDtrue
					\E\fi
				\fi
			\fi
		\fi
		\ifpgfplots@clip@limits@y
			\ifpgfplots@autocompute@ymin
			\else
				\ifpgfplots@yislinear
					\E\pgfmathfloatlessthan@{##2}{\E\pgfplots@ymin}%
					\E\ifpgfmathfloatcomparison
						\E\pgfplots@update@limits@for@one@point@ISCLIPPEDtrue
					\E\fi
				\else
					\E\pgfplotsmathlessthan{##2}{\E\pgfplots@ymin}%
					\E\ifpgfmathfloatcomparison
						\E\pgfplots@update@limits@for@one@point@ISCLIPPEDtrue
					\E\fi
				\fi
			\fi
			\ifpgfplots@autocompute@ymax
			\else
				\ifpgfplots@yislinear
					\E\pgfmathfloatlessthan@{\E\pgfplots@ymax}{##2}%
					\E\ifpgfmathfloatcomparison
						\E\pgfplots@update@limits@for@one@point@ISCLIPPEDtrue
					\E\fi
				\else
					\E\pgfplotsmathlessthan{\E\pgfplots@ymax}{##2}%
					\E\ifpgfmathfloatcomparison
						\E\pgfplots@update@limits@for@one@point@ISCLIPPEDtrue
					\E\fi
				\fi
			\fi
		\fi
		\ifpgfplots@clip@limits@z
			\ifpgfplots@curplot@threedim
				\ifpgfplots@autocompute@zmin
				\else
					\ifpgfplots@zislinear
						\E\pgfmathfloatlessthan@{##3}{\E\pgfplots@zmin}%
						\E\ifpgfmathfloatcomparison
							\E\pgfplots@update@limits@for@one@point@ISCLIPPEDtrue
						\E\fi
					\else
						\E\pgfplotsmathlessthan{##3}{\E\pgfplots@zmin}%
						\E\ifpgfmathfloatcomparison
							\E\pgfplots@update@limits@for@one@point@ISCLIPPEDtrue
						\E\fi
					\fi
				\fi
				\ifpgfplots@autocompute@zmax
				\else
					\ifpgfplots@zislinear
						\E\pgfmathfloatlessthan@{\E\pgfplots@zmax}{##3}%
						\E\ifpgfmathfloatcomparison
							\E\pgfplots@update@limits@for@one@point@ISCLIPPEDtrue
						\E\fi
					\else
						\E\pgfplotsmathlessthan{\E\pgfplots@zmax}{##3}%
						\E\ifpgfmathfloatcomparison
							\E\pgfplots@update@limits@for@one@point@ISCLIPPEDtrue
						\E\fi
					\fi
				\fi
			\fi
		\fi
		%
		%
		%
		% Update limits:
		\E\ifpgfplots@update@limits@for@one@point@ISCLIPPED
		\E\else
			\ifpgfplots@autocompute@xmin
				\ifpgfplots@xislinear
					\E\pgfplotsmathfloatmin{\E\pgfplots@xmin}{##1}%
					\E\global\E\let\E\pgfplots@xmin=\E\pgfmathresult
				\else
					\E\pgfplotsmathmin{\E\pgfplots@xmin}{##1}%
					\E\global\E\let\E\pgfplots@xmin=\E\pgfmathresult
				\fi
			\fi
			\ifpgfplots@autocompute@xmax
				\ifpgfplots@xislinear
					\E\pgfplotsmathfloatmax{\E\pgfplots@xmax}{##1}%
					\E\global\E\let\E\pgfplots@xmax=\E\pgfmathresult
				\else
					\E\pgfplotsmathmax{\E\pgfplots@xmax}{##1}%
					\E\global\E\let\E\pgfplots@xmax=\E\pgfmathresult
				\fi
			\fi
			\ifpgfplots@autocompute@ymin
				\ifpgfplots@yislinear
					\E\pgfplotsmathfloatmin{\E\pgfplots@ymin}{##2}%
					\E\global\E\let\E\pgfplots@ymin=\E\pgfmathresult
				\else
					\E\pgfplotsmathmin{\E\pgfplots@ymin}{##2}%
					\E\global\E\let\E\pgfplots@ymin=\E\pgfmathresult
				\fi
			\fi
			\ifpgfplots@autocompute@ymax
				\ifpgfplots@yislinear
					\E\pgfplotsmathfloatmax{\E\pgfplots@ymax}{##2}%
					\E\global\E\let\E\pgfplots@ymax=\E\pgfmathresult
				\else
					\E\pgfplotsmathmax{\E\pgfplots@ymax}{##2}%
					\E\global\E\let\E\pgfplots@ymax=\E\pgfmathresult
				\fi
			\fi
			\ifpgfplots@curplot@threedim
				\ifpgfplots@autocompute@zmin
					\ifpgfplots@zislinear
						\E\pgfplotsmathfloatmin{\E\pgfplots@zmin}{##3}%
						\E\global\E\let\E\pgfplots@zmin=\E\pgfmathresult
					\else
						\E\pgfplotsmathmin{\E\pgfplots@zmin}{##3}%
						\E\global\E\let\E\pgfplots@zmin=\E\pgfmathresult
					\fi
				\fi
				\ifpgfplots@autocompute@zmax
					\ifpgfplots@zislinear
						\E\pgfplotsmathfloatmax{\E\pgfplots@zmax}{##3}%
						\E\global\E\let\E\pgfplots@zmax=\E\pgfmathresult
					\else
						\E\pgfplotsmathmax{\E\pgfplots@zmax}{##3}%
						\E\global\E\let\E\pgfplots@zmax=\E\pgfmathresult
					\fi
				\fi
			\fi
		\E\fi
		%
		% Compute data range:
		\ifpgfplots@autocompute@all@limits
			% the data range will be acquired simply from the axis
			% range, see below!
		\else
			% Attention: it is only done for linear axis!
			\ifpgfplots@xislinear
				\E\pgfplotsmathfloatmin{\E\pgfplots@data@xmin}{##1}%
				\E\global\E\let\E\pgfplots@data@xmin=\E\pgfmathresult
				\E\pgfplotsmathfloatmax{\E\pgfplots@data@xmax}{##1}%
				\E\global\E\let\E\pgfplots@data@xmax=\E\pgfmathresult
			\fi
			\ifpgfplots@yislinear
				\E\pgfplotsmathfloatmin{\E\pgfplots@data@ymin}{##2}%
				\E\global\E\let\E\pgfplots@data@ymin=\E\pgfmathresult
				\E\pgfplotsmathfloatmax{\E\pgfplots@data@ymax}{##2}%
				\E\global\E\let\E\pgfplots@data@ymax=\E\pgfmathresult
			\fi
			\ifpgfplots@curplot@threedim
				\ifpgfplots@zislinear
					\E\pgfplotsmathfloatmin{\E\pgfplots@data@zmin}{##3}%
					\E\global\E\let\E\pgfplots@data@zmin=\E\pgfmathresult
					\E\pgfplotsmathfloatmax{\E\pgfplots@data@zmax}{##3}%
					\E\global\E\let\E\pgfplots@data@zmax=\E\pgfmathresult
				\fi
			\fi
		\fi
%\E\pgfplots@message{Updated limits: (\E\pgfplots@xmin,\E\pgfplots@ymin) rectangle  (\E\pgfplots@xmax,\E\pgfplots@ymax).}%
%
%\E\tracingmacros=0\E\tracingcommands=0
	}%
	%
	% A routine which parses a coordinate.
	% Here, 'coordinate' means (x,y) for a two dimensional plot and
	% '(x,y,z)' for a three dimensional one.
	%
	% The preparation consists of
	% - filtering.
	% - coordinate parsing and high level transformations.
	% - logs.
	%
	% It might happen that \pgfplotsaxisparsecoordinate is called
	% multiple times for a single data "point" (for example, a quiver
	% point might call it for the point where the vector starts and
	% where the vector ends).
	%
	% @PRECONDITION
	% 	- the plot's survey phase is running (has already been started)
	% 	- \pgfplots@current@point@[xyz] contains the coordinates of the
	% 	point. I assume they are unparsed.
	%
	%
	% @POSTCONDITION
	% 	- the axis' state will be updated.
	% 	- the \pgfplots@current@point@[xyz] macros will contain parsed data.
	% 	- \ifpgfplotsaxisparsecoordinateok will be true if and only
	% 	if the data point has not been filtered away. If it has been
	% 	filtered away, \pgfplots@current@point@[xyz] will be empty.
	% 	- \pgfplots@current@point@[xyz]@unfiltered contain unparsed
	% 	data.
	%
	% @see \pgfplotsaxispreparecoordinate
	% @see \pgfplotsaxisdatapointsurveyed
	\xdef\pgfplotsaxisparsecoordinate@{%
		% These things are necessary for error bars and are available
		% as public results in math parser invocations (for meta and
		% filters)
		\E\let\E\pgfplots@current@point@x@unfiltered=\E\pgfplots@current@point@x
		\E\let\E\pgfplots@current@point@y@unfiltered=\E\pgfplots@current@point@y
		\E\let\E\pgfplots@current@point@z@unfiltered=\E\pgfplots@current@point@z
		\E\def\E\pgfplots@unbounded@dir{}%
		%
		\E\pgfplots@invoke@prefilter
		%
		\E\expandafter\E\pgfplots@prepare@xcoord\E\expandafter{\E\pgfplots@current@point@x}%
		\E\expandafter\E\pgfplots@invoke@filter\E\expandafter{\E\pgfmathresult}{x}%
		\E\let\E\pgfplots@current@point@x=\E\pgfmathresult
		%
		\E\expandafter\E\pgfplots@prepare@ycoord\E\expandafter{\E\pgfplots@current@point@y}%
		\E\expandafter\E\pgfplots@invoke@filter\E\expandafter{\E\pgfplots@current@point@y}{y}%
		\E\let\E\pgfplots@current@point@y=\E\pgfmathresult
		%
		\ifpgfplots@curplot@threedim
			\E\expandafter\E\pgfplots@prepare@zcoord\E\expandafter{\E\pgfplots@current@point@z}%
			\E\expandafter\E\pgfplots@invoke@filter\E\expandafter{\E\pgfplots@current@point@z}{z}%
			\E\let\E\pgfplots@current@point@z=\E\pgfmathresult
		\fi
		%
		\E\pgfplots@invoke@filter@xyz
		%
		\E\ifx\E\pgfplots@current@point@x\E\pgfutil@empty
		\E\else
			% parse for numbers. Note that this might cause
			% unnecessary overhead of logs (which are already
			% normalized unless someone provided filters). But do it
			% anyway to ensure that filters produce valid output.
			\E\pgfplotscoordmath{x}{parsenumber}{\E\pgfplots@current@point@x}%
			\E\let\E\pgfplots@current@point@x=\E\pgfmathresult
			\E\pgfplotscoordmath{x}{if is bounded}{\E\pgfplots@current@point@x}%
				{}%
				{% this clears nan, inf and -inf points.
					\E\let\E\pgfplots@current@point@x=\E\pgfutil@empty
					\E\def\E\pgfplots@unbounded@dir{x}%
				}%
		\E\fi
		%
		\E\ifx\E\pgfplots@current@point@y\E\pgfutil@empty
		\E\else
			\E\pgfplotscoordmath{y}{parsenumber}{\E\pgfplots@current@point@y}%
			\E\let\E\pgfplots@current@point@y=\E\pgfmathresult
			\E\pgfplotscoordmath{y}{if is bounded}{\E\pgfplots@current@point@y}%
				{}%
				{% this clears nan, inf and -inf points.
					\E\let\E\pgfplots@current@point@y=\E\pgfutil@empty
					\E\def\E\pgfplots@unbounded@dir{y}%
				}%
		\E\fi
		%
		\ifpgfplots@curplot@threedim
			%
			\E\ifx\E\pgfplots@current@point@z\E\pgfutil@empty
			\E\else
				\E\pgfplotscoordmath{z}{parsenumber}{\E\pgfplots@current@point@z}%
				\E\let\E\pgfplots@current@point@z=\E\pgfmathresult
				\E\pgfplotscoordmath{z}{if is bounded}{\E\pgfplots@current@point@z}%
					{}%
					{% this clears nan, inf and -inf points.
						\E\let\E\pgfplots@current@point@z=\E\pgfutil@empty
						\E\def\E\pgfplots@unbounded@dir{z}%
					}%
			\E\fi
		\fi
		%
		% check if coordinates are bounded:
		\E\pgfplotsaxisparsecoordinateoktrue
		\E\ifx\E\pgfplots@current@point@x\E\pgfutil@empty
			\E\pgfplotsaxisparsecoordinateokfalse
		\E\else
			\E\ifx\E\pgfplots@current@point@y\E\pgfutil@empty
				\E\pgfplotsaxisparsecoordinateokfalse
			\E\else
				\ifpgfplots@curplot@threedim
					\E\ifx\E\pgfplots@current@point@z\E\pgfutil@empty
						\E\pgfplotsaxisparsecoordinateokfalse
					\E\fi
				\fi
			\E\fi
		\E\fi
		%
	}%
	%
	\endgroup
}%

% ==================================
% The quiver plot handler.
% It draws a lot of arrows.
% Its input is (x_i,y_i); (u_i,v_i) for data point i and it draws a
% vector in direction (u_i,v_i) starting from (x_i,y_i) .
% It also supports 3D arrows (involving z_i and w_i).

\newif\ifpgfplots@quiver@usetikz
\newif\ifpgfplots@quiver@updatelimits

\pgfplotsset{
	% The 'quiver' plot handler for two- and three dimensional plots.
	%
	% User Interface:
	% use /pgfplots/quiver to enable the plot handler.
	% Then, provide `quiver/u value' or `quiver/u' to
	% tell where to find the 'x' coordinates of the vectors, and similarly
	% for 'v' and 'w' instead of 'u'.
	quiver/.code={%
		\let\tikz@plot@handler=\pgfplotsplothandlerquiver
		\pgfqkeys{/pgfplots/quiver}{quiver legend,#1}%
	},%
	quiver/u value*/.initial=0,
	quiver/v value*/.initial=0,
	quiver/w value*/.initial=0,
	quiver/u value is expr/.initial=0,
	quiver/v value is expr/.initial=0,
	quiver/w value is expr/.initial=0,
	quiver/quiver legend/.style={
		/pgfplots/legend image code/.code={%
			\draw[x=0.6cm,y=0cm,z=0pt,##1,
				/pgfplots/quiver/before arrow/.add code={}{
					% quiver is a pgfplots-specific plot-handler. We need to
					% fix the additional input data somehow.
					%
					% this is an *absolute* coordinate, interpreted
					% relative to 'x', 'y', 'z'
					\def\pgfplots@quiver@u{1}%
					\def\pgfplots@quiver@v{1}%
					\def\pgfplots@quiver@w{1}%
				},%
			]
				plot coordinates {
					(0cm,0cm)
				};%
		}%
	},
	quiver/u filter/.code=,
	quiver/v filter/.code=,
	quiver/w filter/.code=,
	quiver/u filter/.expression/.code=\pgfplots@install@filter@expression{quiver/u filter}{#1},
	quiver/v filter/.expression/.code=\pgfplots@install@filter@expression{quiver/v filter}{#1},
	quiver/w filter/.expression/.code=\pgfplots@install@filter@expression{quiver/w filter}{#1},
	quiver/u value/.code	=\pgfplots@set@source@for{quiver/u}{#1}{0},%
	quiver/u/.code			=\pgfplots@set@source@for{quiver/u}{#1}{1},%
	quiver/v value/.code	=\pgfplots@set@source@for{quiver/v}{#1}{0},%
	quiver/v/.code			=\pgfplots@set@source@for{quiver/v}{#1}{1},%
	quiver/w value/.code	=\pgfplots@set@source@for{quiver/w}{#1}{0},%
	quiver/w/.code			=\pgfplots@set@source@for{quiver/w}{#1}{1},%
	quiver/before arrow/.code=,
	quiver/after arrow/.code=,
	quiver/every arrow/.style={},
	quiver/arrow color/.initial=,
	quiver/scale arrows/.initial=1,
	quiver/update limits/.is if=pgfplots@quiver@updatelimits,
	quiver/update limits=true,
	quiver/colored/.code={%
		\def\pgfplots@loc@TMPa{#1}%
		\ifx\pgfplots@loc@TMPa\pgfutil@empty
		\else
			\pgfkeyslet{/pgfplots/quiver/arrow color}\pgfplots@loc@TMPa
			\pgfkeysalso{/pgfplots/set point meta if empty=f(x)}%
		\fi
	},%
	quiver/colored/.default=mapped color,
}%

\def\pgfplots@set@source@for#1#2#3{%
	\pgfkeyssetvalue{/pgfplots/#1 value*}{#2}%
	\pgfkeyssetvalue{/pgfplots/#1 is expr}{#3}%
}%
% To be used to create a simple parser for keys initialised by
% \pgfplots@set@source@for:
%
% #1: the key path (relative to /pgfplots/) of the data.
%     Can be empty in which case /pgfplots/#2 is used to access data.
% #2: the key name of the data
% #3: a macro name which be will defined to be a parser for the data.
%
% The parser will check whether the '#2 is expr' key is set.
% Furthermore, it defines /data point/#2 to be the result.
%
% Example:
% \pgfplots@set@source@for{hist/data}{...}
%
% ->
%  \pgfplots@prepare@source@parser@for{hist/}{data}\parser
%
%  then, invoking \parser
%  will define \pgfmathresult to be the argument provided to
%  \pgfplots@set@source@for.
\def\pgfplots@prepare@source@parser@for#1#2#3{%
	\pgfkeyslet{/data point/#1#2}\pgfutil@empty%
	%
	\pgfkeysgetvalue{/pgfplots/#1#2 value*}\pgfplots@loc@TMPa
	\ifx\pgfplots@loc@TMPa\pgfutil@empty
		% assume the '/data point/#1#2' is set by some input
		% routine.
		% Invoke math parser in this case.
		\pgfkeyssetvalue{/pgfplots/#1#2 is expr}{1}%
	\else
		\pgfkeyslet{/data point/#1#2}\pgfplots@loc@TMPa
	\fi
	\def#3{%
		\edef\pgfmathresult{\pgfkeysvalueof{/data point/#1#2}}%
%\message{parse coordinate #1#2 (\pgfmathresult) ...^^J}%
	}%
	%
	%
	\pgfkeysifdefined{/pgfplots/#1#2 coord trafo/.@cmd}{%
		\pgfkeysgetvalue{/pgfplots/#1#2 coord trafo/.@cmd}\pgfplots@loc@TMPa
		\ifx\pgfplots@loc@TMPa\pgfplots@empty@command@key
		\else
			\t@pgfplots@toka=\expandafter{#3}%
			\t@pgfplots@tokb={%
				\def\pgfplots@loc@TMPa{\pgfkeysvalueof{/pgfplots/#1#2 coord trafo/.@cmd}}%
				\expandafter\pgfplots@loc@TMPa\expandafter{\pgfmathresult}\pgfeov%
			}%
			\t@pgfplots@tokc={%
				\ifx\pgfmathresult\pgfutil@empty
				\else
					\pgfmathfloatparsenumber{\pgfmathresult}%
				\fi
			}%
			\edef#3{%
				\the\t@pgfplots@toka
				\the\t@pgfplots@tokb
				\the\t@pgfplots@tokc
			}%
		\fi
	}{}%
	%
	\t@pgfplots@toka=\expandafter{#3}%
	\if1\pgfkeysvalueof{/pgfplots/#1#2 is expr}%
		\t@pgfplots@tokc={%
			\ifx\pgfmathresult\pgfutil@empty
			\else
				\pgfmathparse{\pgfmathresult}%
			\fi
		}%
	\else
		\t@pgfplots@tokc={%
			\ifx\pgfmathresult\pgfutil@empty
			\else
				\pgfmathfloatparsenumber{\pgfmathresult}%
			\fi
		}%
	\fi
	\edef#3{%
		\the\t@pgfplots@toka
		\the\t@pgfplots@tokc
	}%
	%
	\pgfkeysgetvalue{/pgfplots/#1#2 filter/.@cmd}\pgfplots@loc@TMPa
	\ifx\pgfplots@loc@TMPa\pgfplots@empty@command@key
	\else
		\t@pgfplots@toka=\expandafter{#3}%
		\t@pgfplots@tokb={%
			\def\pgfplots@loc@TMPa{\pgfkeysvalueof{/pgfplots/#1#2 filter/.@cmd}}%
			\expandafter\pgfplots@loc@TMPa\expandafter{\pgfmathresult}\pgfeov%
		}%
		\t@pgfplots@tokc={%
			\ifx\pgfmathresult\pgfutil@empty
			\else
				\pgfmathfloatparsenumber{\pgfmathresult}%
			\fi
		}%
		\edef#3{%
			\the\t@pgfplots@toka
			\the\t@pgfplots@tokb
			\the\t@pgfplots@tokc
		}%
	\fi
	%
	\iftrue
		\t@pgfplots@toka=\expandafter{#3}%
		\t@pgfplots@tokc={%
%\message{parse coordinate (#1#2) = \pgfmathresult.^^J}%
		}%
		\edef#3{%
			\the\t@pgfplots@toka
			\the\t@pgfplots@tokc
		}%
	\fi
}%

% Invokes /pgfplots/#1 coord inv trafo  on \pgfmathresult if that key
% exists.
% #1 the argument is in float (will become the 'default' coordmath
% eventually).
%
% On output, it should either '#1' if there was no coord inv trafo or
% the result of the trafo.
\def\pgfplots@coord@trafo@inv@for#1{%
	\def\pgfplots@loc@TMPb{/pgfplots/#1 coord inv trafo/.@cmd}%
	\pgfkeysifdefined{\pgfplots@loc@TMPb}{%
		\pgfkeysgetvalue{\pgfplots@loc@TMPb}\pgfplots@loc@TMPa
		\ifx\pgfplots@loc@TMPa\pgfplots@empty@command@key
		\else
			\pgfmathfloattofixed{\pgfmathresult}%
			\expandafter\pgfplots@loc@TMPa\expandafter{\pgfmathresult}\pgfeov
		\fi
	}{}%
}
% Like \pgfplots@coord@trafo@inv@for, but for the normal trafo
% direction
\def\pgfplots@coord@trafo@for#1{%
	\def\pgfplots@loc@TMPb{/pgfplots/#1 coord trafo/.@cmd}%
	\pgfkeysifdefined{\pgfplots@loc@TMPb}{%
		\pgfkeysgetvalue{\pgfplots@loc@TMPb}\pgfplots@loc@TMPa
		\ifx\pgfplots@loc@TMPa\pgfplots@empty@command@key
		\else
			\expandafter\pgfplots@loc@TMPa\expandafter{\pgfmathresult}\pgfeov
		\fi
	}{}%
}
\def\pgfplotsplothandlerquiver{%
	\pgfplotsresetplothandler
	\let\pgf@plotstreamstart=\pgfplotsplothandlervisbegin@quiver
	\let\pgfplotsplothandlersurveystart=\pgfplotsplothandlersurveystart@quiver
	\let\pgfplotsplothandlersurveypoint=\pgfplotsplothandlersurveypoint@quiver
	\let\pgfplotsplothandlerserializepointto=\pgfplotsplothandlerserializepointto@quiver
	\let\pgfplotsplothandlerdeserializepointfrom=\pgfplotsplothandlerdeserializepointfrom@quiver
	\let\pgfplotsplothandlerpointtokeys=\pgfplotsplothandlerpointtokeys@quiver
	\let\pgfplotsplothandlerquiver@vis@hook=\pgfutil@empty
	\def\pgfplotsplothandlername{quiver}%
	%
	\ifpgfplots@xislinear \else \pgfplotsplothandlerquivererror \fi
	\ifpgfplots@yislinear \else \pgfplotsplothandlerquivererror\fi
	\pgfplotsifcurplotthreedim{%
		\ifpgfplots@zislinear \else \pgfplotsplothandlerquivererror \fi
	}{}%
}%
\def\pgfplotsplothandlerquivererror{\pgfplots@error{Sorry, quiver plots for logarithmic axes are not yet implemented. In fact, the implementation does something -- but it will probably change in future releases. Contact the mailing list if you have questions}}%

\def\pgfplotsplothandlersurveystart@quiver{%
	\pgfkeysgetvalue{/pgfplots/quiver/scale arrows}\pgfplots@quiver@scale
	\ifx\pgfplots@quiver@scale\pgfutil@empty
	\else
		\def\pgfplots@loc@TMPa{1}%
		\ifx\pgfplots@loc@TMPa\pgfplots@quiver@scale
			\let\pgfplots@quiver@scale=\pgfutil@empty
		\else
			\pgfmathparse{\pgfplots@quiver@scale}%
			\pgfmathfloatparsenumber\pgfplots@quiver@scale
			\let\pgfplots@quiver@scale=\pgfmathresult
		\fi
	\fi
	%
	\pgfplots@prepare@source@parser@for@quiver u\pgfplots@quiver@prepare@u%
	\pgfplots@prepare@source@parser@for@quiver v\pgfplots@quiver@prepare@v%
	\pgfplots@prepare@source@parser@for@quiver w\pgfplots@quiver@prepare@w%
}%

\def\pgfplots@prepare@source@parser@for@quiver#1#2{%
	\pgfplots@prepare@source@parser@for{quiver/}{#1}{#2}%
	\t@pgfplots@toka=\expandafter{#2}%
	\t@pgfplots@tokb=\expandafter{\csname pgfplots@quiver@#1\endcsname}%
	\edef#2{%
		\the\t@pgfplots@toka
		\noexpand\let\the\t@pgfplots@tokb=\noexpand\pgfmathresult
		\ifx\pgfplots@quiver@scale\pgfutil@empty
		\else
			\noexpand\pgfmathfloatmultiply@{\pgfplots@quiver@scale}{\the\t@pgfplots@tokb}%
			\noexpand\let\the\t@pgfplots@tokb=\noexpand\pgfmathresult
		\fi
	}%
}%

\def\pgfplotsplothandlersurveypoint@quiver{%
	\pgfplots@quiver@prepare@u
	\pgfplots@quiver@prepare@v
	\pgfplotsifcurplotthreedim{%
		\pgfplots@quiver@prepare@w
	}{%
		\let\pgfplots@quiver@w=\pgfutil@empty
	}%
	\pgfplotsaxisparsecoordinate
	\pgfplotsaxispreparecoordinate
	\ifpgfplotsaxisparsecoordinateok
		\pgfplotsaxisupdatelimitsforcoordinate\pgfplots@current@point@x\pgfplots@current@point@y\pgfplots@current@point@z
		%
		\pgfmathadd@{\pgfplots@quiver@u}{\pgfplots@current@point@x}%
		\let\pgfplots@quiver@u=\pgfmathresult
		\pgfmathadd@{\pgfplots@quiver@v}{\pgfplots@current@point@y}%
		\let\pgfplots@quiver@v=\pgfmathresult
		\pgfplotsifcurplotthreedim{%
			\pgfmathadd@{\pgfplots@quiver@w}{\pgfplots@current@point@z}%
			\let\pgfplots@quiver@w=\pgfmathresult
		}{}%
		\ifpgfplots@quiver@updatelimits
			\pgfplotsaxisupdatelimitsforcoordinate\pgfplots@quiver@u\pgfplots@quiver@v\pgfplots@quiver@w
		\fi
	\fi
	\pgfplotsaxisdatapointsurveyed
}%
\def\pgfplotsplothandlerserializepointto@quiver#1{%
	\edef#1{\pgfplots@current@point@x,\pgfplots@current@point@y,\pgfplots@current@point@z>\pgfplots@quiver@u,\pgfplots@quiver@v,\pgfplots@quiver@w}%
}%
\def\pgfplotsplothandlerdeserializepointfrom@quiver#1{%
	\expandafter\pgfplotsplothandlerdeserializepointfrom@quiver@#1\relax
}%
\def\pgfplotsplothandlerdeserializepointfrom@quiver@#1,#2,#3>#4,#5,#6\relax{%
	\def\pgfplots@current@point@x{#1}%
	\def\pgfplots@current@point@y{#2}%
	\def\pgfplots@current@point@z{#3}%
	\def\pgfplots@quiver@u{#4}%
	\def\pgfplots@quiver@v{#5}%
	\def\pgfplots@quiver@w{#6}%
}%
\def\pgfplotsplothandlerpointtokeys@quiver#1{%
	\pgfplotsplothandlerpointtokeys@default
	\pgfkeyslet{#1/u}\pgfplots@quiver@u
	\pgfkeyslet{#1/v}\pgfplots@quiver@v
	\pgfkeyslet{#1/w}\pgfplots@quiver@w
}%
\def\pgfplotsplothandlervisbegin@quiver{%
	\def\pgfplots@quiver@has@handled@point@meta{0}%
	\pgfkeysgetvalue{/pgfplots/quiver/arrow color}\pgfplots@quiver@color
	\ifx\pgfplots@quiver@color\pgfutil@empty
	\else
		% prepare the color data and define 'mapped color':
		\def\pgfplots@quiver@has@handled@point@meta{1}%
		\expandafter\def\expandafter\pgfplotsplothandlerquiver@vis@hook\expandafter{%
			\pgfplotsplothandlerquiver@vis@hook
			\pgfplotsaxisvisphasetransformpointmeta
			\pgfplotscolormapdefinemappedcolor{\pgfplotspointmetatransformed}%
		}%
		% SEE BELOW AS WELL FOR HOW TO ENABLE THE COLOR.
	\fi
	%
	%
	\pgfkeysgetvalue{/pgfplots/quiver/every arrow/.@cmd}\pgfplots@quiver@everyarrow
	\ifx\pgfplots@quiver@everyarrow\pgfplots@empty@style@key
		% use PGF basic level methods to set the 'arrow color':
		\ifx\pgfplots@quiver@color\pgfutil@empty
		\else
			\expandafter\def\expandafter\pgfplotsplothandlerquiver@vis@hook\expandafter{%
				\pgfplotsplothandlerquiver@vis@hook
				\pgfsetstrokecolor{\pgfkeysvalueof{/pgfplots/quiver/arrow color}}%
				% for arrow heads:
				\pgfsetfillcolor{\pgfkeysvalueof{/pgfplots/quiver/arrow color}}%
			}%
		\fi
	\else
		% 'every arrow' should provide a high level user interface.
		% Use tikz instead of pgf. This is slower, but more powerful.
		\pgfplots@quiver@usetikztrue
		\pgfplotsaxisifhaspointmeta{%
			% ASSERT(mapped color is available)
			\if0\pgfplots@quiver@has@handled@point@meta%
				% -> define mapped color
				\expandafter\def\expandafter\pgfplotsplothandlerquiver@vis@hook\expandafter{%
					\pgfplotsplothandlerquiver@vis@hook
					\pgfplotsaxisvisphasetransformpointmeta
					\pgfplotscolormapdefinemappedcolor{\pgfplotspointmetatransformed}%
				}%
				\def\pgfplots@quiver@has@handled@point@meta{1}%
			\fi
		}{}%
		% use tikz methods to set the 'arrow color':
		\ifx\pgfplots@quiver@color\pgfutil@empty
		\else
			\t@pgfplots@toka=\expandafter{\pgfplots@quiver@color}%
			\edef\pgfplots@loc@TMPa{\noexpand\pgfkeysalso{/pgfplots/quiver/every arrow/.prefix style={\the\t@pgfplots@toka}}}%
			\pgfplots@loc@TMPa
		\fi
	\fi
	%
	\global\let\pgf@plotstreampoint=\pgfplotsplothandlerquiver@vis%
	\global\let\pgf@plotstreamspecial=\pgfutil@gobble%
	\global\let\pgf@plotstreamend=\relax
}%

\def\pgfplotsplothandlerquiver@vis#1{%
	\pgfkeysvalueof{/pgfplots/quiver/before arrow/.@cmd}\pgfeov
	\pgfplotsplothandlerquiver@vis@hook
	\ifpgfplots@quiver@usetikz
		\edef\pgfplotsplothandler@quiver@point{\global\pgf@x=\the\pgf@x\space\global\pgf@y=\the\pgf@y\space}%
		\draw[/pgfplots/quiver/every arrow] \pgfextra{\pgfplotsplothandlerquiver@vis@path{\pgfplotsplothandler@quiver@point}};
	\else
		\pgfplotsplothandlerquiver@vis@path{#1}%
		\pgfusepath{stroke}%
	\fi
	\pgfkeysvalueof{/pgfplots/quiver/after arrow/.@cmd}\pgfeov
}%
\def\pgfplotsplothandlerquiver@vis@path#1{%
	\pgfpathmoveto{#1}%
	\pgfplotsaxisvisphasetransformcoordinate\pgfplots@quiver@u\pgfplots@quiver@v\pgfplots@quiver@w
	\pgfpathlineto{%
		\pgfplotsifcurplotthreedim{%
			\pgfplotsqpointxyz\pgfplots@quiver@u\pgfplots@quiver@v\pgfplots@quiver@w
		}{%
			\pgfplotsqpointxy\pgfplots@quiver@u\pgfplots@quiver@v
		}%
	}%
}%

\newif\ifpgfplotsplothandlerhistogram@intervals
\newif\ifpgfplotsplothandlerhistogram@cumulative
\newif\ifpgfplotsplothandlerhistogram@density
\pgfplotsset{
	hist/.code={%
		\let\tikz@plot@handler=\pgfplotsplothandlerhistogram
		\pgfqkeys{/pgfplots/hist}{#1}%
	},
	hist/data value/.code	=\pgfplots@set@source@for{hist/data}{#1}{0},%
	hist/data/.code			=\pgfplots@set@source@for{hist/data}{#1}{1},%
	hist/data filter/.code=,
	hist/data filter/.expression/.code=\pgfplots@install@filter@expression{hist/data filter}{#1},
	hist/data value=\pgfkeysvalueof{/data point/y},
%	hist/data=y,
	hist/data min/.initial=\pgfkeysvalueof{/pgfplots/xmin},
	hist/data max/.initial=\pgfkeysvalueof{/pgfplots/xmax},
	hist/bins/.initial=10,
	hist/intervals/.is if=pgfplotsplothandlerhistogram@intervals,
	hist/intervals/.default=true,
	hist/intervals=true,
	hist/cumulative/.is if=pgfplotsplothandlerhistogram@cumulative,
	hist/cumulative/.default=true,
	hist/density/.is if=pgfplotsplothandlerhistogram@density,
	hist/density/.default=true,
	hist/density=false,
	hist/handler/.style={/tikz/ybar interval},
	hist/symbolic coords/.style={%
		/pgfplots/symbolic coords={hist/data}{#1},
		/pgfplots/symbolic coords={x}{#1},
	},%
}%
\def\pgfplotsplothandlerhistogram{%
	\pgfplotsresetplothandler
	\def\pgf@plotstreamstart{%
		\pgfplotsset{/pgfplots/hist/handler}%
		\pgfplotsresetplothandler
		\tikz@plot@handler
		\pgf@plotstreamstart
	}%
	%
	% let \pgfplotsplothandlername to the one of the handler.
	% Note that \pgfplotsplothandlername is the *visualization* layer
	\begingroup
		\pgfplotsset{/pgfplots/hist/handler}%
		\pgfplotsresetplothandler
		\tikz@plot@handler
		\xdef\pgfplots@glob@TMPa{\pgfplotsplothandlername}%
	\endgroup
	\let\pgfplotsplothandlername=\pgfplots@glob@TMPa%
	%
	\let\pgfplotsplothandlersurveypoint=\pgfplotsplothandlersurveypoint@hist
	\let\pgfplotsplothandlersurveystart=\pgfplotsplothandlersurveystart@hist
	\let\pgfplotsplothandlersurveyend=\pgfplotsplothandlersurveyend@hist
}%
\def\pgfplotsplothandlersurveystart@hist{%
	\pgfplots@prepare@source@parser@for{hist/}{data}{\pgfplotsplothandlerhistogram@parse}%
	%
	\pgfkeysgetvalue{/pgfplots/hist/data min}\pgfmathresult
	\edef\pgfmathresult{\pgfmathresult}%
	\ifx\pgfmathresult\pgfutil@empty\else
		\pgfplots@coord@trafo@for{hist/data}%
	\fi
	\edef\pgfplotsplothandlerhistogram@datamin{\pgfmathresult}%
	%
	\pgfkeysgetvalue{/pgfplots/hist/data max}\pgfmathresult
	\edef\pgfmathresult{\pgfmathresult}%
	\ifx\pgfmathresult\pgfutil@empty\else
		\pgfplots@coord@trafo@for{hist/data}%
	\fi
	\edef\pgfplotsplothandlerhistogram@datamax{\pgfmathresult}%
	%
	\ifx\pgfplotsplothandlerhistogram@datamin\pgfutil@empty
		\pgfmathfloatcreate{1}{1.0}{2147483645}%
		\let\pgfplotsplothandlerhistogram@datamin=\pgfmathresult
		\def\pgfplotsplothandlerhistogram@datamin@autocompute{1}%
	\else
		\pgfmathfloatparsenumber{\pgfplotsplothandlerhistogram@datamin}%
		\let\pgfplotsplothandlerhistogram@datamin=\pgfmathresult
		\def\pgfplotsplothandlerhistogram@datamin@autocompute{0}%
	\fi
	\ifx\pgfplotsplothandlerhistogram@datamax\pgfutil@empty
		\pgfmathfloatcreate{2}{1.0}{2147483645}%
		\let\pgfplotsplothandlerhistogram@datamax=\pgfmathresult
		\def\pgfplotsplothandlerhistogram@datamax@autocompute{1}%
	\else
		\pgfmathfloatparsenumber{\pgfplotsplothandlerhistogram@datamax}%
		\let\pgfplotsplothandlerhistogram@datamax=\pgfmathresult
		\def\pgfplotsplothandlerhistogram@datamax@autocompute{0}%
	\fi
	%
	\edef\pgfplotsplothandlerhistogram@Nfixed{\pgfkeysvalueof{/pgfplots/hist/bins}}%
	\c@pgf@counta=\pgfplotsplothandlerhistogram@Nfixed\relax
	\advance\c@pgf@counta by-1
	\edef\pgfplotsplothandlerhistogram@Nmax{\the\c@pgf@counta}%
	%
	\pgfmathfloatparsenumber{\pgfplotsplothandlerhistogram@Nfixed}%
	\let\pgfplotsplothandlerhistogram@N=\pgfmathresult
	%
	\pgfplotsapplistXnewempty\pgfp@hist@@
	\def\c@pgfplotsplothandlerhistogram@num{0}%
	\c@pgfplots@coordindex=0
}%

\def\pgfplotsplothandlersurveypoint@hist@limits{%
	\if1\pgfplotsplothandlerhistogram@datamin@autocompute
		\pgfplotsmathfloatmin{\pgfplots@current@point@data}{\pgfplotsplothandlerhistogram@datamin}%
		\let\pgfplotsplothandlerhistogram@datamin=\pgfmathresult
	\fi
	%
	\if1\pgfplotsplothandlerhistogram@datamax@autocompute
		\pgfplotsmathfloatmax{\pgfplots@current@point@data}{\pgfplotsplothandlerhistogram@datamax}%
		\let\pgfplotsplothandlerhistogram@datamax=\pgfmathresult
	\fi
}%
\def\pgfplotsplothandlersurveypoint@hist{%
	% Note that at this point, the coordinate filtering does NOT
	% apply. Perhaps it should...
	\pgfplotsplothandlerhistogram@parse
	\let\pgfplots@current@point@data=\pgfmathresult
	%
	\ifx\pgfplots@current@point@data\pgfutil@empty
	\else
		\pgfmathfloatiffinite\pgfplots@current@point@data{%
			\pgfplotsplothandlersurveypoint@hist@limits
			%
			\pgfplotsutil@advancestringcounter\c@pgfplotsplothandlerhistogram@num
			%
			% store parsed result.
			\edef\pgfmathresult{{\pgfplots@current@point@data}}%
			\expandafter\pgfplotsapplistXpushback\expandafter{\pgfmathresult}\to\pgfp@hist@@
		}{%
		}%
	\fi
	\advance\c@pgfplots@coordindex by1
}%
\def\pgfplotsplothandlersurveyend@hist{%
	\ifnum\c@pgfplotsplothandlerhistogram@num>0
		\expandafter\pgfplotsplothandlersurveyend@hist@
	\fi
}%
\def\pgfplotsplothandlersurveyend@hist@{%
	\pgfmathfloatsubtract@{\pgfplotsplothandlerhistogram@datamax}{\pgfplotsplothandlerhistogram@datamin}%
	\let\pgfplotsplothandlerhistogram@range=\pgfmathresult
	\pgfmathfloatdivide@{\pgfplotsplothandlerhistogram@range}{\pgfplotsplothandlerhistogram@N}%
	\let\pgfplotsplothandlerhistogram@h=\pgfmathresult
	\pgfmathfloatreciprocal@{\pgfplotsplothandlerhistogram@h}%
	\let\pgfplotsplothandlerhistogram@invh=\pgfmathresult
	%
	\pgfplotsarraynewempty{pgfp@hist}%
	\pgfplotsarrayresize{pgfp@hist}{\pgfplotsplothandlerhistogram@Nfixed}%
	\pgfplotsarrayforeachungrouped{pgfp@hist}\as\pgfplots@hist@count{%
		\pgfplotsarrayset{\pgfplotsarrayforeachindex}\of{pgfp@hist}\to{0}%
	}%
	%
	\pgfplotsapplistXlet\pgfplots@hist@data=\pgfp@hist@@
	\pgfplotsapplistXnewempty\pgfp@hist@@
	\expandafter\pgfplotsplothandlersurveyend@hist@loop\pgfplots@hist@data\pgfplots@EOI
	\let\pgfplots@hist@data=\relax
	%

	% Calculate total count
	\c@pgf@counta=0
	\pgfplotsarrayforeachungrouped{pgfp@hist}\as\pgfplots@hist@count{%
		\advance\c@pgf@counta by\pgfplots@hist@count\relax
		\def\pgfplots@loc@TMPa{\pgfplotsarrayset{\pgfplotsarrayforeachindex}\of{pgfp@hist}\to}%
	}%
	\pgfmathfloatparsenumber{\the\c@pgf@counta}%
	\let\pgfp@hist@totalcount=\pgfmathresult
	\pgfmathfloatreciprocal@{\pgfp@hist@totalcount}%
	\let\pgfp@hist@totalcount@inv=\pgfmathresult
	%
	\ifpgfplotsplothandlerhistogram@cumulative
		\c@pgf@counta=0
		\pgfplotsarrayforeachungrouped{pgfp@hist}\as\pgfplots@hist@count{%
			\advance\c@pgf@counta by\pgfplots@hist@count\relax
			\def\pgfplots@loc@TMPa{\pgfplotsarrayset{\pgfplotsarrayforeachindex}\of{pgfp@hist}\to}%
			\ifpgfplotsplothandlerhistogram@density
				\pgfmathfloatparsenumber{\the\c@pgf@counta}%
				\pgfmathfloatmultiply@{\pgfmathresult}{\pgfp@hist@totalcount@inv}%
				\pgfmathfloattosci@{\pgfmathresult}%
				\expandafter\pgfplots@loc@TMPa\expandafter{\pgfmathresult}%
			\else
				\expandafter\pgfplots@loc@TMPa\expandafter{\the\c@pgf@counta}%
			\fi
		}%
	\fi%
	%
	%% Density histogram
	% Divide count in each bin by (totalcount*range/bins)
	\ifpgfplotsplothandlerhistogram@density
		\ifpgfplotsplothandlerhistogram@cumulative
		\else
			\pgfmathfloatmultiply@{\pgfp@hist@totalcount@inv}{\pgfplotsplothandlerhistogram@invh}%
			\let\pgfp@hist@totalcount@times@h@inv=\pgfmathresult
% FIXME : this here is a patch suggestion for
% https://sourceforge.net/tracker/?func=detail&atid=1060656&aid=3609245&group_id=224188
%
% FIXME : this line would actually compute relative frequencies...
% might not be too bad at all, but is no density
%\let\pgfp@hist@totalcount@times@h@inv=\pgfp@hist@totalcount@inv
			\pgfplotsarrayforeachungrouped{pgfp@hist}\as\pgfplots@hist@count{%
				\pgfmathfloatparsenumber{\pgfplots@hist@count}%
				\pgfmathfloatmultiply@{\pgfmathresult}{\pgfp@hist@totalcount@times@h@inv}%
				\pgfmathfloattosci@{\pgfmathresult}%
				\def\pgfplots@loc@TMPa{\pgfplotsarrayset{\pgfplotsarrayforeachindex}\of{pgfp@hist}\to}%
				\expandafter\pgfplots@loc@TMPa\expandafter{\pgfmathresult}%
			}%
		\fi
	\fi
	%
	%% End density histogram
	\pgfplots@curplot@threedimfalse
	%
	\pgfplotsset{/pgfplots/hist/handler}%
	\pgfplotsresetplothandler
	\tikz@plot@handler
	%
	\pgfplotsplothandlersurveystart
	%
	\let\pgfplots@current@point@z=\pgfutil@empty
	\pgfplotsarrayforeachungrouped{pgfp@hist}\as\pgfplots@hist@count{%
		\pgfplotsplothandlerhistgetintervalstartfor\pgfplotsarrayforeachindex
		\pgfplotsplothandlerhist@invtrafo
		\let\pgfplots@current@point@x\pgfmathresult%
		\let\pgfplots@current@point@y\pgfplots@hist@count%
%\message{Survey point (\pgfplots@current@point@x,\pgfplots@current@point@y)^^J}%
		\pgfplotsplothandlersurveypoint
	}%
	\ifpgfplotsplothandlerhistogram@intervals
		% replicate last count.
		\let\pgfmathresult\pgfplotsplothandlerhistogram@datamax%
		\pgfplotsplothandlerhist@invtrafo
		\let\pgfplots@current@point@x\pgfmathresult%
		\let\pgfplots@current@point@y\pgfplots@hist@count%
%\message{Survey point (\pgfplots@current@point@x,\pgfplots@current@point@y)^^J}%
		\pgfplotsplothandlersurveypoint
	\fi
	%
	\pgfplotsplothandlersurveyend
}
\def\pgfplotsplothandlerhist@invtrafo{%
	% This here might be inefficient, because
	% there needs to be a compatible "x coord trafo" as well -- and
	% that transformation will (most likely) do the very same thing as
	% the hist/data coord trafo.
	%
	% But I did not find a way to combine the transformations
	% automatically without resorting to hackery.
	%
	% And: the performance impact might (hopefully) be small...
	\pgfplots@coord@trafo@inv@for{hist/data}%
}%
\def\pgfplotsplothandlersurveyend@hist@loop#1{%
	\def\pgfplots@loc@TMPa{#1}%
	\ifx\pgfplots@loc@TMPa\pgfplots@EOI
	\else
		\pgfplotsplothandlerhistgetbinfor@{#1}%
		\expandafter\pgfplotsplothandlerhistadvancebin\expandafter{\pgfmathresult}%
		%
		\expandafter\pgfplotsplothandlersurveyend@hist@loop
	\fi
}%
\def\pgfplotsplothandlerhistadvancebin#1{%
	\pgfplotsarrayselect{#1}\of{pgfp@hist}\to\pgfplots@loc@TMPa%
	\pgfplotsutil@advancestringcounter\pgfplots@loc@TMPa
	\pgfplotsarrayletentry{#1}\of{pgfp@hist}=\pgfplots@loc@TMPa
}%
\def\pgfplotsplothandlerhistgetintervalstartfor#1{%
	\pgfmathfloatparsenumber{#1}%
	\expandafter\pgfplotsplothandlerhistgetintervalstartfor@\expandafter{\pgfmathresult}%
}%
\def\pgfplotsplothandlerhistgetintervalstartfor@#1{%
	\pgfmathfloatmultiply@{\pgfplotsplothandlerhistogram@h}{#1}%
	\expandafter\pgfmathfloatadd@\expandafter{\pgfmathresult}{\pgfplotsplothandlerhistogram@datamin}%
}%
\def\pgfplotsplothandlerhistgetbinfor#1{%
	\pgfmathfloatparsenumber{#1}%
	\expandafter\pgfplotsplothandlerhistgetbinfor@\expandafter{\pgfmathresult}%
}%

\def\pgfplotsplothandlerhistsettol#1{%
	\begingroup
	\pgfmathfloatparsenumber{#1}%
	\global\let\pgfplotsplothandlerhisttol@parsed=\pgfmathresult
	\endgroup
}%
\pgfplotsplothandlerhistsettol{1e-4}%

\def\pgfplotsplothandlerhistgetbinfor@#1{%
	\pgfmathfloatsubtract@{#1}{\pgfplotsplothandlerhistogram@datamin}%
	\expandafter\pgfmathfloatmultiply@\expandafter{\pgfmathresult}{\pgfplotsplothandlerhistogram@invh}%
	\expandafter\pgfmathfloatadd@\expandafter{\pgfmathresult}{\pgfplotsplothandlerhisttol@parsed}%
	\pgfmathfloattofixed{\pgfmathresult}%
	\afterassignment\pgfplots@gobble@until@relax
	\c@pgf@counta=\pgfmathresult\relax
	\ifnum\pgfplotsplothandlerhistogram@Nfixed>\c@pgf@counta
		\ifnum\c@pgf@counta<0
			\def\pgfmathresult{0}%
		\else
			\def\pgfmathresult{\the\c@pgf@counta}%
		\fi
	\else
		\let\pgfmathresult=\pgfplotsplothandlerhistogram@Nmax
	\fi
}%


%%%%%%%%%%%%%%%%%%%%%%%%%%%%%%%%%%%%%%%%%%%%%%%%%%%%%%%%%%%%%5
%
% Contour plots
%
%%%%%%%%%%%%%%%%%%%%%%%%%%%%%%%%%%%%%%%%%%%%%%%%%%%%%%%%%%%%%5

\newif\ifpgfplotsplothandlercontour@labels
\newif\ifpgfplotsplothandlercontour@filled
\newif\ifpgfplotsplothandlercontour@corners

\def\pgfplotsplothandlercontour@axis@set@inverse#1#2#3#4{%
	\pgfplotsutilifstringequal{#1}{#4}{%
		\expandafter\def\csname pgfplotsplothandlercontour@axis@inv@#4\endcsname{x}%
	}{%
		\pgfplotsutilifstringequal{#2}{#4}{%
			\expandafter\def\csname pgfplotsplothandlercontour@axis@inv@#4\endcsname{y}%
		}{%
			\pgfplotsutilifstringequal{#3}{#4}{%
		\expandafter\def\csname pgfplotsplothandlercontour@axis@inv@#4\endcsname{z}%
			}{%
				\pgfplotsthrow{invalid argument}{\pgfplots@loc@TMPa}{Sorry, the choice axes={#1}{#2}{#3} is illegal. Please use only values x, y, and z and ensure that there is a 1:1 correspondence}\pgfeov%
			}%
		}%
	}%
}

\pgfplotsset{
	contour prepared/.code={%
		\let\tikz@plot@handler=\pgfplotsplothandlercontourprepared
		\pgfqkeys{/pgfplots/contour}{%
			every contour plot,%
			#1,%
		}%
	},
	contour prepared filled/.style={%
		/pgfplots/contour prepared={draw color=mapped color!80!black,labels=false,filled=true},
	},
	contour filled/.code={%
		\pgfqkeys{/pgfplots/contour}{%
			every filled contour plot,%
			/pgfplots/surf,%
			/pgfplots/shader=interp,%
			/pgfplots/colormap access=const,%
			#1,%
		}%
		\let\tikz@plot@handler=\pgfplotsplothandlercontourfilled
	},
	% FIXME : should use \tikz@plot@handler instead of just the
	% visualization thing!
	contour/handler/.style={/tikz/sharp plot},
	contour prepared format/.is choice,
	contour prepared format/standard/.code=	{\def\pgfplotsplothandlercontourprepared@format{s}},
	contour prepared format/matlab/.code=	{\def\pgfplotsplothandlercontourprepared@format{m}},
	contour prepared format/standard,
	contour/draw color/.initial=mapped color,
	contour/fill color/.initial=mapped color,
	contour/label distance/.initial=70pt,
	contour/label node code/.code={%
		\node {\pgfmathprintnumber{#1}};
	},%
	contour/labels/.is if=pgfplotsplothandlercontour@labels,
	contour/labels=true,
	%
	% XXX : this here is dead code... it has been superseded by
	% \pgfplotsplothandlercontourfilled:
	contour/filled/.is if=pgfplotsplothandlercontour@filled,
	contour/filled/.default=true,
	% EXPERIMENTAL: WILL CHANGE EVENTUALLY!
	% this is (only) used for contour filled:
	% data limits={(xmin,ymin,zmin,cmin) (xmax,ymax,ymax,cmax}
	contour/data limits/.initial=,% FIXME EXPERIMENTAL
	contour/every filled contour plot/.style={
		/pgfplots/area legend,
	},
	contour/every contour plot/.style={
		/pgfplots/legend image post style={sharp plot},
	},
	contour/every contour label/.style={%
		sloped,%
		transform shape,%
		inner sep=2pt,
		font=\scriptsize,
		every node/.style={mapped color!50!black,fill=white},%
		/pgf/number format/relative*={\pgfplotspointmetarangeexponent},%
		%every node/.style={yshift=10pt},%
	},
	contour/labels over line/.style={
		/pgfplots/contour/every contour label/.append style={%
			every node/.append style={%
				fill=none,
				anchor=base,
				yshift=1pt,
			},
		},
	},
	contour/contour label style/.style={
		/pgfplots/contour/every contour label/.append style={#1}},
	%
	contour/corners/.is if=pgfplotsplothandlercontour@corners,
	contour/corners=false,
	%
	%  Styles to actually *compute* the contour.
	%  These are mostly placeholders here: As long as the
	%  algorithm is not ready, we have to resort to external tools.
	%
	%  Shared parameters:
	contour/number/.initial=5,
	contour/levels/.initial=,
	contour/levels from colormap/.initial=,
	%
	%
	contour/contour dir/.is choice,
	contour/contour dir/x/.style={/pgfplots/contour/axes={y}{z}{x}},
	contour/contour dir/y/.style={/pgfplots/contour/axes={x}{z}{y}},
	contour/contour dir/z/.style={/pgfplots/contour/axes={x}{y}{z}},
	contour/axes/.code args={#1#2#3}{%
		\edef\pgfplotsplothandlercontour@axis@x{#1}%
		\edef\pgfplotsplothandlercontour@axis@y{#2}%
		\edef\pgfplotsplothandlercontour@axis@z{#3}%
		%
		\edef\pgfplots@loc@TMPa{{\pgfplotsplothandlercontour@axis@x}{\pgfplotsplothandlercontour@axis@y}{\pgfplotsplothandlercontour@axis@z}}%
		%
		\expandafter\pgfplotsplothandlercontour@axis@set@inverse\pgfplots@loc@TMPa{x}%
		\expandafter\pgfplotsplothandlercontour@axis@set@inverse\pgfplots@loc@TMPa{y}%
		\expandafter\pgfplotsplothandlercontour@axis@set@inverse\pgfplots@loc@TMPa{z}%
		%
	},%
	contour/contour dir=z,
	%
	%
	% Interface to external tools:
	contour external/.code={%
		\edef\tikz@plot@handler{\noexpand\pgfplotsplothandlercontourexternal}%
		\pgfqkeys{/pgfplots/contour external}{%
			every contour plot,%
			#1%
		}%
	},
	contour external/scanline marks/.is choice,
	contour external/scanline marks/false/.code={\def\pgfplotsplothandlercontourexternal@scanlinemode{0}},
	contour external/scanline marks/if in input/.code={\def\pgfplotsplothandlercontourexternal@scanlinemode{1}},
	contour external/scanline marks/required/.code={\def\pgfplotsplothandlercontourexternal@scanlinemode{2}},
	contour external/scanline marks/true/.code={\def\pgfplotsplothandlercontourexternal@scanlinemode{2}},
	contour external/scanline marks/if in input,
	contour external/output point meta/.initial=,
	contour external/file/.initial=,% auto-generate
	contour external/script extension/.initial=script,
	contour external/script/.initial=,% not yet initialised
	contour external/cmd/.initial=,% not yet initialised
	contour external/script type/.is choice=,%
	contour external/script type/system call/.code={\def\pgfplotsplothandlercontourexternal@script@type{S}},%
	contour external/script type/lua/.code={\def\pgfplotsplothandlercontourexternal@script@type{L}\pgfkeyssetvalue{/pgfplots/contour external/script extension}{lua}},%
	contour external/script type/system call,%
	contour external/script input format/.is choice,%
	contour external/script input format/x y meta meta/.code={\def\pgfplotsplothandlercontourexternal@script@input@format{M}},%
	contour external/script input format/x y z meta/.code={\def\pgfplotsplothandlercontourexternal@script@input@format{Z}},%
	contour external/script input format=x y meta meta,%
	contour external/@before cmd/.initial=,
	contour external/@after cmd/.initial=,
	contour external/.search also=/pgfplots/contour,
	contour gnuplot/.style={
		contour external={%
			scanline marks=required,
			script={
				unset surface;
				\ifx\thecontourlevels\empty
				set cntrparam levels \thecontournumber;
				\else
				set cntrparam levels discrete \thecontourlevels;
				\fi
				set contour;
				set table \"\outfile\";
				splot \"\infile\";
			},
			cmd={gnuplot \"\script\"},%
			#1,%
		},
	},
}%

\def\pgfplotsplothandlercontourprepared{%
	\pgfplotsresetplothandler
	\pgfplotsset{empty line=jump}%
	\let\pgf@plotstreamstart=\pgfplotsplothandlervisstart@contour
	\let\pgfplotsplothandlersurveystart=\pgfplotsplothandlersurveystart@contour
	\let\pgfplotsplothandlersurveyend=\pgfplotsplothandlersurveyend@contour
	\let\pgfplotsplothandlersurveypoint=\pgfplotsplothandlersurveypoint@contour
	\if m\pgfplotsplothandlercontourprepared@format
		\let\pgfplotsplothandlersurveystart=\pgfplotsplothandlersurveystart@contourmatlabformat
		\let\pgfplotsplothandlersurveypoint=\pgfplotsplothandlersurveypoint@contourmatlabformat
		\pgfplotsaxisifhaspointmeta{}{%
			\pgfplotsset{/pgfplots/point meta=explicit}%
		}%
	\fi
	\pgfplotsplothandlercontour@prepare@point@meta
	\def\pgfplotsplothandlersurveybeforesetpointmeta{%
		% ensure that 'point meta=z' respects 'axes={y}{z}{x}' -- it
		% should assign the 'x' coordinate as point meta!
		% To this end, we have to convert to reordered axes
		% temporarily.
		\pgfplotsplothandlersurveypoint@contour@axes@std@to@reordered
		\pgfplotsplothandlersurveybeforesetpointmeta@default
	}%
	\def\pgfplotsplothandlersurveyaftersetpointmeta{%
		\pgfplotsplothandlersurveyaftersetpointmeta@default
		% ... and undo the reordering after point meta has been set:
		\pgfplotsplothandlersurveypoint@contour@axes@reordered@to@std
	}%
	\def\pgfplotsplothandlername{contour prepared}%
}%

\def\pgfplotsplothandlersurveystart@contour{%
	\pgfplotsplothandlersurveystart@default
	%
	\pgfplotsplothandlersurveypoint@contour@prepare@axes x%
	\pgfplotsplothandlersurveypoint@contour@prepare@axes y%
	\pgfplotsplothandlersurveypoint@contour@prepare@axes z%
}%

\def\pgfplotsplothandlercontour@prepare@point@meta{%
	\ifpgfplots@curplot@threedim
		\pgfplotsset{/pgfplots/set point meta if empty=z}%
	\else
		\pgfplotsaxisifhaspointmeta{}{%
			\pgfkeysgetvalue{/pgfplots/table/meta index}\pgfplots@loc@TMPa
			\ifx\pgfplots@loc@TMPa\pgfutil@empty
% FIXME : this here is reasonable, but it should be sanity checked!
				\pgfkeyssetvalue{/pgfplots/table/meta index}{2}%
			\fi
			\pgfplotsset{/pgfplots/set point meta if empty=explicit}%
		}%
		%\pgfplots@error{Sorry, 'contour external' requires \string\addplot3 (or a non-empty `point meta' used as label data)}%
	\fi
	\def\pgfplotsplothandlercontour@empty@pointmeta@error@@{0}%
}%
\def\pgfplotsplothandlercontour@empty@pointmeta@error{%
	\if0\pgfplotsplothandlercontour@empty@pointmeta@error@@
		\def\pgfplotsplothandlercontour@empty@pointmeta@error@@{1}%
		\def\pgfplots@current@point@meta{0}%
		\pgfplotsthrow{invalid argument}{\pgfplots@current@point@meta}{Sorry, contour plots require non--empty `point meta'. Please use '\string\addplot3[contour ...] together with 'view={0}{90}' or provide a valid 'point meta=<value>' (for example `<value>=z'?)}\pgfeov%
	\fi
}%

\def\pgfplotsplothandlersurveystart@contourmatlabformat{%
	\def\c@pgfplotsplothandlercontourprepared@matlabformat@cur{0}%
	\def\c@pgfplotsplothandlercontourprepared@matlabformat@count{0}%
	\pgfmathfloatparsenumber{nan}%
	\let\pgfplotsplothandlercontourprepared@matlabformat@meta=\pgfmathresult
	%
	\pgfplotsplothandlersurveystart@contour
}
\def\pgfplotsplothandlersurveypoint@contourmatlabformat{%
	\ifnum\c@pgfplotsplothandlercontourprepared@matlabformat@cur=\c@pgfplotsplothandlercontourprepared@matlabformat@count\relax
		\pgfmathfloatparsenumber{\pgfplots@current@point@y}%
		\pgfmathfloattoint\pgfmathresult
		\let\c@pgfplotsplothandlercontourprepared@matlabformat@count=\pgfmathresult
		\def\c@pgfplotsplothandlercontourprepared@matlabformat@cur{0}%
		%
		\pgfmathfloatparsenumber{\pgfplots@current@point@x}%
		\let\pgfplots@loc@TMPa=\pgfplotsplothandlercontourprepared@matlabformat@meta
		\let\pgfplotsplothandlercontourprepared@matlabformat@meta=\pgfmathresult
		%
		\ifx\pgfplots@loc@TMPa\pgfplotsplothandlercontourprepared@matlabformat@meta
			% oh. We have two successive segments of the SAME contour
			% level. Call the jump handler:
			\pgfplotsscanlinecomplete
		\fi
		%
	\else
		\let\pgfplots@current@point@z=\pgfplotsplothandlercontourprepared@matlabformat@meta
		\let\pgfplots@current@point@meta=\pgfplotsplothandlercontourprepared@matlabformat@meta
		\pgfplotsplothandlersurveypoint@contour
		\pgfplotsutil@advancestringcounter\c@pgfplotsplothandlercontourprepared@matlabformat@cur
	\fi
}%

\def\pgfplotsplothandlersurveypoint@contour{%
	%
	\pgfplotsplothandlersurveypoint@contour@axes@reordered@to@std
	%
	\pgfplotsplothandlersurveypoint@default
}

\def\pgfplotsplothandlersurveyend@contour{%
	\pgfplotsplothandlercontour@init@limits
	\ifx\pgfplotsplothandlercontour@limits@low@meta\pgfutil@empty
	\else
		\pgfplotsaxisupdatelimitsforpointmeta\pgfplotsplothandlercontour@limits@low@meta
	\fi
}%

\def\pgfplotsplothandlervisstart@contour{%
	%
	\ifpgfplotsplothandlercontour@labels
		\pgfkeysgetvalue{/pgfplots/contour/label distance}\pgfplotsplothandlercontour@labeldist
		\pgfmathparse{\pgfplotsplothandlercontour@labeldist}%
		\edef\pgfplotsplothandlercontour@labeldist{\pgfmathresult pt}%
		%
		\pgfplotsapplistXnewempty\pgfplotsplothandlercontour@storedlabels
		\let\pgfplotsplothandlercontour@handlesplinesegment=\pgfplotsplothandlercontour@handlesplinesegment@forlabels
	\else
		\def\pgfplotsplothandlercontour@handlesplinesegment##1##2##3{}%
	\fi
	%
	\ifpgfplotsplothandlercontour@filled
		\pgfseteorule
		\let\pgfplotsplothandlercontour@sequence@sort@cmp=\pgfutil@empty
		\pgfplotsapplistXnewempty\pgfplotsplothandlercontour@stored@lastcontourpath
		\pgfplotsplothandlercontour@init@limits
	\fi
	%
	\pgfplotsresetplothandler
	\pgfplotsset{/pgfplots/contour/handler}%
	\tikz@plot@handler
	%
	\def\pgfplotsplothandlercontour@haspendingjump{0}%
	\let\pgfplotsplothandlervisualizejump=\pgfplotsplothandlercontour@jump
	%
	\let\pgfplotsplothandlercontour@handler@start=\pgf@plotstreamstart
	\let\pgfplotsplothandlercontour@lastcontour=\pgfutil@empty
	\global\let\pgf@plotstreampoint=\pgfplotsplothandlercontour@streampoint
	\global\let\pgf@plotstreamend=\pgfplotsplothandlercontour@streamend
}%

\def\pgfplotsplothandlercontour@jump{%
	\def\pgfplotsplothandlercontour@haspendingjump{1}%
}%

\def\pgfplotsplothandlercontour@streamend{%
	\ifx\pgfplotsplothandlercontour@lastcontour\pgfutil@empty
	\else
		%
		\def\pgfplotsplothandlercontour@haspendingjump{0}% important! (this is a different case than a jump)
		\pgfplotsplothandlercontour@finishcontourline
		%
		\ifpgfplotsplothandlercontour@filled
			\if0\pgfplotsplothandlercontour@processed@outer
				\pgfplotsplothandlercontour@stream@bbcontour
				\pgfplotsplothandlercontour@finishcontourline
			\fi
		\fi
		%
		\ifpgfplotsplothandlercontour@labels
			\scope[/pgfplots/contour/every contour label]
			\let\pgfplots@restore@drawmodes=\relax% FIXME : necessary?
			\pgfplotsapplistXlet\pgfplotsplothandlercontour@storedlabels@=\pgfplotsplothandlercontour@storedlabels
			\pgfplotsapplistXnewempty\pgfplotsplothandlercontour@storedlabels
			\expandafter\pgfplotsplothandlercontourplacelabels\pgfplotsplothandlercontour@storedlabels@\pgfplots@EOI
			\endscope
		\fi
	\fi
}%
\def\pgfplotsplothandlercontour@check@bbcontour{%
	\ifpgfplotsplothandlercontour@filled
		\if0\pgfplotsplothandlercontour@processed@outer
			\pgfplotscoordmath{meta}{if less than}
				{\pgfplotsplothandlercontour@act@contour}
				{\pgfplotsplothandlercontour@limits@low@meta}
				{%
					% TRUE!
					\pgfplotsplothandlercontour@stream@bbcontour@inbetween
				}{}%
		\fi
	\fi
}%
\def\pgfplotsplothandlercontour@stream@bbcontour@inbetween{%
	\def\pgfplotsplothandlercontour@processed@outer{1}%
}%

\def\pgfplotsplothandlercontour@stream@bbcontour{%
%\message{contour: processing OUTER data limit contour...}%
	\def\pgfplotsplothandlercontour@haspendingjump{0}% important
	%
	% forbid labels:
	\let\pgfplotsplothandlercontour@stream@bbcontour@oldlabelsetting=\pgfplotsplothandlercontour@handlesplinesegment
	\def\pgfplotsplothandlercontour@handlesplinesegment##1##2##3{}%
	%
	\let\pgfplots@current@point@meta=\pgfplotsplothandlercontour@limits@low@meta
	\pgfplotsplothandlercontour@streampoint{%
		\pgfplotsplothandlerpointxyz
			\pgfplotsplothandlercontour@limits@low@x
			\pgfplotsplothandlercontour@limits@low@y
			\pgfplotsplothandlercontour@limits@low@z
	}%
	\pgfplotsplothandlercontour@streampoint{%
		\pgfplotsplothandlerpointxyz
			\pgfplotsplothandlercontour@limits@high@x
			\pgfplotsplothandlercontour@limits@low@y
			\pgfplotsplothandlercontour@limits@low@z
	}%
	\pgfplotsplothandlercontour@streampoint{%
		\pgfplotsplothandlerpointxyz
			\pgfplotsplothandlercontour@limits@high@x
			\pgfplotsplothandlercontour@limits@high@y
			\pgfplotsplothandlercontour@limits@low@z
	}%
	\pgfplotsplothandlercontour@streampoint{%
		\pgfplotsplothandlerpointxyz
			\pgfplotsplothandlercontour@limits@low@x
			\pgfplotsplothandlercontour@limits@high@y
			\pgfplotsplothandlercontour@limits@low@z
	}%
	\pgfplotsplothandlercontour@streampoint{%
		\pgfplotsplothandlerpointxyz
			\pgfplotsplothandlercontour@limits@low@x
			\pgfplotsplothandlercontour@limits@low@y
			\pgfplotsplothandlercontour@limits@low@z
	}%
	\def\pgfplotsplothandlercontour@processed@outer{1}%
	%
	% restore labels:
	\let\pgfplotsplothandlercontour@handlesplinesegment=\pgfplotsplothandlercontour@stream@bbcontour@oldlabelsetting
}%
\def\pgfplotsplothandlercontourplacelabels#1{%
	\def\pgfplots@loc@TMPa{#1}%
	\ifx\pgfplots@loc@TMPa\pgfplots@EOI
	\else
		%
		\pgfplotsplothandlercontourplacelabels@act#1\relax
		%
		\expandafter\pgfplotsplothandlercontourplacelabels
	\fi
}%
\def\pgfplotsplothandlercontourplacelabels@act#1,#2;#3,#4;#5\relax{%
	\begingroup
	\def\pgfplots@current@point@meta{#5}%
	\pgfplotsaxisvisphasetransformpointmeta
	\pgfplotscolormapdefinemappedcolor{\pgfplotspointmetatransformed}%
	%
	\pgftransformlineattime{0.5}{\pgfqpoint{#1}{#2}}{\pgfqpoint{#3}{#4}}%
	\pgfkeysvalueof{/pgfplots/contour/label node code/.@cmd}#5\pgfeov
	\endgroup
}%

\def\pgfplotsplothandlercontour@streampoint#1{%
	\pgf@process{#1}%
	% remember point:
	\edef\pgfplotsplothandlercontour@act@canvas{\pgf@x=\the\pgf@x\space\pgf@y=\the\pgf@y\space}%
	%
	\let\pgfplotsplothandlercontour@act@contour=\pgfplots@current@point@meta
	\ifx\pgfplotsplothandlercontour@act@contour\pgfutil@empty
		% oh. No Z data!? That should not happen!
		\pgfplotsplothandlercontour@empty@pointmeta@error
	\fi
	\ifx\pgfplotsplothandlercontour@lastcontour\pgfutil@empty
		% oh. its the very first point.
		\def\pgfplotsplothandlercontour@haspendingjump{0}% important! (this is a different case than a jump)
		\pgfplotsplothandlercontour@preparenewcontourline
	\else
		\ifx\pgfplotsplothandlercontour@lastcontour\pgfplotsplothandlercontour@act@contour
			% belongs to the same contour.
			\if1\pgfplotsplothandlercontour@haspendingjump
				\pgfplotsplothandlercontour@finishcontourline
				\pgfplotsplothandlercontour@preparenewcontourline
			\fi
		\else
			% oh, a new contour line.
			\def\pgfplotsplothandlercontour@haspendingjump{0}% important! (this is a different case than a jump)
			\pgfplotsplothandlercontour@finishcontourline
			\pgfplotsplothandlercontour@check@bbcontour
			\pgfplotsplothandlercontour@preparenewcontourline
		\fi
	\fi
	%
	%
	% handle difference vector for label placement:
	\ifx\pgfplotsplothandlercontour@lastcanvas\pgfutil@empty
	\else
		\pgfplotsplothandlercontour@handlesplinesegment
			{\pgfplotsplothandlercontour@lastcanvas}
			{\pgfplotsplothandlercontour@act@canvas}%
			{\pgfplotsplothandlercontour@act@contour}%
	\fi
	%
	\pgfplotsplothandlercontour@handler@streampoint{\pgfplotsplothandlercontour@act@canvas}%
	\let\pgfplotsplothandlercontour@lastlastcanvas=\pgfplotsplothandlercontour@lastcanvas
	\let\pgfplotsplothandlercontour@lastcanvas=\pgfplotsplothandlercontour@act@canvas
	\let\pgfplotsplothandlercontour@lastcontour=\pgfplotsplothandlercontour@act@contour
	%
	\ifpgfplotsplothandlercontour@filled
		\t@pgfplots@toka=\expandafter{\expandafter{\pgfplotsplothandlercontour@act@canvas}}%
		\expandafter\pgfplotsapplistXpushback\expandafter\pgf@plotstreampoint\the\t@pgfplots@toka
			\to\pgfplotsplothandlercontour@stored@lastcontourpath
	\fi
	%
	\ifx\pgf@plotstreampoint\pgfplotsplothandlercontour@streampoint
	\else
		\let\pgfplotsplothandlercontour@handler@streampoint=\pgf@plotstreampoint
	\fi
	\global\let\pgf@plotstreampoint=\pgfplotsplothandlercontour@streampoint
	\global\let\pgf@plotstreamend=\pgfplotsplothandlercontour@streamend
	% ATTENTION : if the low level plot handler introduces extra
	% levels of scopes, this *will* fail!
}%

%
% #1 : source
% #2 : target
% #3 : contour value
\def\pgfplotsplothandlercontour@handlesplinesegment@forlabels#1#2#3{%
	#2%
	\pgf@xb=\pgf@x
	\pgf@yb=\pgf@y
	#1%
	\pgf@xc=\pgf@x
	\pgf@yc=\pgf@y
	\pgfpointdiff
		{\pgf@x=\pgf@xc \pgf@y=\pgf@yc}
		{\pgf@x=\pgf@xb \pgf@y=\pgf@yb}%
	\edef\pgfplots@loc@TMPa{{\pgf@sys@tonumber\pgf@x}{\pgf@sys@tonumber\pgf@y}}%
	\expandafter\pgfmath@basic@veclen@\pgfplots@loc@TMPa
	\pgf@xa=\pgfmathresult pt
	%
	\advance\pgf@xa by\pgfplotsplothandlercontour@len\relax
	\edef\pgfplotsplothandlercontour@len{\the\pgf@xa}%
%\message{contour(\pgfplotsplothandlercontour@act@contour): cur len=\pgfplotsplothandlercontour@len > \pgfplotsplothandlercontour@labeldist ? [segment from #1--#2];}%
	\ifdim\pgf@xa>\pgfplotsplothandlercontour@labeldist\relax
		\def\pgfplotsplothandlercontour@haslabel{1}%
		\edef\pgfplots@loc@TMPa{{\the\pgf@xc,\the\pgf@yc;\the\pgf@xb,\the\pgf@yb;{#3}}}%
		\expandafter\pgfplotsapplistXpushback\expandafter{\pgfplots@loc@TMPa}\to\pgfplotsplothandlercontour@storedlabels
		\advance\pgf@xa by-\pgfplotsplothandlercontour@labeldist\relax
		\edef\pgfplotsplothandlercontour@len{\the\pgf@xa\relax}%
	\fi
}%

% #1: the transformed point meta of the last contour.
\def\pgfplotsplothandlercontour@filled@assert@is@sorted#1{%
	\ifx\pgfplotsplothandlercontour@lastcontour\pgfutil@empty
	\else
		\ifx\pgfplotsplothandlercontour@sequence@sort@cmp\pgfutil@empty
			% we need to determine the sorting:
			\pgfplotscoordmath{float}{if less than}{\pgfplotsplothandlercontour@lastcontour}{\pgfplots@current@point@meta}{%
				\def\pgfplotsplothandlercontour@sequence@sort@cmp{+}%
			}{%
				\pgfplotscoordmath{float}{if less than}{\pgfplots@current@point@meta}{\pgfplotsplothandlercontour@lastcontour}{%
					\def\pgfplotsplothandlercontour@sequence@sort@cmp{-}%
				}{%
					% undecided - the contours have the same label.
				}%
			}%
		\else
			\if +\pgfplotsplothandlercontour@sequence@sort@cmp
				\pgfplotscoordmath{float}{if less than}{\pgfplots@current@point@meta}{\pgfplotsplothandlercontour@lastcontour}{%
					\pgfplotsplothandlercontour@filled@assert@is@sorted@fail{ASCENDING}%
				}{%
				}%
			\else
				\pgfplotscoordmath{float}{if less than}{\pgfplotsplothandlercontour@lastcontour}{\pgfplots@current@point@meta}{%
					\pgfplotsplothandlercontour@filled@assert@is@sorted@fail{DESCENDING}%
				}{%
				}%
			\fi
		\fi
	\fi
}%
\def\pgfplotsplothandlercontour@filled@assert@is@sorted@fail#1{%
	\begingroup
	\pgfplotscoordmath{float}{tostring}{\pgfplots@current@point@meta}%
	\let\offending=\pgfmathresult
	\pgfplotscoordmath{float}{tostring}{\pgfplotsplothandlercontour@lastcontour}%
	\let\lastcontour=\pgfmathresult
	\pgfplots@error{Sorry, filled contours implicitly assume that the contour levels are SORTED (in the given case, they appear to be sorted in #1 order). Please make sure your input data has sorted contour levels. The offending level is \offending; the previous contour level is \lastcontour}%
	\endgroup
}
\def\pgfplotsplothandlercontour@preparenewcontourline{%
	\if0\pgfplotsplothandlercontour@haspendingjump
%\message{PREPARE NEW CONTOUR LEVEL (\pgfplots@current@point@meta).}%
		% a completely new level has been
		% started, not due to jumps inside of one level.
		%
		\let\pgfplotspointmetatransformed@lastcontour=\pgfplotspointmetatransformed
		% thus, we need to compute 'mapped color':
		\pgfplotsaxisvisphasetransformpointmeta
		\pgfplotscolormapdefinemappedcolor{\pgfplotspointmetatransformed}%
		\pgfsetstrokecolor{\pgfkeysvalueof{/pgfplots/contour/draw color}}%
		\ifpgfplotsplothandlercontour@filled
			\pgfplotsplothandlercontour@filled@assert@is@sorted{\pgfplotspointmetatransformed@lastcontour}%
			\pgfsetfillcolor{\pgfkeysvalueof{/pgfplots/contour/fill color}}%
		\fi
		%
		% furthermore, we need to handle the 'filled' style:
		\ifpgfplotsplothandlercontour@filled
			\pgfplotsapplistXlet\pgfplotsplothandlercontour@stored@lastcontourpath@=\pgfplotsplothandlercontour@stored@lastcontourpath
			% flush:
			\pgfplotsapplistXnewempty\pgfplotsplothandlercontour@stored@lastcontourpath
			%
			% process it:
			\ifx\pgfplotsplothandlercontour@stored@lastcontourpath@\pgfutil@empty
				% oh. This here is the very first contour. Nothing to do!
			\else
				% ok. This here is the i'th contour level, i>0.
				% Thus, we have the 'i-1'th contour stored.
				% REPLICATE its path to fill the space between the (i-1)'th contour and the i'th one!
				\let\pgf@plotstreamstart=\pgfplotsplothandlercontour@handler@start
				\pgf@plotstreamstart
				\pgfplotsplothandlercontour@stored@lastcontourpath@
				\pgf@plotstreamend
			\fi
		\fi
	\else
%\message{PREPARE NEW CONTOUR for the already begun level \pgfplots@current@point@meta.}%
		\ifpgfplotsplothandlercontour@filled
			\pgfplotsapplistXpushback\pgf@plotstreamstart\to\pgfplotsplothandlercontour@stored@lastcontourpath
		\fi
	\fi
	%
	\let\pgfplotsplothandlercontour@lastcanvas=\pgfutil@empty
	\let\pgfplotsplothandlercontour@lastlastcanvas=\pgfutil@empty
	\ifpgfplotsplothandlercontour@labels
		\def\pgfplotsplothandlercontour@haslabel{0}%
		%
		% this here means that 20% of labeldist are already there.
		% it moves the first label nearer to its start.
		\pgf@xa=\pgfplotsplothandlercontour@labeldist\relax
		\pgf@xa=0.2\pgf@xa
		\edef\pgfplotsplothandlercontour@len{\the\pgf@xa}%
	\fi
	\def\pgfplotsplothandlercontour@haspendingjump{0}%
	%
	\pgfplotsplothandlercontour@handler@start
	\let\pgfplotsplothandlercontour@handler@end=\pgf@plotstreamend
	\let\pgfplotsplothandlercontour@handler@streampoint=\pgf@plotstreampoint
}%
\def\pgfplotsplothandlercontour@finishcontourline{%
	\ifpgfplotsplothandlercontour@labels
		\if0\pgfplotsplothandlercontour@haslabel
			\ifx\pgfplotsplothandlercontour@lastlastcanvas\pgfutil@empty
			\else
				% force a label:
%\message{FORCING A LABEL for contour \pgfplotsplothandlercontour@lastcontour\space from (\pgfplotsplothandlercontour@lastlastcanvas --\pgfplotsplothandlercontour@lastcanvas)}%
				\edef\pgfplotsplothandlercontour@len{\pgfplotsplothandlercontour@labeldist}%
				\pgfplotsplothandlercontour@handlesplinesegment
					{\pgfplotsplothandlercontour@lastlastcanvas}
					{\pgfplotsplothandlercontour@lastcanvas}%
					{\pgfplotsplothandlercontour@lastcontour}%
			\fi
		\fi
	\fi
	\pgfplotsplothandlercontour@handler@end
	%
%	\ifpgfplotsplothandlercontour@filled
%		\pgfpathclose % FIXME  this is, in general, not good enough.
%	\fi
	\if0\pgfplotsplothandlercontour@haspendingjump
%\message{usepath{} to finalize level \pgfplotsplothandlercontour@lastcontour.}%
		% flush paths if the complete contour level is ready.
		% do *not* flush paths if we have just a new part of the
		% existing contour level.
		%
		% This makes a difference for the filled contour.
		\ifpgfplotsplothandlercontour@filled
			\pgfusepath{fill,stroke}%
		\else
			\pgfusepath{stroke}%
		\fi
	\else
%\message{usepath{} for level \pgfplotsplothandlercontour@lastcontour.}%
		\ifpgfplotsplothandlercontour@filled
			\pgfplotsapplistXpushback\pgf@plotstreamend\to\pgfplotsplothandlercontour@stored@lastcontourpath
		\fi
	\fi
}%

\def\pgfplotsplothandlercontour@init@limits{%
	\def\pgfplotsplothandlercontour@processed@outer{0}%
	\let\pgfplotsplothandlercontour@limits@low@meta=\pgfutil@empty
	\let\pgfplotsplothandlercontour@limits@low@x=\pgfutil@empty
	\let\pgfplotsplothandlercontour@limits@low@y=\pgfutil@empty
	\let\pgfplotsplothandlercontour@limits@low@z=\pgfutil@empty
	\let\pgfplotsplothandlercontour@limits@high@meta=\pgfutil@empty
	\let\pgfplotsplothandlercontour@limits@high@x=\pgfutil@empty
	\let\pgfplotsplothandlercontour@limits@high@y=\pgfutil@empty
	\let\pgfplotsplothandlercontour@limits@high@z=\pgfutil@empty
	%
	\pgfkeysgetvalue{/pgfplots/contour/data limits}\pgfplots@loc@TMPa
	\ifx\pgfplots@loc@TMPa\pgfutil@empty
		\def\pgfplotsplothandlercontour@processed@outer{1}%
	\else
		\expandafter\pgfplotsplothandlercontour@init@limits@getboundingboxlow\pgfplots@loc@TMPa\pgfplots@EOI
	\fi
%\message{contour: I processed data limits to MIN = (\pgfplotsplothandlercontour@limits@low@x,\pgfplotsplothandlercontour@limits@low@y,\pgfplotsplothandlercontour@limits@low@z;\pgfplotsplothandlercontour@limits@low@meta) and MAX = (\pgfplotsplothandlercontour@limits@high@x,\pgfplotsplothandlercontour@limits@high@y,\pgfplotsplothandlercontour@limits@high@z;\pgfplotsplothandlercontour@limits@high@meta).}%
}%
\def\pgfplotsplothandlercontour@init@limits@getboundingboxlow(#1,#2,#3,#4){%
	\def\pgfplotsplothandlercontour@init@limits@cur{low}%
	\pgfplotsplothandlercontour@init@limits@read(#1,#2,#3,#4)%
	\pgfutil@ifnextchar(%
		\pgfplotsplothandlercontour@init@limits@getboundingboxhigh
		{\pgfplots@error{Sorry, 'data limits=(xmin,ymin,zmin,contourmin) (xmax,ymax,zmax,contourmax)' has been expected, not \pgfkeysvalueof{/pgfplots/contour/data limits}}\pgfplots@gobble@until@EOI}%
}%
\def\pgfplotsplothandlercontour@init@limits@getboundingboxhigh(#1,#2,#3,#4){%
	\def\pgfplotsplothandlercontour@init@limits@cur{high}%
	\pgfplotsplothandlercontour@init@limits@read(#1,#2,#3,#4)%
	\pgfutil@gobble
}%
\def\pgfplotsplothandlercontour@init@limits@read(#1,#2,#3,#4){%
	% FIXME : logs and trafos and user interface!
	\pgfplotscoordmath{x}{parsenumber}{#1}%
	\pgfplotscoordmath{x}{datascaletrafo}\pgfmathresult
	\expandafter\let\csname pgfplotsplothandlercontour@limits@\pgfplotsplothandlercontour@init@limits@cur @x\endcsname=\pgfmathresult
	%
	\pgfplotscoordmath{y}{parsenumber}{#2}%
	\pgfplotscoordmath{y}{datascaletrafo}\pgfmathresult
	\expandafter\let\csname pgfplotsplothandlercontour@limits@\pgfplotsplothandlercontour@init@limits@cur @y\endcsname=\pgfmathresult
	%
	\ifpgfplots@curplot@threedim
		\pgfplotscoordmath{z}{parsenumber}{#3}%
		\pgfplotscoordmath{z}{datascaletrafo}\pgfmathresult
		\expandafter\let\csname pgfplotsplothandlercontour@limits@\pgfplotsplothandlercontour@init@limits@cur @z\endcsname=\pgfmathresult
	\fi
	%
	\pgfplotscoordmath{meta}{parsenumber}{#4}%
	\expandafter\let\csname pgfplotsplothandlercontour@limits@\pgfplotsplothandlercontour@init@limits@cur @meta\endcsname=\pgfmathresult
}%

%%%%%%%%%%%%%%%%%%%%%%%%%%%%%%%%%%%%%%%%%%%%%%%%%%%%%%%%%%%%%%%%%%%%%%%%%%%%%%%
%
% contour external implementation
%
%%%%%%%%%%%%%%%%%%%%%%%%%%%%%%%%%%%%%%%%%%%%%%%%%%%%%%%%%%%%%%%%%%%%%%%%%%%%%%%

\def\pgfplotsplothandlercontourexternal{%
	\pgfplotsresetplothandler
	\def\pgfplotsplothandlername{contour external}%
	%
	\pgfplotsplothandlercontour@prepare@point@meta
	%
	\let\pgfplotsplothandlersurveystart=\pgfplotsplothandlersurveystart@contourexternal
	\def\pgf@plotstreamstart{%
		\pgfplotsplothandlercontourprepared
		\pgf@plotstreamstart
	}%
}%
\gdef\c@pgfplotsplothandlersurveystart@contourexternal@fileno{0}%
\def\pgfplotsplothandlersurveystart@contourexternal{%
	%
	\pgfkeysgetvalue{/pgfplots/contour external/file}\pgfplotsplothandlercontourexternal@file
	%
	\ifx\pgfplotsplothandlercontourexternal@file\pgfutil@empty
		\pgfutil@ifundefined{pgfactualjobname}{%
			\let\pgfplots@loc@TMPa=\jobname
		}{%
			% be compatible with external lib:
			\let\pgfplots@loc@TMPa=\pgfactualjobname
		}%
		\edef\pgfplotsplothandlercontourexternal@file{\pgfplots@loc@TMPa_contourtmp\c@pgfplotsplothandlersurveystart@contourexternal@fileno}%
		\pgfplotsutil@advancestringcounter@global\c@pgfplotsplothandlersurveystart@contourexternal@fileno
	\fi
	%
	\if0\pgfplotsplothandlercontourexternal@scanlinemode
		\pgfplotsset{plot to file/scanline marks/false}%
	\else
		\pgfplotsset{plot to file/scanline marks/if in input}% the choice 'always' is unaware of existing end-of-scanline marks
	\fi
	%
	\pgfplotsplothandlertofile{\pgfplotsplothandlercontourexternal@file.dat}%
	\pgfplotsplothandlersurveystart
	\let\pgfplotsplothandlersurveypoint@tofile=\pgfplotsplothandlersurveypoint
	\let\pgfplotsplothandlersurveyend@tofile=\pgfplotsplothandlersurveyend
	\let\pgfplotsplothandlersurveypoint=\pgfplotsplothandlersurveypoint@contourexternal
	\let\pgfplotsplothandlersurveyend=\pgfplotsplothandlersurveyend@contourexternal
	%
	\pgfplotsplothandlersurveypoint@contour@prepare@axes x%
	\pgfplotsplothandlersurveypoint@contour@prepare@axes y%
	\pgfplotsplothandlersurveypoint@contour@prepare@axes z%
	%
	\global\let\pgfplotsplothandlercontourexternal@metamin=\relax
	\global\let\pgfplotsplothandlercontourexternal@metamax=\relax
	%
}%

% #1: either x, y, or z
% POSTCONDITION:
% 	The following three macros are defined:
%   \pgfplotsplothandlersurveypoint@contour@set@axes@x
%   \pgfplotsplothandlersurveypoint@contour@set@axes@y
%   \pgfplotsplothandlersurveypoint@contour@set@axes@z
%   they take no arguments and assign their result to \pgfmathresult.
\def\pgfplotsplothandlersurveypoint@contour@prepare@axes#1{%
	\expandafter\edef\csname pgfplotsplothandlersurveypoint@contour@set@axes@#1\endcsname{%
		\noexpand\let\noexpand\pgfmathresult=\expandafter\noexpand\csname pgfplots@current@point@\csname pgfplotsplothandlercontour@axis@#1\endcsname\endcsname
	}%
	\expandafter\edef\csname pgfplotsplothandlersurveypoint@contour@set@axes@@#1\endcsname{%
		\noexpand\let\noexpand\pgfmathresult=\expandafter\noexpand\csname pgfplots@current@point@\csname pgfplotsplothandlercontour@axis@inv@#1\endcsname\endcsname
	}%
	%
%\message{axes #1: STD -> \expandafter\meaning\csname pgfplotsplothandlersurveypoint@contour@set@axes@#1\endcsname^^J}%
%\message{axes #1: INV -> \expandafter\meaning\csname pgfplotsplothandlersurveypoint@contour@set@axes@@#1\endcsname^^J}%
}

\def\pgfplotsplothandlersurveypoint@contour@axes@reordered@to@std{%
	\pgfplotsplothandlersurveypoint@contour@set@axes@@x
	\let\pgfplotsplothandlersurveypoint@contour@set@axes@x@=\pgfmathresult
	%
	\pgfplotsplothandlersurveypoint@contour@set@axes@@y
	\let\pgfplotsplothandlersurveypoint@contour@set@axes@y@=\pgfmathresult
	%
	\pgfplotsplothandlersurveypoint@contour@set@axes@@z
	\let\pgfplotsplothandlersurveypoint@contour@set@axes@z@=\pgfmathresult
	%
	\let\pgfplots@current@point@x=\pgfplotsplothandlersurveypoint@contour@set@axes@x@
	\let\pgfplots@current@point@y=\pgfplotsplothandlersurveypoint@contour@set@axes@y@
	\let\pgfplots@current@point@z=\pgfplotsplothandlersurveypoint@contour@set@axes@z@
}%
\def\pgfplotsplothandlersurveypoint@contour@axes@std@to@reordered{%
	\pgfplotsplothandlersurveypoint@contour@set@axes@x
	\let\pgfplotsplothandlersurveypoint@contour@set@axes@x@=\pgfmathresult
	%
	\pgfplotsplothandlersurveypoint@contour@set@axes@y
	\let\pgfplotsplothandlersurveypoint@contour@set@axes@y@=\pgfmathresult
	%
	\pgfplotsplothandlersurveypoint@contour@set@axes@z
	\let\pgfplotsplothandlersurveypoint@contour@set@axes@z@=\pgfmathresult
	%
	\let\pgfplots@current@point@x=\pgfplotsplothandlersurveypoint@contour@set@axes@x@
	\let\pgfplots@current@point@y=\pgfplotsplothandlersurveypoint@contour@set@axes@y@
	\let\pgfplots@current@point@z=\pgfplotsplothandlersurveypoint@contour@set@axes@z@
}%

\def\pgfplotsplothandlersurveypoint@contourexternal{%
	\ifx\pgfplotsplothandlercontourexternal@metamin\relax
		% remember values. We want to restore them right before we parse the output of contour external:
		\global\let\pgfplotsplothandlercontourexternal@metamin=\pgfplots@metamin
		\global\let\pgfplotsplothandlercontourexternal@metamax=\pgfplots@metamax
	\fi
	\pgfplotsplothandlersurveypoint@contour@axes@std@to@reordered
	%
	\pgfplotsaxissurveysetpointmeta
	%
	\if M\pgfplotsplothandlercontourexternal@script@input@format
		\let\pgfplots@current@point@z=\pgfplots@current@point@meta
	\fi
	%
	%
%\message{contour external collecting point (\pgfplots@current@point@x,\pgfplots@current@point@y,\pgfplots@current@point@z) [\pgfplots@current@point@meta]...}%
	\pgfplotsplothandlersurveypoint@tofile
}%

% Utility function which is NOT normally invoked. It can be invoked inside of 'contour external/@before cmd'. Should it become the default?
\def\pgfplotscontourpopulateallkeys{%
	\pgfkeysgetvalue{/pgfplots/contour/levels}\pgfplots@loc@TMPa
	\ifx\pgfplots@loc@TMPa\pgfutil@empty
		\ifx\pgfplots@metamax\pgfutil@empty
		\else
			\edef\pgfplots@loc@TMPa{\pgfplots@metamin:\pgfplots@metamax}%
			\edef\thecontournumber{\pgfkeysvalueof{/pgfplots/contour/number}}%
			\expandafter\pgfplots@domain@to@foreach\pgfplots@loc@TMPa\relax{\thecontournumber}%
			\expandafter\pgfplots@foreach@to@list@macro\expandafter{\pgfplotsretval}%
			\pgfkeyslet{/pgfplots/contour/levels}\pgfplotsretval
		\fi
	\fi
}%

{
\catcode`\`=12
\catcode`\'=12
\catcode`\"=12
\catcode`\|=12
\catcode`\;=12
\catcode`\:=12
\catcode`\#=12
\gdef\pgfplotsplothandlersurveyend@contourexternal{%
	\pgfplotsplothandlersurveyend@tofile
	\if2\pgfplotsplothandlercontourexternal@scanlinemode
		\if0\pgfplotsplothandlertofilegeneratedscanlinemarks
			\pgfplots@error{Sorry, processing the input stream did not lead to end-of-scanline markers; the generated temporary file for 'contour external' does not contain any of them (indicating that matrix structure is lost). To fix this, you have the following options:^^J - Insert end-of-scanline markers into your input data (i.e. empty lines),^^J - provide two of the three options 'mesh/rows=<num matrix rows>, mesh/cols=<num matrix cols>, mesh/num points=<total number points>'}%
		\fi
	\fi
	%
	\pgfkeysvalueof{/pgfplots/contour external/@before cmd}\relax%
	\begingroup
	\let\numcoords=\pgfplots@current@point@coordindex%
	\pgfplotsautocompletemeshkeys
	\def\"{"}%
	\def\|{|}%
	\def\;{;}%
	\def\:{:}%
	\def\#{#}%
	\def\'{'}%
	\def\`{`}%
	\edef\ordering{\pgfplots@plot@mesh@ordering}%
	\edef\infile{\pgfplotsplothandlercontourexternal@file.dat}%
	\edef\outfile{\pgfplotsplothandlercontourexternal@file.table}%
	%
	\pgfkeysgetvalue{/pgfplots/contour external/cmd}\cmd
	\pgfkeysgetvalue{/pgfplots/contour external/script extension}\scriptext
	\if S\pgfplotsplothandlercontourexternal@script@type
	\else
		% script type=lua
		\ifx\cmd\pgfutil@empty
			\pgfkeyssetvalue{/pgfplots/contour external/cmd}{require(\"\scriptbase\")}%
			\pgfkeysgetvalue{/pgfplots/contour external/cmd}\cmd
		\fi
	\fi
	%
	\pgfplotscoordmath{meta}{tofixed}{\pgfplots@metamin}%
	\let\pgfplotsmetamin=\pgfmathresult
	\pgfplotscoordmath{meta}{tofixed}{\pgfplots@metamax}%
	\let\pgfplotsmetamax=\pgfmathresult
	%
	\edef\script{\pgfplotsplothandlercontourexternal@file.\scriptext}%
	\edef\scriptbase{\pgfplotsplothandlercontourexternal@file}%
	\edef\thecontourlevels{\pgfkeysvalueof{/pgfplots/contour/levels}}%
	\edef\thecontournumber{\pgfkeysvalueof{/pgfplots/contour/number}}%
	\immediate\openout\w@pgf@writea=\script
	\immediate\write\w@pgf@writea{\pgfkeysvalueof{/pgfplots/contour external/script}}%
	\immediate\closeout\w@pgf@writea
	%
	\if S\pgfplotsplothandlercontourexternal@script@type
		\pgfplots@shellescape{\cmd}%
	\else
		\ifpgfplots@LUA@supported
			\pgfplotsutil@directlua{\pgfkeysvalueof{/pgfplots/contour external/cmd}}%
		\else
			\pgfplots@error{Calls to 'lua' are currently unsupported. Please invoke lualatex instead.}%
		\fi
	\fi
	\pgfkeysvalueof{/pgfplots/contour external/@after cmd}\relax%
	\endgroup
	%
	% RESET point meta computation here such that it can be collected from the 'contour prepared' stream (i.e. the output of contour external):
	\global\let\pgfplots@metamin=\pgfplotsplothandlercontourexternal@metamin
	\global\let\pgfplots@metamax=\pgfplotsplothandlercontourexternal@metamax
	%
	\pgfplotsplothandlercontourprepared
	% the PREPARE steps in the coord stream start/end have already
	% been done. we only need to init the plot handler survey
	% start/end:
	\def\pgfplots@coord@stream@start{\pgfplotsplothandlersurveystart}%
	\def\pgfplots@coord@stream@end{\pgfplotsplothandlersurveyend}%
	\pgfkeysgetvalue{/pgfplots/contour external/output point meta}\pgfplots@loc@TMPa
	\ifx\pgfplots@loc@TMPa\pgfutil@empty
		\ifpgfplots@curplot@threedim
			\pgfplotssetpointmetainput{z}{}%
		\else
			\pgfplotssetpointmetainput{explicit}{}%
		\fi
	\else
		\expandafter\pgfplots@pgfkeys@set@point@meta@value\expandafter{\pgfplots@loc@TMPa}%
	\fi
	\pgfplots@addplotimpl@file@streamit{\pgfplotsplothandlercontourexternal@file.table}%
	\closein\r@pgfplots@reada
}%
}





% the `plot unprocessed to file' plot handler simply copies the input data
% UNPROCESSED to an output file.
%
% I use it as helper tool for `contour gnuplot'.
%
% this cannot be combined with other plot handlers unless the other
% plot handlers controls that explicitly
%
% Immediately after the survey ends, this plot handler will set
% \pgfplotsplothandlertofilegeneratedscanlinemarks to '1' or '0'.
\pgfplotsset{
	plot unprocessed to file/.code={
		\edef\tikz@plot@handler{\noexpand\pgfplotsplothandlertofile{#1}}%
	},
	plot to file/scanline marks/.is choice,
	plot to file/scanline marks/false/.code={\def\pgfplotsplothandlertofile@scanlinemarks{0}},
	plot to file/scanline marks/if in input/.code={\def\pgfplotsplothandlertofile@scanlinemarks{1}},
	plot to file/scanline marks/always/.code={\def\pgfplotsplothandlertofile@scanlinemarks{2}},
	plot to file/scanline marks/true/.code={\def\pgfplotsplothandlertofile@scanlinemarks{2}},
	plot to file/scanline marks/if in input,
	plot to file/col sep/.initial=\pgfplots@SPACE,
	plot to file/end-of-scanline content/.initial=,
}

\def\pgfplotsplothandlertofile#1{%
	\edef\pgfplotsplothandlertofile@name{#1}%
	\pgfplotsresetplothandler
	\pgfplotsset{empty line=scanline}%
	\let\pgfplotsplothandlersurveystart=\pgfplotsplothandlersurveystart@tofile
	\def\pgfplotsplothandlername{plot to file}%
	\let\pgfplotsplothandlersurveyend=\pgfplotsplothandlersurveyend@tofile
	\let\pgfplotsplothandlersurveypoint=\pgfplotsplothandlersurveypoint@tofile
	\let\pgfplotsplothandlernotifyscanlinecomplete=\pgfplotsplothandlernotifyscanlinecomplete@tofile
	%
	\if1\pgfplotsplothandlertofile@scanlinemarks
		% plot to file/scanline marks=if in input
		% check if we have mesh input:
		\def\pgfplots@loc@TMP@has@mesh@information{0}%
		\pgfkeysgetvalue{/pgfplots/mesh/rows}\pgfplots@loc@TMPa
		\pgfkeysgetvalue{/pgfplots/mesh/cols}\pgfplots@loc@TMPb
		\pgfkeysgetvalue{/pgfplots/mesh/num points}\pgfplots@loc@TMPc
		\ifx\pgfplots@loc@TMPa\pgfutil@empty
			\ifx\pgfplots@loc@TMPc\pgfutil@empty
			\else
				% has rows and num points:
				\def\pgfplots@loc@TMP@has@mesh@information{1}%
			\fi
		\else
			\ifx\pgfplots@loc@TMPb\pgfutil@empty
				\ifx\pgfplots@loc@TMPc\pgfutil@empty
				\else
					% has cols and num points:
					\def\pgfplots@loc@TMP@has@mesh@information{1}%
				\fi
			\else
				% has rows,cols:
				\def\pgfplots@loc@TMP@has@mesh@information{1}%
			\fi
		\fi
		\if1\pgfplots@loc@TMP@has@mesh@information
			% activate 'plot to file/scanline marks=always
			% --> we have the required information!
			\def\pgfplotsplothandlertofile@scanlinemarks{2}%
		\fi
	\fi
	% NO \else here!
	\if2\pgfplotsplothandlertofile@scanlinemarks
		% plot to file/scanline marks=always
		\pgfplotsautocompletemeshkeys%
		\def\c@pgfplotsplothandlertofile@scanlinelength{0}%
		\if\pgfplots@plot@mesh@ordering0%
			% ordering = x varies= rowwise -> scanline is cols!
			\pgfkeysgetvalue{/pgfplots/mesh/cols}\c@pgfplotsplothandlertofile@expectedscanline
		\else
			% ordering = y varies = colwise: scanline is rows!
			\pgfkeysgetvalue{/pgfplots/mesh/rows}\c@pgfplotsplothandlertofile@expectedscanline
		\fi
	\fi
	\pgfplotsplothandlertofile@scanlinependingfalse
	\pgfkeysgetvalue{/pgfplots/plot to file/col sep}\pgfplotsplothandlertofile@colsep
	\pgfkeysgetvalue{/pgfplots/plot to file/end-of-scanline content}\pgfplotsplothandlertofile@scanlinemark
	%
	%
	\def\pgfplotsplothandlertofilegeneratedscanlinemarks{0}%
}%
\def\pgfplotsplothandlertofile@scanlinemarks@check{%
	\if2\pgfplotsplothandlertofile@scanlinemarks
		% plot to file/scanline marks=always
		\pgfplotsutil@advancestringcounter\c@pgfplotsplothandlertofile@scanlinelength
		\ifnum\c@pgfplotsplothandlertofile@scanlinelength=\c@pgfplotsplothandlertofile@expectedscanline
			\def\c@pgfplotsplothandlertofile@scanlinelength{0}%
			\pgfplotsplothandlernotifyscanlinecomplete@tofile
		\fi
	\fi
}%
\newif\ifpgfplotsplothandlertofile@scanlinepending
\def\pgfplotsplothandlernotifyscanlinecomplete@tofile{%
	\if0\pgfplotsplothandlertofile@scanlinemarks
	\else
		\pgfplotsplothandlertofile@scanlinependingtrue
	\fi
}%
\def\pgfplotsplothandlersurveypoint@tofile{%
	\ifpgfplotsplothandlertofile@scanlinepending
		\def\pgfplotsplothandlertofilegeneratedscanlinemarks{1}%
		\immediate\write\w@pgf@writea{\pgfplotsplothandlertofile@scanlinemark}% end-of-scanline
		\pgfplotsplothandlertofile@scanlinependingfalse
	\fi
	\pgfplotscoordmath{x}{parsenumber}{\pgfplots@current@point@x}%
	\pgfplotscoordmath{x}{tostring}{\pgfmathresult}%
	\let\pgfplots@current@point@x=\pgfmathresult
	%
	\pgfplotscoordmath{y}{parsenumber}{\pgfplots@current@point@y}%
	\pgfplotscoordmath{y}{tostring}{\pgfmathresult}%
	\let\pgfplots@current@point@y=\pgfmathresult
	%
	\ifpgfplots@curplot@threedim
		\pgfplotscoordmath{z}{parsenumber}{\pgfplots@current@point@z}%
		\pgfplotscoordmath{z}{tostring}{\pgfmathresult}%
		\let\pgfplots@current@point@z=\pgfmathresult
	\fi
	%
	\pgfplotsaxissurveysetpointmeta
	\pgfplotsaxisifhaspointmeta{%
		\pgfplotscoordmath{meta}{tostring}{\pgfplots@current@point@meta}%
		\let\pgfplots@current@point@meta=\pgfmathresult
	}{}%
	%
	\immediate\write\w@pgf@writea{%
		\pgfplots@current@point@x\pgfplotsplothandlertofile@colsep
		\pgfplots@current@point@y\pgfplotsplothandlertofile@colsep
		\ifpgfplots@curplot@threedim\pgfplots@current@point@z\pgfplotsplothandlertofile@colsep\fi
		\pgfplots@current@point@meta
	}%
	\advance\c@pgfplots@coordindex by1
	\pgfplotsplothandlertofile@scanlinemarks@check
}%
\def\pgfplotsplothandlersurveyend@tofile{%
	\immediate\closeout\w@pgf@writea
}%
\def\pgfplotsplothandlersurveystart@tofile{%
	\immediate\openout\w@pgf@writea=\pgfplotsplothandlertofile@name\relax
}%


\input pgfplotsmeshplothandler.code.tex
%
%--------------------------------------------


% Implementation of 'contour filled': it is actually just a small
% wrapper around 'surf,shader=interp,colormap access=const' -- with
% dedicated color map definitions.
\def\pgfplotsplothandlercontourfilled{%
	\pgfplotsplothandlermesh
	\let\pgfplotsplothandlersurveyend@contourfilled@orig=\pgfplotsplothandlersurveyend
	\def\pgfplotsplothandlersurveyend{%
		\pgfplotsplothandlersurveyend@contourfilled@orig
		\pgfplotsplothandlersurveyend@contourfilled
	}%
}%

\def\pgfplotscontourfilledcolormap{internal:contourfilled}
\pgfplotscreatecolormap{\pgfplotscontourfilledcolormap}{color=(black) color=(black)}

\def\pgfplotsplothandlersurveyend@contourfilled{%
	\pgfplotsplothandlersurveyaddoptions{
		% ensure that the point meta of the contour makes its way into
		% the colorbar and the associated color mapping.
		%
		% FIXME: that fails if there are more than one contours in the
		% same axis (or someone else uses 'colorbar source')
		colorbar source,%
		%
		% ensure that other plots to not "corrupt" our mapping:
		point meta rel=per plot,%
	}%
	\def\pgfplots@contour@cm@options{%
		colorbar style={%
			/pgfplots/colormap access=const,
		},%
	}%
	%
	\pgfkeysgetvalue{/pgfplots/contour/levels}\pgfplots@loc@TMPa
	\ifx\pgfplots@loc@TMPa\pgfutil@empty
		% no levels.
		% Ok, check the other keys in the order of precedence:
		\pgfkeysgetvalue{/pgfplots/contour/levels from colormap}\pgfplots@loc@TMPa
		\ifx\pgfplots@loc@TMPa\pgfutil@empty
			% no 'levels from colormap'.
			% Ok, check the other keys in the order of precedence:
			\pgfkeysgetvalue{/pgfplots/contour/number}\pgfplots@loc@TMPa
			\ifx\pgfplots@loc@TMPa\pgfutil@empty
				% not sure if this is a use-case...
				\pgfplots@log3{contour plot: skipping definition of suitable colormap (none of 'number', levels', 'levels from colormap' is set)}%
				\t@pgfplots@toka=\expandafter{\pgfplots@contour@cm@options}%
				\edef\pgfplots@contour@cm@options{%
					\the\t@pgfplots@toka%
					% just copy the current colormap to ensure that it
					% has the correct name:
					colormap={\pgfplotscontourfilledcolormap}{of colormap={}},%
				}%
			\else
				% Ah -- we have a number of samples!
				\t@pgfplots@toka=\expandafter{\pgfplots@contour@cm@options}%
				\edef\pgfplots@contour@cm@options{%
					\the\t@pgfplots@toka%
					colormap={\pgfplotscontourfilledcolormap}{samples of colormap={\pgfplots@loc@TMPa}},%
				}%
			\fi
		\else
			% Ah - we want to take levels from a colormap definition!
			% We only need to ensure that the colormap definition fits
			% into the range [metamin:metamax]. Do that here:
			\t@pgfplots@toka=\expandafter{\pgfplots@contour@cm@options}%
			\t@pgfplots@tokb=\expandafter{\pgfplots@loc@TMPa}%
			\edef\pgfplots@contour@cm@options{%
				\the\t@pgfplots@toka%
				/utils/exec={%
					\noexpand\edef\noexpand\pgfplots@targetpos@restore{%
						of colormap/target pos min=\pgfkeysvalueof{/pgfplots/of colormap/target pos min},%
						of colormap/target pos max=\pgfkeysvalueof{/pgfplots/of colormap/target pos max},%
						of colormap/target pos min/insert=\ifpgfplots@createcolormap@target@pos@includes@limits@min true\else false\fi,%
						of colormap/target pos max/insert=\ifpgfplots@createcolormap@target@pos@includes@limits@max true\else false\fi,%
						\if c\pgfplots@ofcolorbar@sample@for@choice
							of colormap/sample for=const,%
						\else
							of colormap/sample for=default,%
						\fi
					}%
				},%
				of colormap/target pos min*=\pgfplots@metamin,%
				of colormap/target pos max*=\pgfplots@metamax,%
				of colormap/sample for=const,%
				colormap={\pgfplotscontourfilledcolormap}{%
					\the\t@pgfplots@tokb
				},%
				/utils/exec={%
					\noexpand\expandafter
					\noexpand\pgfkeysalso
					\noexpand\expandafter{\noexpand\pgfplots@targetpos@restore}%
				},%
			}%
		\fi
	\else
		% Ah - we have a list of positions for the contours!
		\t@pgfplots@toka=\expandafter{\pgfplots@contour@cm@options}%
		\edef\pgfplots@contour@cm@options{%
			\the\t@pgfplots@toka%
			colormap={\pgfplotscontourfilledcolormap}{%
				of colormap={%
					target pos min*=\pgfplots@metamin,
					target pos max*=\pgfplots@metamax,
					sample for=const,%
					target pos={\pgfplots@loc@TMPa},
				}%
			},%
		}%
	\fi
	%
	\ifx\pgfplots@contour@cm@options\pgfutil@empty
	\else
		% move to "late options" such that we carry them outside of
		% any \begingroup...\endgroup constructions:
		\expandafter\pgfplotssetlateoptions\expandafter{\pgfplots@contour@cm@options}%
%\message{contour filled: assembled options \meaning\pgfplots@contour@cm@options^^J}%
	\fi
}%
