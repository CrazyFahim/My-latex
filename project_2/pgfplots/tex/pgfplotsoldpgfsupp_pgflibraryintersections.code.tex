%%%%%%%%%%%%%%%%%%%%%%%%%%%%%%%%%%%%%%%%%%%%%%%%%
%%% This file is a copy of some part of PGF/Tikz.
%%% It has been copied here to provide :
%%%  - compatibility with older PGF versions
%%%  - availability of PGF contributions by Christian Feuersaenger
%%%    which are necessary or helpful for pgfplots.
%%%
%%% For reasons of simplicity, I have copied the whole file, including own contributions AND
%%% PGF parts. The copyrights are as they appear in PGF.
%%%
%%% Note that pgfplots has compatible licenses.
%%%
%%% This copy has been modified in the following ways:
%%%  - nested \input commands have been updated
%%%
%
% Support for the contents of this file will NOT be done by the PGF/TikZ team.
% Please contact the author and/or maintainer of pgfplots (Christian Feuersaenger) if you need assistance in conjunction
% with the deployment of this patch or partial content of PGF. Note that the author and/or maintainer of pgfplots has no obligation to fix anything:
% This file comes without any warranty as the rest of pgfplots; there is no obligation for help.
%%%%%%%%%%%%%%%%%%%%%%%%%%%%%%%%%%%%%%%%%%%%%%%%%%
%%% Date of this copy: Di 14. Nov 22:41:45 CET 2017 %%%



% Copyright 2008 by Mark Wibrow
%
% This file may be distributed and/or modified
%
% 1. under the LaTeX Project Public License and/or
% 2. under the GNU Free Documentation License.
%
% See the file doc/generic/pgf/licenses/LICENSE for more details.

% Note: at the time of this writing, the library has quadratic runtime.
%   Experimentally, it performed well while computing ~12 intersections of two
%   plots, each with 600 samples. It failed when the number of samples exceeded 700.

\usepgflibrary{fpu}

\newcount\pgf@intersect@solutions

\newif\ifpgf@intersect@sort
\newif\ifpgf@intersect@sort@by@second@path

\def\pgfintersectionsortbyfirstpath{%
	\pgf@intersect@sorttrue%
	\pgf@intersect@sort@by@second@pathfalse%
}

\def\pgfintersectionsortbysecondpath{%
	\pgf@intersect@sorttrue%
	\pgf@intersect@sort@by@second@pathtrue%
}

% #1: the index. It starts at 1 and ends with \pgfintersectionsolutions (inclusive).
% Invalid values will implicitly result in the origin.
\def\pgfpointintersectionsolution#1{%
	\ifnum#1<1\relax%
		\pgfpoint@intersect@solution@orgin%
	\else%
		\ifnum#1>\pgfintersectionsolutions\relax%
			\pgfpoint@intersect@solution@orgin%
		\else%
			\csname pgfpoint@intersect@solution@#1\endcsname%
		\fi%
	\fi%
}

% Gets the segment indices of solution #1.
%
% #1: the solution index (i.e. the same argument as in \pgfpointintersectionsolution)
% #2: [output] a macro name which will contain the segment index of the first path which contains the solution
% #3: [output] a macro name which will contain the segment index of the second path which contains the solution
%
% Example: \pgfintersectiongetsolutionsegmentindices{0}{\first}{\second}
%
% -> \first may be 0 if point #0 is in the 0'th segment
% -> \second may be 42 if point #0 is in the 42'th segment 
%
% The "segment index" is actually close to the "time" of the solution.
% If a solution is at "time" 42.2, it will have segment index 42.
\def\pgfintersectiongetsolutionsegmentindices#1#2#3{%
	\ifnum#1<1\relax%
		\let#2=\pgfutil@empty
		\let#3=\pgfutil@empty
	\else%
		\ifnum#1>\pgfintersectionsolutions\relax%
			\let#2=\pgfutil@empty
			\let#3=\pgfutil@empty
		\else%
			\def\pgf@temp##1##2##3##4{%
				\edef#2{##1}%
				\edef#3{##2}%
			}%
			\expandafter\let\expandafter\pgf@tempb\csname pgf@intersect@solution@props@#1\endcsname
			\expandafter\pgf@temp\pgf@tempb
		\fi%
	\fi%
}%

% Gets the time indices of solution #1.
%
% #1: the solution index (i.e. the same argument as in \pgfpointintersectionsolution)
% #2: [output] a macro name which will contain the time of the first path which contains the solution
%   It will never be empty.
% #3: [output] a macro name which will contain the time of the second path which contains the solution
%   It will never be empty.
%
% Example: \pgfintersectiongetsolutiontimes{0}{\first}{\second}
%
% -> \first may be 0.5 if point #0 is in just in the middle of the path
% -> \second may be 42.8 if point #0 is in the 42'th segment (compare
%  \pgfintersectiongetsolutionsegmentindices) and is at 80% of the
%  42'th segment
%
% Note that the precise time inside of a segment may be unavailable
% (currently, it is only computed for curveto paths and not
% necessarily for lineto). If the precise time is unavailable, this
% call will return the value of
% \pgfintersectiongetsolutionsegmentindices (which is a
% "coarse-grained" time).
\def\pgfintersectiongetsolutiontimes#1#2#3{%
	\ifnum#1<1\relax%
		\let#2=\pgfutil@empty
		\let#3=\pgfutil@empty
	\else%
		\ifnum#1>\pgfintersectionsolutions\relax%
			\let#2=\pgfutil@empty
			\let#3=\pgfutil@empty
		\else%
			\def\pgf@temp##1##2##3##4{%
				\edef#2{##3}%
				\edef#3{##4}%
				% 
				% check for fallback to segment indices:
				\ifx#2\pgfutil@empty \edef#2{##1}\fi
				\ifx#3\pgfutil@empty \edef#3{##2}\fi
			}%
			\expandafter\let\expandafter\pgf@tempb\csname pgf@intersect@solution@props@#1\endcsname
			\expandafter\pgf@temp\pgf@tempb
		\fi%
	\fi%
}%

\def\pgfpoint@intersect@solution@orgin{%
	\begingroup%
		\pgftransforminvert%
		\pgfpointorigin%
		\pgf@pos@transform@glob
		\global\pgf@x=\pgf@x%
		\global\pgf@y=\pgf@y%
	\endgroup%
}

% #1 code which assigns the first path using \pgfsetpath.
% #2 code which assigns the second path using \pgfsetpath.
%
% On output, the points, their properties, and the number of points are set.
% Use \pgfintersectionsolutions which expands to the number of intersections
\long\def\pgfintersectionofpaths#1#2{%
	\begingroup%
		\pgfinterruptpath%
		#1% 
		\pgfgetpath\pgf@intersect@path@a%
		\global\let\pgf@intersect@path@temp=\pgf@intersect@path@a%
		\endpgfinterruptpath%
	\endgroup%
	\let\pgf@intersect@path@a=\pgf@intersect@path@temp%
	%
	\begingroup%
		\pgfinterruptpath%
		#2% 
		\pgfgetpath\pgf@intersect@path@b%
		\global\let\pgf@intersect@path@temp=\pgf@intersect@path@b%
		\endpgfinterruptpath%
	\endgroup%
	\let\pgf@intersect@path@b=\pgf@intersect@path@temp%
	%
	\pgf@intersect@solutions=0\relax%
	\pgf@intersect@path@reset@a
	%
	\ifpgf@intersect@sort@by@second@path%
		\let\pgf@intersect@temp=\pgf@intersect@path@a%
		\let\pgf@intersect@path@a=\pgf@intersect@path@b%
		\let\pgf@intersect@path@b=\pgf@intersect@temp%
	\fi%
	%
	\pgfprocessround\pgf@intersect@path@a\pgf@intersect@path@a%
	\pgfprocessround\pgf@intersect@path@b\pgf@intersect@path@b%
	%
	\let\pgf@intersect@token@after=\pgf@intersect@path@process@a%
	\expandafter\pgf@intersectionofpaths\pgf@intersect@path@a\pgf@stop%
	\edef\pgfintersectionsolutions{\the\pgf@intersect@solutions}%
	\pgfmathloop%
		\ifnum\pgfmathcounter>\pgfintersectionsolutions\relax%
		\else%
			\pgfutil@namelet{pgfpoint@intersect@solution@\pgfmathcounter}%
				{pgfpoint@g@intersect@solution@\pgfmathcounter}%
			\edef\pgf@marshal{\noexpand\pgf@intersection@set@properties{\csname pgfpoint@g@intersect@solution@\pgfmathcounter @props\endcsname}}%
			\pgf@marshal
			\ifpgf@intersect@sort%
				\pgfutil@namelet{pgf@intersect@solution@\pgfmathcounter @time@a}%
					{pgf@g@intersect@solution@\pgfmathcounter @time@a}%
			\fi%
	\repeatpgfmathloop%
	\ifpgf@intersect@sort%
		\pgfintersectionsolutionsortbytime%
	\fi%
}

\def\pgf@intersection@set@properties#1{%
	\pgfutil@namedef{pgf@intersect@solution@props@\pgfmathcounter}{#1}%
}%

% #1 a global name prefix to store properties.
\def\pgf@intersection@store@properties#1{%
	% we store the time offsets as well and make them available programmatically:
	% note that \pgf@intersect@time@a and \pgf@intersect@time@b may be empty.
	%
	% However, \pgf@intersect@time@offset and
	% \pgf@intersect@time@offset@b are *always* valid. In fact,they
	% resemble a part of the time: it holds 
	%   0 <= \pgf@intersect@time@a < 1
	% and \pgf@intersect@time@offset > 0. 
	%
	% If we have an intersection in segment 42 of path A,
	% \pgf@intersect@time@offset will be 42. The time inside of that
	% segment is given as number in the interval [0,1]. If it is 0.3,
	% the total time will be 42.3 and that number will be stored as
	% \pgf@intersect@time@a.
	%
	\expandafter\xdef\csname #1@props\endcsname{{\pgf@intersect@time@offset}{\pgf@intersect@time@offset@b}{\pgf@intersect@time@a}{\pgf@intersect@time@b}}%
}

\def\pgf@intersectionofpaths#1{%
	\ifx#1\pgf@stop%
		\let\pgf@intersect@next=\relax%
	\else%
		\ifx#1\pgfsyssoftpath@movetotoken%
			\let\pgf@intersect@next=\pgf@intersect@token@moveto%
		\else%
			\ifx#1\pgfsyssoftpath@linetotoken%
				\let\pgf@intersect@next=\pgf@intersect@token@lineto%
			\else%
				\ifx#1\pgfsyssoftpath@closepathtoken%
					\let\pgf@intersect@next=\pgf@intersect@token@lineto%
				\else%
					\ifx#1\pgfsyssoftpath@curvetosupportatoken%
						\let\pgf@intersect@next=\pgf@intersect@token@curveto%
					\else%
						\ifx#1\pgfsyssoftpath@rectcornertoken%
							\let\pgf@intersect@next=\pgf@intersect@token@rect%
						\fi%
					\fi%
				\fi%
			\fi%
		\fi%
	\fi%
	\pgf@intersect@next}

\def\pgf@intersect@token@moveto#1#2{%
	\def\pgfpoint@intersect@start{\pgfqpoint{#1}{#2}}%
	\pgf@intersectionofpaths%
}

\def\pgf@intersect@token@lineto#1#2{%
	\def\pgfpoint@intersect@end{\pgfqpoint{#1}{#2}}%
	\def\pgf@intersect@type{line}%
	\pgf@intersect@token@after%
}
\def\pgf@intersect@token@curveto#1#2\pgfsyssoftpath@curvetosupportbtoken#3#4\pgfsyssoftpath@curvetotoken#5#6{%
	\def\pgfpoint@intersect@firstsupport{\pgfqpoint{#1}{#2}}%
	\def\pgfpoint@intersect@secondsupport{\pgfqpoint{#3}{#4}}%
	\def\pgfpoint@intersect@end{\pgfqpoint{#5}{#6}}%
	\def\pgf@intersect@type{curve}%
	\pgf@intersect@token@after%
}

\def\pgf@intersect@token@rect#1#2\pgfsyssoftpath@rectsizetoken#3#4{%
	\pgf@xa=#1\relax%
	\advance\pgf@xa by#3\relax%
	\pgf@ya=#2\relax%
	\advance\pgf@ya by#4\relax%
	\edef\pgf@marshal{%
		\noexpand\pgfsyssoftpath@movetotoken{#1}{#2}%
		\noexpand\pgfsyssoftpath@linetotoken{#1}{\the\pgf@ya}%
		\noexpand\pgfsyssoftpath@linetotoken{\the\pgf@xa}{\the\pgf@ya}%
		\noexpand\pgfsyssoftpath@linetotoken{\the\pgf@xa}{#2}%
		\noexpand\pgfsyssoftpath@closepathtoken{#1}{#2}%
	}%
	\expandafter\pgf@intersectionofpaths\pgf@marshal%
}

\def\pgf@intersect@path@process@a{%
	\pgf@intersect@path@getpoints@a%
	\let\pgf@intersect@token@after=\pgf@intersect@path@process@b%
	\pgf@intersect@path@reset@b
	\expandafter\pgf@intersectionofpaths\pgf@intersect@path@b\pgf@stop%
	\let\pgfpoint@intersect@start=\pgfpoint@intersect@end@a%
	\let\pgf@intersect@token@after=\pgf@intersect@path@process@a%
	\c@pgf@counta=\pgf@intersect@time@offset\relax%
	\advance\c@pgf@counta by1\relax%
	\edef\pgf@intersect@time@offset{\the\c@pgf@counta}%
	\pgf@intersectionofpaths%
}

\def\pgf@intersect@path@reset@a{%
	\def\pgf@intersect@time@offset{0}%
	\def\pgf@intersect@time@a{}%
}%

\def\pgf@intersect@path@reset@b{%
	\def\pgf@intersect@time@offset@b{0}%
	\def\pgf@intersect@time@b{}%
}%

\def\pgf@intersect@path@getpoints@a{%
	\let\pgfpoint@intersect@start@a=\pgfpoint@intersect@start%
	\let\pgfpoint@intersect@end@a=\pgfpoint@intersect@end%
	\let\pgfpoint@intersect@firstsupport@a=\pgfpoint@intersect@firstsupport%
	\let\pgfpoint@intersect@secondsupport@a=\pgfpoint@intersect@secondsupport%
	\let\pgf@intersect@type@a=\pgf@intersect@type%
}

\def\pgf@intersect@path@process@b{%
	\pgf@intersect@path@getpoints@b%
	\csname pgf@intersect@\pgf@intersect@type@a @and@\pgf@intersect@type@b\endcsname%
	\let\pgfpoint@intersect@start=\pgfpoint@intersect@end@b%
	\c@pgf@counta=\pgf@intersect@time@offset@b\relax%
	\advance\c@pgf@counta by1\relax%
	\edef\pgf@intersect@time@offset@b{\the\c@pgf@counta}%
	\pgf@intersectionofpaths}

\def\pgf@intersect@path@getpoints@b{%
	\let\pgfpoint@intersect@start@b=\pgfpoint@intersect@start%
	\let\pgfpoint@intersect@end@b=\pgfpoint@intersect@end%
	\let\pgfpoint@intersect@firstsupport@b=\pgfpoint@intersect@firstsupport%
	\let\pgfpoint@intersect@secondsupport@b=\pgfpoint@intersect@secondsupport%
	\let\pgf@intersect@type@b=\pgf@intersect@type%
}

\def\pgf@intersect@line@and@line{%
	\pgf@intersectionoflines{\pgfpoint@intersect@start@a}{\pgfpoint@intersect@end@a}%
		{\pgfpoint@intersect@start@b}{\pgfpoint@intersect@end@b}%
}%

\def\pgf@intersect@line@and@curve{%
	\pgf@intersectionoflineandcurve%
		{\pgf@process{\pgfpoint@intersect@start@a}}{\pgf@process{\pgfpoint@intersect@end@a}}%
		{\pgf@process{\pgfpoint@intersect@start@b}}{\pgf@process{\pgfpoint@intersect@firstsupport@b}}%
		{\pgf@process{\pgfpoint@intersect@secondsupport@b}}{\pgf@process{\pgfpoint@intersect@end@b}}%
}

\def\pgf@intersect@curve@and@line{%
	\pgf@intersectionofcurveandline%
		{\pgf@process{\pgfpoint@intersect@start@a}}{\pgf@process{\pgfpoint@intersect@firstsupport@a}}%
		{\pgf@process{\pgfpoint@intersect@secondsupport@a}}{\pgf@process{\pgfpoint@intersect@end@a}}%
		{\pgf@process{\pgfpoint@intersect@start@b}}{\pgf@process{\pgfpoint@intersect@end@b}}%
}

\def\pgf@intersect@curve@and@curve{%
	\pgf@intersectionofcurves%
		{\pgf@process{\pgfpoint@intersect@start@a}}{\pgf@process{\pgfpoint@intersect@firstsupport@a}}%
		{\pgf@process{\pgfpoint@intersect@secondsupport@a}}{\pgf@process{\pgfpoint@intersect@end@a}}%
		{\pgf@process{\pgfpoint@intersect@start@b}}{\pgf@process{\pgfpoint@intersect@firstsupport@b}}%
		{\pgf@process{\pgfpoint@intersect@secondsupport@b}}{\pgf@process{\pgfpoint@intersect@end@b}}%
}


\def\pgfintersectionoflines#1#2#3#4{%
	\pgf@intersect@solutions=0\relax%
	\pgf@intersectionoflines{#1}{#2}{#3}{#4}%
}

\def\pgf@intersectionoflines#1#2#3#4{%
	\pgf@iflinesintersect{#1}{#2}{#3}{#4}%
	{%
		\pgfextract@process\pgf@intersect@solution@candidate{%
			% pgf@x and pgf@y are already assigned by \pgf@iflinesintersect
		}%
		\pgf@ifsolution@duplicate{\pgf@intersect@solution@candidate}{%
			% ah - we a duplicate. Apparently, we have a hit on an
			% endpoint.
		}{%
			\global\advance\pgf@intersect@solutions by1\relax%
			\expandafter\global\expandafter\let\csname pgfpoint@g@intersect@solution@\the\pgf@intersect@solutions\endcsname=\pgf@intersect@solution@candidate
			\ifpgf@intersect@sort%
				\pgf@xc=\pgf@x%
				\pgf@yc=\pgf@y%
				\pgf@process{\pgfpointdiff{\pgfpoint@intersect@start@a}{\pgfpoint@intersect@end@a}}%
				\edef\pgf@marshal{%
					\noexpand\pgfmathveclen@{\pgfmath@tonumber{\pgf@xa}}{\pgfmath@tonumber{\pgf@ya}}%
				}%
				\pgf@marshal%
				\let\pgf@intersect@length@a=\pgfmathresult%
				\pgf@process{\pgfpointdiff{\pgfpoint@intersect@start@a}{\pgfqpoint{\pgf@xc}{\pgf@yc}}}%
				\edef\pgf@marshal{%
					\noexpand\pgfmathveclen@{\pgfmath@tonumber{\pgf@x}}{\pgfmath@tonumber{\pgf@y}}%
				}%
				\pgf@marshal%
				\pgfmathdivide@{\pgfmathresult}{\pgf@intersect@length@a}%
				\pgf@x=\pgfmathresult pt\relax%
				\advance\pgf@x by\pgf@intersect@time@offset pt\relax%		
				\edef\pgf@intersect@time@a{\pgfmath@tonumber{\pgf@x}}%
				\expandafter\global\expandafter\let\csname pgf@g@intersect@solution@\the\pgf@intersect@solutions @time@a\endcsname=
					\pgf@intersect@time@a
			\else
				\let\pgf@intersect@time@a=\pgfutil@empty
			\fi%
			\let\pgf@intersect@time@b=\pgfutil@empty
			\pgf@intersection@store@properties{pgfpoint@g@intersect@solution@\the\pgf@intersect@solutions}%
		}%
		%
	}{%
	}%		
}

% Test if two lines L1 and L2 intersect.
%
% #1 - first point P1 on L1.
% #2 - second point P2 on L1.
% #3 - first point P3 on L2.
% #2 - second point P4 on L2.
% #5 - code executed if intersection occurs.
% #6 - code executed if intersection does no occur.
%
% Let L1 be represented by P1+(P2-P1)s where 0<=s<=1
% Let L2 be represented by P3+(P4-P3)t where 0<=t<=1
%
% Then L1 and L2 intersect at
%
% s = |x2-x1  x3-x1| / |x4-x3  x2-x1|
%     |y2-y1  y3-y1|   |y4-y3  y2-y1|
%
% t = |x4-x3  x3-x1| / |x4-x3  x2-x1|
%     |y4-y3  y3-y1|   |y4-y3  y2-y1|
% 
% with 0<=s,t<=1
%		
% s and t do not need to be calculated:
%
% Let s = A / C and t = B / C
% 
% Then 0<=s<=1 if !(C=0) && ((A=0) || ((A>0) && !(C<A)) || ((A<0) && !(C>A)))
%      0<=t<=1 if !(C=0) && ((B=0) || ((B>0) && !(C<B)) || ((B<0) && !(C>B)))
% 
\newif\ifpgf@s
\newif\ifpgf@t
\def\pgfiflinesintersect#1#2#3#4{%
	\begingroup%
		\pgf@iflinesintersect{\pgf@process{#1}}{\pgf@process{#2}}{\pgf@process{#3}}{\pgf@process{#4}}%
			{\aftergroup\pgfutil@firstoftwo}{\aftergroup\pgfutil@secondoftwo}%
	\endgroup%
}

% queried by pgfplots. Do not delete, only increase.
\def\pgf@intersections@version{2}%

% #1,#2: line 1
% #3,#4: line 2
\def\pgf@iflinesintersect#1#2#3#4{%
	% first: check bounding boxes -- but somewhat increased such that we do not
	% exclude "visible" hits due to rounding issues (i.e. use an upper bound):
	\pgf@intersect@boundingbox@reset%
	\pgf@intersect@boundingbox@update{#1}%
	\pgf@intersect@boundingbox@update{#2}%
	\pgf@intersect@boundingbox@assign@b%
	%
	\pgf@intersect@boundingbox@reset%
	\pgf@intersect@boundingbox@update{#3}%
	\pgf@intersect@boundingbox@update{#4}%
	\pgf@intersect@boundingbox@assign@a%
	%
	\pgf@intersect@boundingbox@a%
	\pgf@intersect@boundingbox@b%
	%
	\pgf@intersect@boundingbox@ifoverlap@upperbound{%
		\pgf@iflinesintersect@{#1}{#2}{#3}{#4}%
	}{%
		\let\pgf@intersect@next=\pgfutil@secondoftwo%
	}%
	\pgf@intersect@next%
}%

% a helper routine which simply defines \pgf@intersect@next.
% 
% In principle, this routine is capable of computing the entire intersection... but we only invoke it after checking for bounding box overlaps. This has two reasons:
%   1. robustness. almost-parallel lines could cause "dimension too large" when solving the linear equation system
% 		XXX : I still needed to replace the linear solver by one using the FPU. Perhaps I do not need the BB check anymore?
%   2. performance. I hope it is faster to first check for BB (but this is not sure in TeX)
%
% #1,#2: line 1
% #3,#4: line 2
\def\pgf@iflinesintersect@#1#2#3#4{%
	% we have two lines of sorts
	% l_1(s) := #1 + s * (#2 - #1),  0<= s <= 1
	% and
	% l_2(t) := #3 + t * (#4 - #3),  0<= t <= 1
	% ->
	% set up LGS
	% ( #2 - #1 ) *s + (#3-#4) * t = #3-#1
	% we have a hit if 0<= s,t <= 1 .
	#1\relax%
	\pgf@xa=\pgf@x%
	\pgf@ya=\pgf@y%
	#2\relax%
	\pgf@xb=\pgf@x%
	\pgf@yb=\pgf@y%
	#3\relax%
	\pgf@xc=\pgf@x%
	\pgf@yc=\pgf@y%
	#4\relax%
	%
	% will be overwritten, remember it:
	\edef\pgf@intersect@A{%
		\pgf@xa=\the\pgf@xa\space
		\pgf@ya=\the\pgf@ya\space
	}%
	%
	% B := (2-1)
	\advance\pgf@xb by-\pgf@xa
	\advance\pgf@yb by-\pgf@ya
	%
	% A := (3-1)
	\advance\pgf@xa by-\pgf@xc
	\advance\pgf@ya by-\pgf@yc
	\pgf@xa=-\pgf@xa
	\pgf@ya=-\pgf@ya
	%
	% C := (3-4)
	\advance\pgf@xc by-\pgf@x
	\advance\pgf@yc by-\pgf@y
	%
	\begingroup
	% compute the |.|_1 norm of each of lines. We need to compute
	% tolerance factors in order to decide if we have an intersection.
	% line 1: compute |#2 - #1|_1 :
	\ifdim\pgf@xb<0sp \pgf@xb=-\pgf@xb\fi
	\ifdim\pgf@yb<0sp \pgf@yb=-\pgf@yb\fi
	\advance\pgf@xb by\pgf@yb
	\xdef\pgf@intersect@len@a{\pgf@sys@tonumber\pgf@xb}%
	%
	% line 2: compute |#3 - #4|_1 :
	\ifdim\pgf@xc<0sp \pgf@xc=-\pgf@xc\fi
	\ifdim\pgf@yc<0sp \pgf@yc=-\pgf@yc\fi
	\advance\pgf@xc by\pgf@yc
	\xdef\pgf@intersect@len@b{\pgf@sys@tonumber\pgf@xc}%
	\endgroup
	%
	\edef\pgf@marshal{%
		\noexpand\pgfutilsolvetwotwoleqfloat{%
			{\pgf@sys@tonumber\pgf@xb}{\pgf@sys@tonumber\pgf@xc}%
			{\pgf@sys@tonumber\pgf@yb}{\pgf@sys@tonumber\pgf@yc}%
		}{%
			{\pgf@sys@tonumber\pgf@xa}%
			{\pgf@sys@tonumber\pgf@ya}%
		}%
	}%
	\pgf@marshal
	%
	\let\pgf@intersect@next=\pgfutil@secondoftwo%
	\ifx\pgfmathresult\pgfutil@empty
		% matrix was singular.
	\else
		\def\pgf@marshal##1##2{%
			\global\pgf@x=##1pt %
			\global\pgf@y=##2pt %
		}%
		\expandafter\pgf@marshal\pgfmathresult
		%
		\def\pgf@marshal{XXXX}% this should never be read
		% FIRST: check line 1:
		\ifdim\pgf@x<0sp
			% let it count as hit if
			%  || l_1(s) - l_1(0) || < eps
			% <=>  |s| * ||#2 - #1|| < eps
			% and, since s< 0  here:
			% <=>  -s * ||#2 - #1|| < eps
			\pgf@xa=-\pgf@intersect@len@a\pgf@x
			\ifdim\pgf@xa<\pgfintersectiontolerance\relax
				% close enough to first endpoint of line 1:
				\def\pgf@marshal{1}%
			\else
				\def\pgf@marshal{0}%
			\fi
		\else
			\ifdim\pgf@x>1pt
				% let it count as hit if
				%  || l_1(s) - l_1(1) || < eps
				% <=>  |s-1| * ||#2 - #1|| < eps
				% and, since s > 1  here:
				% <=>  s * ||#2 - #1|| - ||#2 - #1|| < eps
				\pgf@xa=\pgf@intersect@len@a\pgf@x
				\advance\pgf@xa by-\pgf@intersect@len@a pt %
				\ifdim\pgf@xa<\pgfintersectiontolerance\relax
					% close enough to second endpoint of line 1:
					\def\pgf@marshal{1}%
				\else
					\def\pgf@marshal{0}%
				\fi
			\else
				% 0<= s <= 1: we have an intersection within line 1.
				\def\pgf@marshal{1}%
			\fi
		\fi
		%
		% SECOND: check line 2:
		\if1\pgf@marshal
			\ifdim\pgf@y<0sp
				% see remarks for line 1. same applies here.
				\pgf@xa=-\pgf@intersect@len@b\pgf@y
				\ifdim\pgf@xa<\pgfintersectiontolerance\relax
					% close enough to first endpoint of line 2:
					\def\pgf@marshal{1}%
				\else
					\def\pgf@marshal{0}%
				\fi
			\else
				\ifdim\pgf@y>1pt
					% see remarks for line 1. same applies here.
					\pgf@xa=\pgf@intersect@len@b\pgf@y
					\advance\pgf@xa by-\pgf@intersect@len@b pt %
					\ifdim\pgf@xa<\pgfintersectiontolerance\relax
						% close enough to second endpoint of line 2:
						\def\pgf@marshal{1}%
					\else
						\def\pgf@marshal{0}%
					\fi
				\else
					% 0<= t <= 1: we have an intersection within line 2.
					\def\pgf@marshal{1}%
				\fi
			\fi
		\fi
		%
		\if1\pgf@marshal
			% Ok, compute the intersection point and return it:
			% we use (x,y) = A + s * (B-A)
			% keep in mind that (s,t) == (\pgf@x,\pgf@y)
			\pgf@intersect@A
			\pgf@yc=\pgf@x
			\global\pgf@x=\pgf@sys@tonumber\pgf@xb\pgf@yc
			\global\pgf@y=\pgf@sys@tonumber\pgf@yb\pgf@yc
			\global\advance\pgf@x by \pgf@xa
			\global\advance\pgf@y by \pgf@ya
			\let\pgf@intersect@next=\pgfutil@firstoftwo%
		\fi
	\fi
}




\def\pgfintersectionoflineandcurve#1#2#3#4#5#6{%
	\pgf@intersect@solutions=0\relax%
	\pgf@intersectionoflineandcurve{#1}{#2}{#3}{#4}{#5}{#6}%
}

\def\pgf@intersectionoflineandcurve#1#2#3#4#5#6{%
	\pgf@intersectionofcurves%
		{\pgf@process{#1}}%
		{%
			\pgf@process{%
				\pgfpointadd{#1\relax\pgf@x=0.666666\pgf@x\pgf@y=0.666666\pgf@y}%
					{#2\relax\pgf@x=0.333333\pgf@x\pgf@y=0.333333\pgf@y}%
			}%
		}%
		{%
			\pgf@process{%
				\pgfpointadd{#1\relax\pgf@x=0.333333\pgf@x\pgf@y=0.333333\pgf@y}%
					{#2\relax\pgf@x=0.666666\pgf@x\pgf@y=0.666666\pgf@y}%
			}%
		}%
		{\pgf@process{#2}}%
		{\pgf@process{#3}}%
		{\pgf@process{#4}}%
		{\pgf@process{#5}}%
		{\pgf@process{#6}}%
}%

\def\pgf@intersectionofcurveandline#1#2#3#4#5#6{%
	\pgf@intersectionofcurves%
		{\pgf@process{#1}}%
		{\pgf@process{#2}}%
		{\pgf@process{#3}}%
		{\pgf@process{#4}}%
		{\pgf@process{#5}}%
		{%
			\pgf@process{%
				\pgfpointadd{#5\relax\pgf@x=0.666666\pgf@x\pgf@y=0.666666\pgf@y}%
					{#6\relax\pgf@x=0.333333\pgf@x\pgf@y=0.333333\pgf@y}%
			}%
		}%
		{%
			\pgf@process{%
				\pgfpointadd{#5\relax\pgf@x=0.333333\pgf@x\pgf@y=0.333333\pgf@y}%
					{#6\relax\pgf@x=0.666666\pgf@x\pgf@y=0.666666\pgf@y}%
			}%
		}%
		{\pgf@process{#6}}%
}%




\def\pgfintersectiontolerance{0.1pt}
\def\pgfintersectiontoleranceupperbound{1pt}
\def\pgfintersectiontolerancefactor{0.1}



% Find the intersections of two bezier curves.
% 
% #1 - #4 = curve 1.
% #5 - #8 = curve 2.
% #9      = the solution number.
%
% There is no guarantee of ordering of solutions. If there are 
% no solutions, the origin is returned.
%
\def\pgfpointintersectionofcurves#1#2#3#4#5#6#7#8#9{%
	\pgf@intersect@solutions=0\relax%
	\pgf@intersectionofcurves%
		{\pgf@process{#1}}{\pgf@process{#2}}{\pgf@process{#3}}{\pgf@process{#4}}%
		{\pgf@process{#5}}{\pgf@process{#6}}{\pgf@process{#7}}{\pgf@process{#8}}%
	\pgfpointintersectionsolution{#9}%
}

% Return any intersection points of two curves C1 and C2.
% No order can be guaranteed for the solutions.
%
% #1, #2, #3, #4 - the points on C1 
% #5, #6, #7, #8 - the points on C2
%
% Returns:
%
% \pgf@intersect@solutions          - the number of solutions.
%	\pgfpointintersectionsolution{<S>} - the point for soultion S.
%
% (Sort of) use:
%
%	intersection(C1,C2)
%		S = {};
%	  intersection'(C1,C2);
%		return S;
%		
%	intersection'(C1,C2)
%		B1 = boundingbox(C1);
%		B2 = boundingbox(C2);
%		if intersect(B1,B2)
%			if (B1.width < q) and (B1.height < q) and
%        (B2.width < q) and (B2.height < q)
%				S = S + {average_of_all_points(B1,B2)}; \\ is there a better choice?
%			else
%				Q = subdivideLeft(C1);
%				R = subdivideRight(C1);
%				intersection'(C2,Q);
%				intersection'(C2,R);
%
% where q is a small value (tolerance).
%
\def\pgfintersectionofcurves#1#2#3#4#5#6#7#8{%
	\pgf@intersect@solutions=0\relax%
	\pgf@intersectionofcurves%
		{\pgf@process{#1}}{\pgf@process{#2}}{\pgf@process{#3}}{\pgf@process{#4}}%
		{\pgf@process{#5}}{\pgf@process{#6}}{\pgf@process{#7}}{\pgf@process{#8}}%
}%
\def\pgf@intersectionofcurves#1#2#3#4#5#6#7#8{%
	\begingroup%
		\dimendef\pgf@time@a=2\relax%
		\dimendef\pgf@time@aa=4\relax%
		\dimendef\pgf@time@b=6\relax%
		\dimendef\pgf@time@bb=8\relax%
		\pgf@time@a=0pt\relax%
		\pgf@time@aa=1pt\relax%
		\pgf@time@b=0pt\relax%
		\pgf@time@bb=1pt\relax%
		\let\pgf@intersect@subdivide@curve=\pgf@intersect@subdivide@curve@b%
		\let\pgf@curve@subdivde@after=\pgf@@intersectionofcurves%
		\pgf@@intersectionofcurves{#1}{#2}{#3}{#4}{#5}{#6}{#7}{#8}%
	\endgroup%
}

\def\pgf@intersect@boundingbox@assign@a{%
	\edef\pgf@intersect@boundingbox@a{%
		% lower left:
		\noexpand\pgf@xb=\the\pgf@xa\space%
		\noexpand\pgf@yb=\the\pgf@ya\space%
		% upper right:
		\noexpand\pgf@xc=\the\pgf@xb\space%
		\noexpand\pgf@yc=\the\pgf@yb\space%
	}%	
}
\def\pgf@intersect@boundingbox@assign@b{%
	\edef\pgf@intersect@boundingbox@b{%
		% lower left:
		\noexpand\global\noexpand\pgf@x=\the\pgf@xa\space% 
		\noexpand\global\noexpand\pgf@y=\the\pgf@ya\space%
		% upper right:
		\noexpand\pgf@xa=\the\pgf@xb\space%
		\noexpand\pgf@ya=\the\pgf@yb\space%
	}%
}

% see \pgf@intersect@boundingbox@assign@a and \pgf@intersect@boundingbox@assign@b for the naming conventions
\def\pgf@intersect@boundingbox@ifoverlap{%
	\def\pgf@intersect@next{\pgfutil@secondoftwo}%
	%
	\ifdim\pgf@xa<\pgf@xb%
	\else%
		\ifdim\pgf@x>\pgf@xc%
		\else%
			\ifdim\pgf@ya<\pgf@yb%
			\else%
				\ifdim\pgf@y>\pgf@yc%
				\else%
					\def\pgf@intersect@next{\pgfutil@firstoftwo}%
				\fi
			\fi
		\fi
	\fi
	\pgf@intersect@next
}%
\def\pgf@intersect@boundingbox@ifoverlap@upperbound{%
	\begingroup
	\def\pgf@intersect@next{\pgfutil@secondoftwo}%
	%
	\advance\pgf@xa by+\pgfintersectiontolerance\relax
	\ifdim\pgf@xa<\pgf@xb%
	\else%
		\global\advance\pgf@x by-\pgfintersectiontolerance\relax
		\ifdim\pgf@x>\pgf@xc%
		\else%
			\advance\pgf@ya by\pgfintersectiontolerance\relax
			\ifdim\pgf@ya<\pgf@yb%
			\else%
				\global\advance\pgf@y by-\pgfintersectiontolerance\relax
				\ifdim\pgf@y>\pgf@yc%
				\else%
					\def\pgf@intersect@next{\pgfutil@firstoftwo}%
				\fi
			\fi
		\fi
	\fi
	\expandafter
	\endgroup
	\pgf@intersect@next
}%
\def\pgf@intersect@boundingbox@ifoverlap@UNUSED{%
	\let\pgf@intersect@next=\pgfutil@secondoftwo%
	\ifdim\pgf@xa<\pgf@xb%
	\else%
		\ifdim\pgf@x>\pgf@xc%
		\else%
			\ifdim\pgf@ya<\pgf@yb%
			\else%
				\ifdim\pgf@y>\pgf@yc%
				\else%
					\let\pgf@intersect@next=\pgfutil@firstoftwo%
				\fi
			\fi
		\fi
	\fi
	\pgf@intersect@next
}%
\def\pgf@@intersectionofcurves#1#2#3#4#5#6#7#8{%
	\pgf@intersect@boundingbox@reset%
	\pgf@intersect@boundingbox@update{#1}%
	\pgf@intersect@boundingbox@update{#2}%
	\pgf@intersect@boundingbox@update{#3}%
	\pgf@intersect@boundingbox@update{#4}%
	\pgf@intersect@boundingbox@assign@b%
	%
	\pgf@intersect@boundingbox@reset%
	\pgf@intersect@boundingbox@update{#5}%
	\pgf@intersect@boundingbox@update{#6}%
	\pgf@intersect@boundingbox@update{#7}%
	\pgf@intersect@boundingbox@update{#8}%
	\pgf@intersect@boundingbox@assign@a%
	%
	\pgf@intersect@boundingbox@a%
	\pgf@intersect@boundingbox@b%
	%
	\pgf@intersect@boundingbox@ifoverlap{%
		\pgf@@@intersectionofcurves{#1}{#2}{#3}{#4}{#5}{#6}{#7}{#8}%
	}{%
		% no overlap -- no intersection.
	}%
}

\def\pgf@@@intersectionofcurves#1#2#3#4#5#6#7#8{%
					% compute DIFFERENCE vectors:
					\advance\pgf@xc by-\pgf@xb%
					\advance\pgf@yc by-\pgf@yb%
					\advance\pgf@xa by-\pgf@x%
					\advance\pgf@ya by-\pgf@y%
					\let\pgf@intersect@subdivde=\relax%
					% check if both difference vectors are point wise
					% less than tolerance (i.e. |v|_infty < eps ).
					% That means that both bounding boxes are "small enough"
					\ifdim\pgf@xc<\pgfintersectiontolerance\relax%
						\ifdim\pgf@xa<\pgfintersectiontolerance\relax%
							\ifdim\pgf@yc<\pgfintersectiontolerance\relax%
								\ifdim\pgf@ya<\pgfintersectiontolerance\relax%
									\pgfextract@process\pgf@intersect@solution@candidate{%
										% set (x,y) = mean(the 4 points of the two bounding boxes):
										\pgf@intersect@boundingbox@a%
										\pgf@intersect@boundingbox@b%
										\pgf@x=0.25\pgf@x%
										\advance\pgf@x by0.25\pgf@xa%
										\advance\pgf@x by0.25\pgf@xb%
										\advance\pgf@x by0.25\pgf@xc%
										\pgf@y=0.25\pgf@y%
										\advance\pgf@y by0.25\pgf@ya%
										\advance\pgf@y by0.25\pgf@yb%
										\advance\pgf@y by0.25\pgf@yc%
									}%
									% We must avoid duplicate solutions.
									\let\pgf@intersect@subdivde=\pgf@stop%
									\pgf@ifsolution@duplicate\pgf@intersect@solution@candidate{}%
									{%
										\global\advance\pgf@intersect@solutions by1\relax%
										\begingroup
										\advance\pgf@time@a by\pgf@time@aa%
										\divide\pgf@time@a by2\relax%
										\advance\pgf@time@a by\pgf@intersect@time@offset pt\relax%
										\edef\pgf@intersect@time@a{\pgfmath@tonumber{\pgf@time@a}}%
										%
										\advance\pgf@time@b by\pgf@time@bb%
										\divide\pgf@time@b by2\relax%
										\advance\pgf@time@b by\pgf@intersect@time@offset@b pt\relax%
										\edef\pgf@intersect@time@b{\pgfmath@tonumber{\pgf@time@b}}%
										%
										\pgf@intersection@store@properties{pgfpoint@g@intersect@solution@\the\pgf@intersect@solutions}%
										\expandafter\global\expandafter\let%
											\csname pgfpoint@g@intersect@solution@\the\pgf@intersect@solutions\endcsname=%
												\pgf@intersect@solution@candidate%
										\ifpgf@intersect@sort%
											\expandafter\xdef%
											\csname pgf@g@intersect@solution@\the\pgf@intersect@solutions @time@a\endcsname%	
												{\pgf@intersect@time@a}%						
										\fi%
										\endgroup
									}%
								\fi%
							\fi%
						\fi%
					\fi%
					\ifx\pgf@intersect@subdivde\pgf@stop%
					\else%
						\pgf@intersect@subdivide@curve{#1}{#2}{#3}{#4}{#5}{#6}{#7}{#8}%
					\fi%
}

\def\pgf@intersect@subdivide@curve@b#1#2#3#4#5#6#7#8{%
	\begingroup%
		\advance\pgf@time@bb by\pgf@time@b\relax%
		\divide\pgf@time@bb by2\relax%
		\let\pgf@intersect@subdivide@curve=\pgf@intersect@subdivide@curve@a%
		\pgf@curve@subdivide@left{#5}{#6}{#7}{#8}{#1}{#2}{#3}{#4}%
	\endgroup%
	\begingroup%
		\advance\pgf@time@b by\pgf@time@bb\relax%
		\divide\pgf@time@b by2\relax%
		\let\pgf@intersect@subdivide@curve=\pgf@intersect@subdivide@curve@a%
		\pgf@curve@subdivide@right{#5}{#6}{#7}{#8}{#1}{#2}{#3}{#4}%
	\endgroup%
}

\def\pgf@intersect@subdivide@curve@a#1#2#3#4#5#6#7#8{%
	\begingroup%
		\advance\pgf@time@aa by\pgf@time@a\relax%
		\divide\pgf@time@aa by2\relax%
		\let\pgf@intersect@subdivide@curve=\pgf@intersect@subdivide@curve@b%
		\pgf@curve@subdivide@left{#5}{#6}{#7}{#8}{#1}{#2}{#3}{#4}%
	\endgroup%
	\begingroup%
		\advance\pgf@time@a by\pgf@time@aa\relax%
		\divide\pgf@time@a by2\relax%
		\let\pgf@intersect@subdivide@curve=\pgf@intersect@subdivide@curve@b%
		\pgf@curve@subdivide@right{#5}{#6}{#7}{#8}{#1}{#2}{#3}{#4}%
	\endgroup%
}

\def\pgf@intersect@boundingbox@reset{%
	\pgf@xa=16000pt\relax%
	\pgf@ya=16000pt\relax%
	\pgf@xb=-16000pt\relax%
	\pgf@yb=-16000pt\relax%
}

\def\pgf@intersect@boundingbox@update#1{%
	#1\relax%
	\ifdim\pgf@x<\pgf@xa\pgf@xa=\pgf@x\fi%
	\ifdim\pgf@y<\pgf@ya\pgf@ya=\pgf@y\fi%
	\ifdim\pgf@x>\pgf@xb\pgf@xb=\pgf@x\fi%
	\ifdim\pgf@y>\pgf@yb\pgf@yb=\pgf@y\fi%
}

% The following subroutines are part of a conversion from pgfbasic
% math to FPU. This transition is necessary due to the restricted
% accuracy of pgfbasic. In order to limit the error rate of the
% transition pgfbasic -> FPU, I chose to
% keep the old "pattern" of sorts \advance\pgf@xa by0.5\pgf@y etc and
% simply adapt to some FPU call.
%
% The following routines constitute the "adapter":

\def\pgf@float@adapter@setxy{%
	\pgfmathfloatparsenumber{\pgf@sys@tonumber\pgf@x}\let\pgf@fpu@x=\pgfmathresult
	\pgfmathfloatparsenumber{\pgf@sys@tonumber\pgf@y}\let\pgf@fpu@y=\pgfmathresult
}%
\def\pgf@float@adapter@mult#1=#2*#3{%
	\pgfmathfloatmultiplyfixed@{#3}{#2}%
	\let#1=\pgfmathresult
}%
\def\pgf@float@adapter@advance#1by#2*#3{%
	\pgfmathfloatmultiplyfixed@{#3}{#2}%
	\let\pgfutil@temp=\pgfmathresult
	\pgfmathfloatadd@{#1}{\pgfutil@temp}%
	\let#1=\pgfmathresult
}%

\def\pgf@float@adapter@tostring#1{%
	\pgfmathfloattofixed{#1}\edef#1{\pgfmathresult pt }%
}%

\def\pgf@curve@subdivide@left#1#2#3#4{%
	%
	% The left curve (from t=0 to t=.5) 
	%
	\begingroup
	#1\relax%
	\pgfutil@tempdima=\pgf@x%
	\pgfutil@tempdimb=\pgf@y%
	\pgf@float@adapter@setxy
	\pgf@float@adapter@mult\pgf@fpu@xa=.5*\pgf@fpu@x  \pgf@float@adapter@mult\pgf@fpu@ya=.5*\pgf@fpu@y%
	\pgf@float@adapter@mult\pgf@fpu@xb=.25*\pgf@fpu@x \pgf@float@adapter@mult\pgf@fpu@yb=.25*\pgf@fpu@y%
	\pgf@float@adapter@mult\pgf@fpu@xc=.125*\pgf@fpu@x\pgf@float@adapter@mult\pgf@fpu@yc=.125*\pgf@fpu@y%
	#2\relax%
	\pgf@float@adapter@setxy
	\pgf@float@adapter@advance\pgf@fpu@xa by.5*\pgf@fpu@x\pgf@float@adapter@advance\pgf@fpu@ya by.5*\pgf@fpu@y%
	\pgf@float@adapter@advance\pgf@fpu@xb by.5*\pgf@fpu@x\pgf@float@adapter@advance\pgf@fpu@yb by.5*\pgf@fpu@y%
	\pgf@float@adapter@advance\pgf@fpu@xc by.375*\pgf@fpu@x\pgf@float@adapter@advance\pgf@fpu@yc by.375*\pgf@fpu@y%
	#3\relax%
	\pgf@float@adapter@setxy
	\pgf@float@adapter@advance\pgf@fpu@xb by.25*\pgf@fpu@x\pgf@float@adapter@advance\pgf@fpu@yb by.25*\pgf@fpu@y%
	\pgf@float@adapter@advance\pgf@fpu@xc by.375*\pgf@fpu@x\pgf@float@adapter@advance\pgf@fpu@yc by.375*\pgf@fpu@y%
	#4\relax%
	\pgf@float@adapter@setxy
	\pgf@float@adapter@advance\pgf@fpu@xc by.125*\pgf@fpu@x\pgf@float@adapter@advance\pgf@fpu@yc by.125*\pgf@fpu@y%
	%
	\pgf@float@adapter@tostring\pgf@fpu@xa
	\pgf@float@adapter@tostring\pgf@fpu@ya
	\pgf@float@adapter@tostring\pgf@fpu@xb
	\pgf@float@adapter@tostring\pgf@fpu@yb
	\pgf@float@adapter@tostring\pgf@fpu@xc
	\pgf@float@adapter@tostring\pgf@fpu@yc
	\edef\pgf@marshal{%
		\noexpand\pgf@curve@subdivde@after%
			{\noexpand\pgf@x=\the\pgfutil@tempdima\noexpand\pgf@y=\the\pgfutil@tempdimb}%
			{\noexpand\pgf@x=\pgf@fpu@xa\noexpand\pgf@y=\pgf@fpu@ya}%
			{\noexpand\pgf@x=\pgf@fpu@xb\noexpand\pgf@y=\pgf@fpu@yb}
			{\noexpand\pgf@x=\pgf@fpu@xc\noexpand\pgf@y=\pgf@fpu@yc}%
	}%
	\expandafter
	\endgroup
	\pgf@marshal%
}

\def\pgf@curve@subdivide@right#1#2#3#4{%
	%
	% The right curve (from t=0.5 to t=1) 
	%
	\begingroup
	#1\relax%
	\pgf@float@adapter@setxy
	\pgf@float@adapter@mult\pgf@float@tmpa=.125*\pgf@fpu@x\pgf@float@adapter@mult\pgf@float@tmpb=.125*\pgf@fpu@y%
	#2\relax%
	\pgf@float@adapter@setxy
	\pgf@float@adapter@advance\pgf@float@tmpa by.375*\pgf@fpu@x\pgf@float@adapter@advance\pgf@float@tmpb by.375*\pgf@fpu@y%
	\pgf@float@adapter@mult\pgf@fpu@xa=.25*\pgf@fpu@x\pgf@float@adapter@mult\pgf@fpu@ya=.25*\pgf@fpu@y%
	#3\relax%
	\pgf@float@adapter@setxy
	\pgf@float@adapter@advance\pgf@float@tmpa by.375*\pgf@fpu@x\pgf@float@adapter@advance\pgf@float@tmpb by.375*\pgf@fpu@y%
	\pgf@float@adapter@advance\pgf@fpu@xa by.5*\pgf@fpu@x\pgf@float@adapter@advance\pgf@fpu@ya by.5*\pgf@fpu@y%
	\pgf@float@adapter@mult\pgf@fpu@xb=.5*\pgf@fpu@x\pgf@float@adapter@mult\pgf@fpu@yb=.5*\pgf@fpu@y%
	#4\relax%
	\pgf@float@adapter@setxy
	\pgf@float@adapter@advance\pgf@float@tmpa by.125*\pgf@fpu@x\pgf@float@adapter@advance\pgf@float@tmpb by.125*\pgf@fpu@y%
	\pgf@float@adapter@advance\pgf@fpu@xa by.25*\pgf@fpu@x\pgf@float@adapter@advance\pgf@fpu@ya by.25*\pgf@fpu@y%
	\pgf@float@adapter@advance\pgf@fpu@xb by.5*\pgf@fpu@x\pgf@float@adapter@advance\pgf@fpu@yb by.5*\pgf@fpu@y%
	\let\pgf@fpu@xc=\pgf@fpu@x\let\pgf@fpu@yc=\pgf@fpu@y%
	%
	\pgf@float@adapter@tostring\pgf@float@tmpa
	\pgf@float@adapter@tostring\pgf@float@tmpb
	\pgf@float@adapter@tostring\pgf@fpu@xa
	\pgf@float@adapter@tostring\pgf@fpu@ya
	\pgf@float@adapter@tostring\pgf@fpu@xb
	\pgf@float@adapter@tostring\pgf@fpu@yb
	\pgf@float@adapter@tostring\pgf@fpu@xc
	\pgf@float@adapter@tostring\pgf@fpu@yc
	\edef\pgf@marshal{%
		\noexpand\pgf@curve@subdivde@after%
			{\noexpand\pgf@x=\pgf@float@tmpa\noexpand\pgf@y=\pgf@float@tmpb}%
			{\noexpand\pgf@x=\pgf@fpu@xa\noexpand\pgf@y=\pgf@fpu@ya}
			{\noexpand\pgf@x=\pgf@fpu@xb\noexpand\pgf@y=\pgf@fpu@yb}
			{\noexpand\pgf@x=\pgf@fpu@xc\noexpand\pgf@y=\pgf@fpu@yc}%
	}%
	\expandafter
	\endgroup
	\pgf@marshal%
}


% A solution S1 is considered a duplicate of S2, if
% 
% |x1 - x2|f < q and |y1 - y2|f < q
%
% where q is a small value (tolerance).
%
% #1 - the solution.
%
\def\pgf@ifsolution@duplicate#1{%
	#1%
	\pgf@xa=\pgf@x%
	\pgf@ya=\pgf@y%
	\let\pgf@intersect@next=\pgfutil@secondoftwo%
	\pgfmathloop%
		\ifnum\pgfmathcounter>\pgf@intersect@solutions\relax%
		\else%			
			\pgf@ifsolution@duplicate@{\pgfmathcounter}%
	\repeatpgfmathloop%
	\pgf@intersect@next%
}

\def\pgf@ifsolution@duplicate@#1{%
	\pgf@process{\csname pgfpoint@g@intersect@solution@#1\endcsname}%
	\advance\pgf@x by-\pgf@xa%
	\advance\pgf@y by-\pgf@ya%
	\ifdim\pgf@x<0pt\relax\pgf@x=-\pgf@x\fi%
	\ifdim\pgf@y<0pt\relax\pgf@y=-\pgf@y\fi%
	%
	\pgf@x=\pgfintersectiontolerancefactor\pgf@x%
	\pgf@y=\pgfintersectiontolerancefactor\pgf@y%
	\ifdim\pgf@x<\pgfintersectiontolerance\relax%
		\ifdim\pgf@y<\pgfintersectiontolerance\relax%
			\let\pgf@intersect@next=\pgfutil@firstoftwo%
		\fi%
	\fi%
}%

\newif\ifpgf@intersect@solutions@sortfinish

% Sort solutions according to their time index.
%
\def\pgfintersectionsolutionsortbytime{%
	\pgf@intersect@solutions@sortfinishtrue%
	\pgfmathloop%
		\ifnum\pgfmathcounter<\pgfintersectionsolutions\relax%
			\pgfutil@tempcnta=\pgfmathcounter%
			\advance\pgfutil@tempcnta by1\relax%
			\ifdim\csname pgf@intersect@solution@\pgfmathcounter @time@a\endcsname pt>%
				\csname pgf@intersect@solution@\the\pgfutil@tempcnta @time@a\endcsname pt\relax%
				\pgf@intersect@solutions@sortfinishfalse%
				%
				\pgfintersectionsolutionsortbytime@swap{pgfpoint@intersect@solution@\pgfmathcounter}%
					{pgfpoint@intersect@solution@\the\pgfutil@tempcnta}%
				%
				\pgfintersectionsolutionsortbytime@swap{pgf@intersect@solution@\pgfmathcounter @time@a}%
					{pgf@intersect@solution@\the\pgfutil@tempcnta @time@a}%
				%
				\pgfintersectionsolutionsortbytime@swap{pgf@intersect@solution@props@\pgfmathcounter}%
					{pgf@intersect@solution@props@\the\pgfutil@tempcnta}%
			\fi%
	\repeatpgfmathloop%
	\ifpgf@intersect@solutions@sortfinish%
	\else%
		\expandafter\pgfintersectionsolutionsortbytime%
	\fi%
}

\def\pgfintersectionsolutionsortbytime@swap#1#2{%
	\pgfutil@namelet{pgf@intersect@temp}{#1}%
	\pgfutil@namelet{#1}{#2}%
	\pgfutil@namelet{#2}{pgf@intersect@temp}%
}%

\endinput
