\documentclass[11pt]{article}
\usepackage[margin=1in]{geometry}
\usepackage{amsfonts, amsmath, amssymb}
\usepackage[none]{hyphenat}
\usepackage{fancyhdr}
\usepackage{graphicx}
\usepackage{float}
%\usepackage[nottoc, notlot, notlof]{tocbibind}
%\usepackage{pgfplots}

\setlength{\headheight}{13.59999pt}

\pagestyle{fancy}
\fancyhead{}
\fancyfoot{}
\fancyhead[L]{\slshape \MakeUppercase{Please Title Here}}
\fancyhead[R]{\slshape {Student Name}}
\fancyhead[C]{\slshape \thepage}
\renewcommand{\footrulewidth}{0pt}

\parindent 0ex
%\setlength{parindent}{4em}
%\setlength{\parskip}{1em}
\renewcommand{\baselinestretch}{1.5}

\begin{document}

\begin{titlepage}
    \begin{center}
        \vspace*{1cm}
        \Large{\textbf{IB Mathematics SL}}\\
        \Large{\textbf{Internal Assessment}}\\
        \vfill
        \line(1,0){400}\\[1mm]
        \huge{\textbf{This is a Sample Title}}\\[3mm]
        \Large{\textbf{- This is a Sample Subtitle -}}\\[1mm]
        \line(1,0){400}\\
        \vfill
        By: Student Name\\
        Candidate Number:\#\\
        \today\\
    \end{center}
\end{titlepage}

\tableofcontents
\thispagestyle{empty}
\clearpage

\setcounter{page}{1}

\section{Introduction}
The internally assessed component in these courses is a mathematical exploration. This is a short report written by the student based on a topic chosen by him or her, and it should focus on the mathematics of that particular area.The emphasis is on mathematical communication (including, formulae, diagrams, graphs and so on), with accompanying commentary , good mathematical writing and thoughtful reflection. A student should develop his or her  own focus , with the teacher providing feedback via, for example , discussiona and interview. This will allow all students to develop an area of interest for them , without a time constraint as in an examination , and will allow all to experience a feeling of sucess.


In addition to testing the objectives of the courses, the exploration is intended to provide students with opportunities 





The primary purpose of the exploration is for students to demonstrate the application of their skills and knowledge to solve a problem, and hence to use mathematics to explore a new area. The exploration should be assessed as a mathematical investigation that uses mathematical processes. It is expected that students will take approximately 10 hours of class time to complete the exploration. The exploration is internally assessed by the teacher and externally moderated by the IB using assessment criteria that relate to the objectives for mathematics HL and SL. The exploration is internally assessed by the teacher and externally moderated by the IB using assessment criteria that relate to the objectives for mathematics HL and SL. The exploration is internally assessed by the teacher and externally moderated by the IB using assessment criteria that relate to the objectives for mathematics HL and SL. The exploration is internally assessed by the teacher and externally moderated by the IB using assessment criteria that relate to the objectives for mathematics HL and SL. The exploration is internally assessed by the teacher and externally moderated by the IB using assessment criteria that relate to the objectives for mathematics HL and SL. The exploration is internally assessed by the teacher and externally moderated by the IB using assessment criteria that relate to the objectives for mathematics HL and SL. The exploration is internally assessed by the teacher and externally moderated by the IB using assessment criteria that relate to the objectives for mathematics HL and SL. 
\section{Sorting Criteria}

\subsection{Communication}


\subsection{Mathenatical Presentation}

\subsection{personal Engagement}

\section{Reflection}

\subsection{Use of Mathematics}

\section{Conclusion}

\section{Using \LaTeX}


\end{document} 

